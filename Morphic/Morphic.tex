% $Author: oscar $
% $Date: 2009-09-28 17:41:15 +0600 (пн, 28 сен 2009) $
% $Revision: 29309 $

% HISTORY:
% 2006-11-19 - Stef added French version of Hilaire's morphic article
% 2006-12-10 - Pollet translating
% 2007-08-16 - Oscar edit
% 2007-11-05 - Andrew edit
% 2009-07-07 - Oscar migrate to Pharo

%=================================================================
\ifx\wholebook\relax\else
% --------------------------------------------
% Lulu:
	\documentclass[a4paper,10pt,twoside]{book}
	\usepackage[
		papersize={6.13in,9.21in},
		hmargin={.75in,.75in},
		vmargin={.75in,1in},
		ignoreheadfoot
	]{geometry}
	\input{../common.tex}
	\pagestyle{headings}
	\setboolean{lulu}{true}
% --------------------------------------------
% A4:
%	\documentclass[a4paper,11pt,twoside]{book}
%	\input{../common.tex}
%	\usepackage{a4wide}
% --------------------------------------------
    \graphicspath{{figures/} {../figures/}}
	\begin{document}
	% \renewcommand{\nnbb}[2]{} % Disable editorial comments
	\sloppy
\fi
%=================================================================
\chapter{Morphic}

%\sd{We should first give a conceptual overview.
%Then we need a cookbook of how to do simple things in Morphic.
%The observer pattern and its implementation with changed:  and update: messages could go here.  Or in ``Idiomatic design patterns'' later.}

%\indmain{Morphic} is the name given to \pharo's graphical interface.
\indmain{Morphic} -- это название графического интерфейса \pharo.
%Morphic is written in \st, so it is fully portable between operating systems; as a consequence, \pharo looks exactly the same on Unix, MacOS and Windows.
Morphic написан на \st, и благодаря этому переносим между операционными системами. Как следствие, \pharo выглядит совершенно одинаково на Unix, MacOS, Windows.
%What distinguishes Morphic from most other user interface toolkits is that it does not have separate modes for ``composing'' and ``running'' the interface: all the graphical elements can be assembled and disassembled by the user, at any time.\footnote{We thank Hilaire Fernandes for permission to base this chapter on his original article in French.}
Отличие Morphic'а от большинства других фреймворков для построения GUI заключается в том, что в нём нет различия между режимами <<компоновки>> и <<запуска>> интерфейса. Все графические элементы могут быть собраны и разобраны пользователем в любое время.\footnote{Мы благодарим Hilaire Fernandes за разрешение использовать его статью на французском в качесте основы для этой главы (прим. авторов).}

\ab{After the first printing, I took an editing pass, correcting some errors and grammatical infelicities.}

\on{I have commented out the LabelstickerMorph and PyramidMorph examples, as they do not really add much over the other examples we have already. The source code is now available in the example subdirectory, in case someone would like to try and use them after all.}

%=================================================================
%\section{The history of Morphic}
\section{История Morpic'а}

%Morphic was developed by John Maloney and Randy Smith for the \ind{Self} programming language, staring around 1993. 
Morphic был разработан Джоном Мэлони (John Maloney) и Ренди Смитом (Randy Smith) для языка программирования \ind{Self} приблизительно в 1993 году.
%Maloney later wrote a new version of Morphic for \squeak, but the basic ideas behind the Self version are still alive and well in \pharo Morphic: \emph{directness} and \emph{liveness}.
Мэлони позже написал новую версию Morphic'а для \squeak, сохранив основные идеи из версии для Self: 
Directness means that the shapes on the screen are objects that can be examined or changed directly, that is, by pointing at them using the mouse.
Liveness means that the user interface is always able to respond to user actions: information on the screen is continuously updated as the world that it describes changes.
A simple example of this is that you can detach a menu item and keep it as a button.

\dothis{Bring up the world menu. \Metaclick once on the world menu to bring up its morphic halo\footnote{Recall that you should set \button{halosEnabled} in the Preferences browser. Alternatively, you can evaluate \ct{Preferences enable: \#halosEnabled} in a workspace.}, then \metaclick again on the menu item you want to detach to bring up its halo.  Now drag that item elsewhere on the screen by grabbing the black handle \grabHandle, as shown in \figref{detachingMenu}.}
\index{Morphic!halo}
\index{blue button}

\begin{figure}[ht]
	\centerline{\includegraphics[width=0.3\textwidth]{detachingMenu}}
	\caption{Detaching a morph, here the \menu{Workspace} menu item, to make it an independent button.
		\figlabel{detachingMenu}}
\end{figure}

All of the objects that you see on the screen when you run \pharo are \emph{Morphs}, that is, they are instances of subclasses of class \ct{Morph}.
\mbox{\ct{Morph}} itself is a large class with many methods; this makes it possible for subclasses to implement interesting behaviour with little code.
You can create a morph to represent any object, although how good a representation you get depends on the object!

\dothis{To create a morph to represent a string object, execute the following code in a workspace.} % , one line at a time.}
\begin{code}{}
'Morph' asMorph openInWorld
\end{code}
\cmindex{Morph}{openInWorld}

%\begin{code}{}
%s := 'Morph' asMorph openInWorld.
%s openViewerForArgument
%\end{code}
%\cmindex{Morph}{openInWorld}
% ON: openViewerForArgument is gone in pharo!

This creates a Morph to represent the string \ct{'Morph'}, and then opens it (that is, displays it) in the ``world'', which is the name that \pharo gives to the screen.  
You should obtain a graphical element\,---\,a Morph\,---\,which you can manipulate by \metaclick{ing}.
%The second line opens a ``viewer'' that shows you attributes of this Morph, such as its \ct{x} and \ct{y}  coordinates on the screen.  Clicking on one of the yellow exclamation marks sends a message to the Morph, which responds appropriately.

Of course, it is possible to define morphs that are more interesting graphical representations than the one that you have just seen.
The method \mthind{Object}{asMorph} has a default implementation in class \ct{Object} class that just creates a StringMorph.
So, for example, \ct{Color tan asMorph} returns a StringMorph labeled with the result of \clsind{Color} \ct{tan printString}.
Let's change this so that we get a coloured rectangle instead.

\dothis{Open a browser on the \ct{Color} class and add the following method to it:}
\needlines{3}
\begin{method}{Getting a morph for an instance of \ct{Color}.}
Color>>>asMorph
	^ Morph new color: self
\end{method}
\noindent
Now execute \ct{Color orange asMorph} \mthind{Morph}{openInWorld} in a workspace. Instead of the string-like morph, you get an orange rectangle!


%=================================================================
%\section{Manipulating morphs}
\section{Манипулирование морфами}

%Morphs are objects, so we can manipulate them like any other object in \st: by sending messages, we can change their properties, create new subclasses of Morph, and so on.
Морфы -- это обычные объекты, поэтому мы можем манипулировать ими также, как и любыми другими объектами в \st. Посылая сообщения, мы можем менять их свойства, создавать наследников класса Morph и т.д.

%Every morph, even if it is not currently open on the screen, has a position and a size.  
Каждый морф, даже если он в данный момент не отображается на экране, имеет позицию и размер.
%For convenience, all morphs are considered to occupy a rectangular region of the screen; if they are irregularly shaped, their position and size are those of the smallest rectangular ``box'' that surrounds them, which is known as the morph's bounding box, or just its ``bounds''.
Для удобства принимается, что любой морф занимает прямоугольную область на экране. Если морф имеет неправильную форму, его позиция и размер определяются наименьшим прямоугольником в который вписывается морф. Этот прямоугольник называется <<обрамляющим>> прямоугольником, или просто <<рамкой>>.
%The \mthind{Morph}{position} method returns a \ct{Point} that describes the location of the morph's upper left corner (or the upper left corner of its bounding box). 
Метод \mthind{Morph}{position} возвращает объект \ct{Point} (точка), определяющий положение левого верхнего угла морфа (или верхнего левого угла его рамки).
%The origin of the coordinate system is the screen's upper left corner, with $y$ coordinates increasing \emph{down} the screen and $x$ coordinates increasing to the right.
Начало системы координат -- это левый верхний угол экрана. Координата $y$ увеличивает при движении по экрану сверху-вниз, а координата $x$ -- при движении слева-направо.
%The \ct{extent} method also returns a point, but this point specifies the width and height of the morph rather than a location.
Метод \ct{extent} также возвращает точку, но эта точка обозначает ширину и высоту морфа, а не его местоположение.

%\dothis{Type the following code into a workspace and \menu{do it}:}
\dothis{Введите следующий код в рабочую область и выполните его (\menu{do it}):}
\begin{code}{}
joe := Morph new color: Color blue.
joe openInWorld.
bill := Morph new color: Color red .
bill openInWorld.
\end{code}
\noindent
%Then type \ct{joe position} and \menu{print it}.
Затем введите \ct{joe position} и напечатайте результат (\menu{print it}).
%To move joe, execute \ct{joe position: (joe position + (10@3))} repeatedly.
Чтобы переместить Джо, выполните \ct{joe position: (joe position + (10@3))} несколько раз.

%It is possible to do a similar thing with size.
То же самое можно проделать и с размером.
%\ct{joe} \mthind{Morph}{extent} answers joe's size; to have joe grow, execute \ct{joe extent: (joe extent * 1.1)}.
\ct{joe} \mthind{Morph}{extent} возвращает размер Джо; чтобы заставить Джо расти, выполните \ct{joe extent: (joe extent * 1.1)}.
%To change the color of a morph, send it the \mthind{Morph}{color:} message with the desired \ct{Color} object as argument, for instance, \ct{joe color: Color orange}.
Чтобы изменить цвет морфа, пошлите сообщение \mthind{Morph}{color:} с желаемым цветом в качестве аргумента. Например, \ct{joe color: Color orange}.
%To add transparency, try \ct{joe color: (Color orange alpha: 0.5)}.
Чтобы добавить прозрачности, попробуйте \ct{joe color: (Color orange alpha: 0.5)}.

%\dothis{To make bill follow joe, you can repeatedly execute this code:}
\dothis{Чтобы заставить Билла следовать за Джо, выполните несколько раз следующий код:}
\begin{code}{}
bill position: (joe position + (100@0))
\end{code}
\noindent
%If you move joe using the mouse and then execute this code, bill will move so that it is 100 pixels to the right of joe.
Если вы переместите Джо с помощью мыши, а потом выполните этот код, Билл передвинется в точку справа от Джо, отстоящую от него на 100 пикселей. 
\ab{It would seem that this would be a good place to introduce the \ct{step} method}.

%=================================================================
\section{Composing morphs}

One way of creating new graphical representations is by placing one morph inside another.
This is called \emph{composition}; morphs can be composed to any depth.
%
%To create new morphs, there are two main techniques that you can combine:
%\begin{enumerate}
%	\item by composing morphs one into another,
%	\item by subclassing \ct{Morph} and overriding \mthind{Morph}{drawOn:} to draw original morph shapes.
%\end{enumerate}
%}
\index{Morph!composing}
You can place a morph inside another by sending the message \mthind{Morph}{addMorph:} to the container morph.

\dothis{Try adding a morph to another one:}
\begin{code}{}
star := StarMorph new color: Color yellow.
joe addMorph: star.
star position: joe position.
\end{code}

\noindent
The last line positions the star at the same coordinates as joe.
Notice that the coordinates of the contained morph are still relative to the screen, not to the containing morph.
There are many  methods available to position a morph; browse the \protind{geometry} protocol of class \ct{Morph} to see for yourself.
For example, 
to center the star inside joe, execute \ct{star} \mthind{Morph}{center:} \ct{joe center}.

\begin{figure}[ht]
	\centerline{\includegraphics{joeStar}}
	\caption{The star is contained inside joe, the translucent blue morph.
		\figlabel{joeStar}}
\end{figure}

If you now try to grab the star with the mouse, you will find that you actually grab joe, and the two morphs move together: the star is \emph{embedded} inside joe.
It is possible to embed more morphs inside joe.  
In addition to doing this programmatically, you can also embed morphs by direct manipulation.

%=================================================================
\section{Creating and drawing your own morphs}

While it is possible to make many interesting and useful graphical representations by composing morphs, sometimes you will need to create something completely different.
\index{Morph!subclassing}
To do this you define a subclass of \ct{Morph} and override the \mthind{Morph}{drawOn:} method to change its appearance.

The morphic framework sends the message \ct{drawOn:} to a morph when it needs to redisplay the morph on the screen.  The parameter to \ct{drawOn:}  is a kind of \clsind{Canvas}; the expected behaviour is that the morph will draw itself on that canvas, inside its bounds.
Let's use this knowledge to create a cross-shaped morph.
\index{Morph!subclassing}

\dothis{Using the browser, define a new class \clsind{CrossMorph} inheriting from \ct{Morph}:}
\begin{classdef}{Defining \ct{CrossMorph}}
Morph subclass: #CrossMorph
	instanceVariableNames: ''
	classVariableNames: ''
	poolDictionaries: ''
	category: 'PBE-Morphic'
\end{classdef}

We can define the \ct{drawOn:} method like this:
\begin{method}[firstDrawOn]{Drawing a \ct{CrossMorph}.}
drawOn: aCanvas 
	| crossHeight crossWidth horizontalBar verticalBar |
	crossHeight := self height / 3.0 .
	crossWidth := self width / 3.0 .
	horizontalBar := self bounds insetBy: 0 @ crossHeight.
	verticalBar := self bounds insetBy: crossWidth @ 0.
	aCanvas fillRectangle: horizontalBar color: self color.
	aCanvas fillRectangle: verticalBar color: self color
\end{method}


\begin{figure}[hbt]
	\ifluluelse
		{\centerline{\includegraphics[width=0.3\textwidth]{NewCross}}}
		{\centerline{\includegraphics{NewCross}}}
	\caption{A \ct{CrossMorph} with its halo; you can resize it as you wish.
		\figlabel{cross}}
\end{figure}


Sending the \mthind{Morph}{bounds} message to a morph answers its bounding box, which is an instance of \clsind{Rectangle}.  Rectangles understand many messages that create other rectangles of related geometry; here we use the \ct{insetBy:} message with a point as its argument to create first a rectangle with reduced height, and then another rectangle with reduced width.

\dothis{To test your new morph, execute \ct{CrossMorph new} \mthind{Morph}{openInWorld}.}
The result should look something like \figref{cross}.
However, you will notice that the sensitive zone\,---\,where you can click to grab the morph\,---\,is still the whole bounding box.  Let's fix this.

When the Morphic framework needs to find out which Morphs lie under the cursor, it sends the message \ct{containsPoint:} to all the morphs whose bounding boxes lie under the mouse pointer.
So, to limit the sensitive zone of the morph to the cross shape, we need to override the \ct{containsPoint:} method.


\dothis{Define the following method in class \ct{CrossMorph}:}

\needlines{4}
\begin{method}[firstContains]{Shaping the sensitive zone of the \ct{CrossMorph}.}
containsPoint: aPoint
	| crossHeight crossWidth horizontalBar verticalBar |
	crossHeight := self height / 3.0.
	crossWidth := self width / 3.0.
	horizontalBar := self bounds insetBy: 0 @ crossHeight.
	verticalBar := self bounds insetBy: crossWidth @ 0.
	^ (horizontalBar containsPoint: aPoint)
		or: [verticalBar containsPoint: aPoint]
\end{method}

This method uses the same logic as \ct{drawOn:}, so we can be confident that the points for which \ct{containsPoint:} answers \ct{true} are the same ones that will be colored in by \ct{drawOn}. 
Notice how we leverage the \mthind{Rectangle}{containsPoint:} method in class \ct{Rectangle} to do the hard work.

There are two problems with the code in \mthsref{firstDrawOn} and \ref{mth:firstContains}.
The most obvious is that we have duplicated code.
This is a cardinal error: if we find that we need to change the way that \ct{horizonatalBar} or \ct{verticalBar} are calculated, we are quite likely to forget to change one of the two occurrences.
The solution is to factor out these calculations into two new methods, which we put in the \ct{private} protocol:

\needlines{4}
\begin{method}{\ct{horizontalBar}.}
horizontalBar
	| crossHeight |
	crossHeight := self height / 3.0.
	^ self bounds insetBy: 0 @ crossHeight
\end{method}

\needlines{4}
\begin{method}{\ct{verticalBar}.}
verticalBar
	| crossWidth |
	crossWidth := self width / 3.0.
	^ self bounds insetBy: crossWidth @ 0
\end{method}

\noindent
We can then define both \ct{drawOn:} and \ct{containsPoint:} using these methods:

\needlines{4}
\begin{method}{Refactored \ct{CrossMorph>>>drawOn:}.}
drawOn: aCanvas 
	aCanvas fillRectangle: self horizontalBar color: self color.
	aCanvas fillRectangle: self verticalBar color: self color
\end{method}

\needlines{4}
\begin{method}{Refactored \ct{CrossMorph>>>containsPoint:}.}
containsPoint: aPoint 
	^ (self horizontalBar containsPoint: aPoint)
		or: [self verticalBar containsPoint: aPoint]
\end{method}

This code is much simpler to understand, largely because we have given meaningful names to the private methods.
In fact, it is so simple that you may have noticed the second problem: the area in the center of the cross, which is under both the horizontal and the vertical bars, is drawn twice.  
This doesn't matter when we fill the cross with an opaque colour, but the bug becomes apparent immediately if we draw a semi-transparent cross, as shown in \figref{overdrawBug}.

\begin{figure}[t]
\begin{minipage}{0.48\textwidth}
	\ifluluelse
		{\centerline{\includegraphics[scale=0.6]{overdrawBug}}}
		{\centerline{\includegraphics{overdrawBug}}}
	\caption{The center of the cross is filled twice with the colour.
		\figlabel{overdrawBug}}
\end{minipage}
\hfill
\begin{minipage}{0.48\textwidth}
	\ifluluelse
		{\centerline{\includegraphics[scale=0.6]{hairlineBug}}}
		{\centerline{\includegraphics{bug}}}
	\caption{The cross-shaped morph, showing a row of unfilled pixels.
		\figlabel{bug}}
\end{minipage}
\end{figure}

\needlines{4}
\dothis{Execute the following code in  a workspace, line by line:}

\begin{code}{}
m := CrossMorph new bounds: (0@0 corner: 300@300).
m openInWorld.
m color: (Color blue alpha: 0.3).
\end{code}

\noindent
The fix is to divide the vertical bar into three pieces, and to fill only the top and bottom.  
Once again we find a method in class \ct{Rectangle} that does the hard work for us: \ct{r1 areasOutside: r2} answers an array of rectangles comprising the parts of \ct{r1} outside \ct{r2}.
Here is the revised code:

\begin{method}{The revised \ct{drawOn:} method, which fills the center of the cross once.}
drawOn: aCanvas 
	| topAndBottom |
	aCanvas fillRectangle: self horizontalBar color: self color.
	topAndBottom := self verticalBar areasOutside: self horizontalBar. 
	topAndBottom do: [ :each | aCanvas fillRectangle: each color: self color]
\end{method}

This code seems to work, but if you try it on some crosses and resize them, you may notice that at some sizes, a one-pixel wide line separates the bottom of the cross from the remainder, as shown in \figref{bug}.
This is due to rounding: when the size of the rectangle to be filled is not an integer, \ct{fillRectangle: color:}
seems to round inconsistently, leaving one row of pixels unfilled.
We can work around this by rounding explicitly when we calculate the sizes of the bars.

\needlines{5}
\begin{method}{\ct{CrossMorph>>>horizontalBar} with explicit rounding.}
horizontalBar
	| crossHeight |
	crossHeight := (self height / 3.0) rounded.
	^ self bounds insetBy: 0 @ crossHeight
\end{method}

\needlines{5}
\begin{method}{\ct{CrossMorph>>>verticalBar} with explicit rounding.}
verticalBar
	| crossWidth |
	crossWidth := (self width / 3.0) rounded.
	^ self bounds insetBy: crossWidth @ 0
\end{method}



%=================================================================
%\section{Composing Morphs}

%\on{The source code is in the examples directory.
%For the moment I prefer to leave out the examples, as they do not add much.}

%Below, we present a few morphs that were designed for a course project.

%\paragraph{An adhesive Label} The \ct{LabelStickerMorph} is a metaphor for an adhesive label with a colored border and three lines of text (\figref{labeler}, \egref{labeler}).

%\begin{figure}[ht]
%	\centerline{\includegraphics[width=0.25\textwidth]{labeler}}
%	\caption{The sticker label morph.
%		\figlabel{labeler}}
%\end{figure}

%\begin{example}[labeler]{Creating a sticker label}{}
%label := LabelstickerMorph new openInWorld.
%label text1: 'Confiture sans sucre';
%	text2: 'Fraises du jardin';
%	text3: '9 mai 2006'.
%label lineColor: Color blue
%\end{example}

%\paragraph{A Number Pyramid}
%The previous morph is designed by overriding the \ct{drawOn:} method.
%We built \ct{PyramidMorph} by composing morphs: we used \ct{TextMorph}s to make the blocks and added them to a base morph (\figref{pyramid}, \egref{pyramid}). \damien{figure does not match text... no numbers? Where is the code?}
%\begin{figure}[ht]
%	\centerline{\includegraphics{pyramid}}
%	\caption{The number pyramid morph.
%		\figlabel{pyramid}}
%\end{figure}

%\begin{example}[pyramid]{Manipulating the number pyramid}{}
%pyramid := (PyramidMorph base: 4) openInWorld.
%pyramid block: 8 value: 2
%\end{example}


%=================================================================
\section{Interaction and animation}

To build live user-interfaces using morphs, we need to be able to interact with them using the mouse and the keyboard.
Moreover, the morphs need to be able respond to user input by changing their appearance and position\,---\,that is, by animating themselves.


\subsection{Mouse events}

When a mouse button is pressed, Morphic sends each morph under the mouse pointer the message \ct{handlesMouseDown:}. If a morph answers \ct{true}, then Morphic immediately sends it the \mthind{Morph}{mouseDown:} message; it also sends the \mthind{Morph}{mouseUp:} message when the user releases the mouse button.
If all morphs answer \ct{false}, then Morphic initiates a drag-and-drop operation.
As we will discuss below, the \ct{mouseDown:} and \ct{mouseUp:} messages are sent with an argument\,---\,a \clsind{MouseEvent} object\,---\,that encodes the details of the mouse action.

Let's extend \ct{CrossMorph} to handle mouse events.  
We start by ensuring that all crossMorphs answer \ct{true} to the \mthind{Morph}{handlesMouseDown:} message. 

\dothis{Add this method to \ct{CrossMorph}:}
\begin{method}{Declaring that \ct{CrossMorph} will react to mouse clicks.}
CrossMorph>>>handlesMouseDown: anEvent
	^true
\end{method}

Suppose that when we click on the cross, we want to change the color of the cross to red, and when we \actclick on it, we want to change the color to yellow.
This can be accomplished by \mthref{mouseDown}.

\needlines{7}
\begin{method}[mouseDown]{Reacting to mouse clicks by changing the morph's color.}
CrossMorph>>>mouseDown: anEvent
	anEvent redButtonPressed "click"
		ifTrue: [self color: Color red].
	anEvent yellowButtonPressed "action-click"
		ifTrue: [self color: Color yellow].
	self changed
\end{method}

\ab{I added this note:}
Notice that in addition to changing the color of the morph, this method also sends \ct{self changed}.
This makes sure that morphic sends \ct{drawOn:} in a timely fashion.
\ab{However, the \ct{self changed} message seems to be entirely unnecessary; the colour changes instantly without it.}
Note also that once the morph handles \ind{mouse events}, you can no longer grab it with the mouse and move it.
Instead you have to use the halo: \metaclick on the morph to make the halo appear and grab either the brown move handle \moveHandle{} or the black pickup handle \grabHandle{} at the top of the morph.

The \ct{anEvent} argument of \ct{mouseDown:} is an instance of \mbox{\clsind{MouseEvent},} which is a subclass of \lct{Mor\-phic\-Event}. \ct{MouseEvent} defines the \mthind{MouseEvent}{redButtonPressed} and \mthind{MouseEvent}{yellowButtonPressed} methods. Browse this class to see what other methods it provides to interrogate the mouse event.

\subsection{Keyboard events}

To catch \ind{keyboard events}, we need to take three steps.
\begin{enumerate}
	\item Give the ``keyboard focus'' to a specific morph: for instance we can give focus to our morph when the mouse is over it.
	\item Handle the keyboard event itself with the \mthind{Morph}{handleKeystroke:} method: this message is sent to the morph that has keyboard focus when the user presses a key.
	\item Release the keyboard focus when the mouse is no longer over our morph.
\end{enumerate}

Let's extend \ct{CrossMorph} so that it reacts to keystrokes.
First, we need to arrange to be notified when the mouse is over the morph.
This will happen if our morph answers \ct{true} to the \mthind{Morph}{handlesMouseOver:} message

\dothis{Declare that \ct{CrossMorph} will react when it is under the mouse pointer.}
\begin{method}{We want to handle ``mouse over'' events.} 
CrossMorph>>>handlesMouseOver: anEvent
	^true
\end{method}

\noindent
This message is the equivalent of \mthind{Morph}{handlesMouseDown:} for the mouse position.
When the mouse pointer enters or leaves the morph, the \mthind{Morph}{mouseEnter:} and \mthind{Morph}{mouseLeave:} messages are sent to it.

\dothis{Define two methods so that \ct{CrossMorph} catches and releases the keyboard focus, and a third method to actually handle the keystrokes.}
\begin{method}{Getting the keyboard focus when the mouse enters the morph.}
CrossMorph>>>mouseEnter: anEvent
	anEvent hand newKeyboardFocus: self
\end{method}

\begin{method}{Handing back the focus when the pointer goes away.}
CrossMorph>>>mouseLeave: anEvent
	anEvent hand newKeyboardFocus: nil
\end{method}

\begin{method}[handleKeystroke]{Receiving and handling keyboard events.}
CrossMorph>>>handleKeystroke: anEvent
	| keyValue |
	keyValue := anEvent keyValue.
	keyValue = 30	 "up arrow"
		ifTrue: [self position: self position - (0 @ 1)].
	keyValue = 31	 "down arrow"
		ifTrue: [self position: self position + (0 @ 1)].
	keyValue = 29	 "right arrow"
		ifTrue: [self position: self position + (1 @ 0)].
	keyValue = 28	 "left arrow"
		ifTrue: [self position: self position - (1 @ 0)]
\end{method}

We have written this method so that you can move the morph using the arrow keys.
Note that when the mouse is no longer over the morph, the \mthind{Morph}{handleKeystroke:} message is not sent, so the morph stops responding to keyboard commands.
To discover the key values, you can open a Transcript window and add \glbind{Transcript} \ct{show: anEvent keyValue} to \mthref{handleKeystroke}.
The \ct{anEvent} argument of \ct{handleKeystroke:} is an instance of \clsind{KeyboardEvent}, another subclass of \clsind{MorphicEvent}. Browse this class to learn more about keyboard events.

\subsection{Morphic animations}

Morphic provides a simple animation system with two main methods: \mthind{Morphic}{step} is sent to a morph at regular intervals of time, while \mthind{Morphic}{stepTime} specifies the time in milliseconds between \ct{step}s.\footnote{\ct{stepTime} is actually the \emph{minimum} time between \ct{step}s.   If you ask for a \ct{stepTime} of 1\,ms, don't be surprised if \pharo is too busy to step your morph that often.}
In addition, 
\mthind{Morphic}{startStepping} turns on the stepping mechanism, while \mthind{Morphic}{stopStepping} turns it off again;  \mthind{Morphic}{isStepping} can be used to find out whether a morph is currently being stepped.
\index{Morphic!animation}

\dothis{Make \ct{CrossMorph} blink by defining these methods as follows:}
\begin{method}{Defining the animation time interval.}
CrossMorph>>>stepTime
	^ 100
\end{method}
\begin{method}{Making a step in the animation.}
CrossMorph>>>step
	(self color diff: Color black) < 0.1
		ifTrue: [self color: Color red]
		ifFalse: [self color: self color darker]
\end{method}
\noindent
To start things off, you can open an inspector on a \ct{CrossMorph} (using the debug handle \debugHandle{} in the morphic halo), type \ct{self startStepping} in the small workspace pane at the bottom, and  \menu{do it}.
Alternatively, you can modify the \ct{handleKeystroke:} method so that you can use the $+$ and $-$ keys to start and stop stepping. 

\dothis{Add the following code to \mthref{handleKeystroke}:}

\begin{code}{}
	keyValue = $+ asciiValue 
		ifTrue: [self startStepping].
	keyValue = $- asciiValue
		ifTrue: [self stopStepping].
\end{code}

% \on{You can also \menu{debug \go inspect morph} and evaluate: \ct{self currentWorld startStepping: self}.}

%=================================================================
\section{Interactors}

To prompt the user for input, the \clsind{UIManager} class provides a large number of ready-to-use dialog boxes.
For instance, the \mthind{UIManager}{request:initialAnswer:} method returns the string entered by the user (\figref{dialogName}).
\begin{code}{}
UIManager default request: 'What''s your name?' initialAnswer: 'no name'
\end{code}

\begin{figure}[htb]
\begin{minipage}{0.48\textwidth}
	\centerline{\includegraphics[width=0.8\textwidth]{dialog}}
	\caption{An input dialog.}\figlabel{dialogName}
\end{minipage}
\hfill
\begin{minipage}{0.48\textwidth}
	\vfill
	\centerline{\includegraphics [width=0.8\textwidth]{popup}}
	\vfill
	\vspace{4ex}
	\caption{Pop-up menu.}\figlabel{popup}
\end{minipage}
\end{figure}

To display a popup menu, use one of the various \ct{chooseFrom:} methods (\figref{popup}):
\begin{code}{}
UIManager default
	chooseFrom: #('circle' 'oval' 'square' 'rectangle' 'triangle')
	lines: #(2 4) message: 'Choose a shape'
\end{code}

\dothis{Browse the \clsind{UIManager} class and try out some of the interaction methods offered.}

%\begin{figure}[ht]
%	\centerline{\includegraphics[width=5cm]{dialog}}
%	\caption{Dialog displayed by \ct{FillInTheBlank request: 'What''s your name?' initialAnswer: 'no name'}.
%		\figlabel{dialogName}}
%\end{figure}

%To display a pop-up menu, use the \clsind{PopupMenu} class:
%\begin{code}{}
%UIManager default chooseFrom: #('circle' 'oval' 'square' 'rectangle' 'triangle') lines: #(2 4) message: 'Choose a shape'
%\end{code}

%\begin{figure}[ht]
%	\centerline{\includegraphics[width=3cm]{popup}}
%	\caption{PopUp displayed by \ct{PopUpMenu>>>startUpWithCaption:}.}
%\end{figure}

%=================================================================
\section{Drag-and-drop}

Morphic also supports drag-and-drop. Let's examine a simple example with two morphs, a receiver morph and a dropped morph. 
The receiver will accept a morph only if the dropped morph matches a given condition: in our example,  the morph should be blue. If it is rejected, the dropped morph decides what to do.

\dothis{Let's first define the receiver morph:}
\begin{classdef}{Defining a morph on which we can drop other morphs}
Morph subclass: #ReceiverMorph
	instanceVariableNames: ''
	classVariableNames: ''
	poolDictionaries: ''
	category: 'PBE-Morphic'
\end{classdef}

\dothis{Now define the initialization method in the usual way:}
\begin{method}{Initializing \ct{ReceiverMorph}.}
ReceiverMorph>>>initialize
	super initialize.
	color := Color red.
	bounds := 0 @ 0 extent: 200 @ 200
\end{method}

How do we decide if the receiver morph will accept or repel the dropped morph?
In general, both of the morphs will have to agree to the interaction.
The receiver does this by responding to \mthind{Morph}{wantsDroppedMorph:event:}; the first argument is the dropped morph, and the second the mouse event, so that the receiver can, for example, see if any modifier keys were held down at the time of the drop. 
The dropped morph is also given the opportunity to check and see if it likes the morph onto which it is being dropped; it is sent the message \ct{wantsToBeDroppedInto:}.  The default implementation of  this method (in class \ct{Morph}) answers \ct{true}.

\begin{method}{Accept dropped morphs based on their color.}
ReceiverMorph>>>wantsDroppedMorph: aMorph event: anEvent
	^ aMorph color = Color blue
\end{method}

What happens to the dropped morph if the receiving morph doesn't want it?  The default behaviour is for it to do nothing, that is, to sit on top of the receiving morph, but without interacting with it.  A more intuitive behavior is for the dropped morph to go back to its original position.  This can be achieved by the receiver answering \ct{true} to the message \mthind{Morph}{repelsMorph:event:} when it doesn't want the dropped morph:

\needlines{4}
\begin{method}{Changing the behaviour of the dropped morph when it is rejected.}
ReceiverMorph>>>repelsMorph: aMorph event: ev
	^ (self wantsDroppedMorph: aMorph event: ev) not
\end{method}

That's all we need as far as the receiver is concerned.

\dothis{Create instances of \clsind{ReceiverMorph} and \clsind{EllipseMorph} in a workspace:}
\begin{code}{}
ReceiverMorph new openInWorld.
EllipseMorph new openInWorld.
\end{code}
\noindent
Try to drag-and-drop the yellow \ct{EllipseMorph} onto the receiver. It will be rejected and sent back to its initial position.

\dothis{To change this behaviour, change the color of the ellipse morph to \ct{Color blue} using an inspector.  Blue morphs should be accepted by the \ct{ReceiverMorph}.}

Let's create a specific subclass of \ct{Morph}, named \ct{DroppedMorph}, so we can experiment a bit more:

\begin{classdef}{Defining a morph we can drag-and-drop onto \ct{ReceiverMorph}}
Morph subclass: #DroppedMorph
	instanceVariableNames: ''
	classVariableNames: ''
	poolDictionaries: ''
	category: 'PBE-Morphic'
\end{classdef}

\needlines{5}
\begin{method}{Initializing \ct{DroppedMorph}.}
DroppedMorph>>>initialize
	super initialize.
	color := Color blue.
	self position: 250@100
\end{method}

Now we can specify what the dropped morph should do when it is rejected by the receiver; here it will stay attached to the mouse pointer:
\begin{method}{Reacting when the morph was dropped but rejected.}
DroppedMorph>>>rejectDropMorphEvent: anEvent
	| h |
	h := anEvent hand.
	WorldState
		addDeferredUIMessage: [h grabMorph: self].
	anEvent wasHandled: true
\end{method}

Sending the \mthind{MorphicEvent}{hand} message to an event answers the \emph{hand}, an instance of \ct{HandMorph} that represents the mouse pointer and whatever it holds.
Here we tell the \ct{World} that the hand should grab \ct{self}, the rejected morph.

\dothis{Create two instances of \ct{DroppedMorph}, and then drag-and-drop them onto the receiver.}
\begin{code}{}
ReceiverMorph new openInWorld.
(DroppedMorph new color: Color blue) openInWorld.
(DroppedMorph new color: Color green) openInWorld.
\end{code}
\noindent
The green morph is rejected and therefore stays attached to the mouse pointer.

%=================================================================
\section{A complete example}

Let's design a morph to roll a die\footnote{NB: One die, two dice.}. Clicking on it will display the values of all sides of the die in a quick loop, and another click will stop the animation.

\begin{figure}[ht]
	\centerline{\includegraphics[scale=0.65]{die}}
	\caption{The die in Morphic.
		\figlabel{dialogDie}}
\end{figure}

\dothis{Define the die as a subclass of \clsind{BorderedMorph} instead of \ct{Morph}, because we will make use of the border.}

\needlines{6}
\begin{classdef}{Defining the die morph}
BorderedMorph subclass: #DieMorph
	instanceVariableNames: 'faces dieValue isStopped'
	classVariableNames: ''
	poolDictionaries: ''
	category: 'PBE-Morphic'
\end{classdef}

The instance variable \ct{faces} records the number of faces on the die; we allow dice with up to 9 faces! \ct{dieValue} records the value of the face that is currently displayed, and \ct{isStopped} is true if the die animation has stopped running.
To create a die instance, we define the \ct{faces: n} method on the \emph{class} side of \clsind{DieMorph} to create a new die with \ct{n} faces.
\begin{method}{Creating a new die with the number of faces we like.}
DieMorph class>>>faces: aNumber
	^ self new faces: aNumber
\end{method}

The \ct{initialize} method is defined on the instance side in the usual way; remember that \ct{new} sends \ct{initialize} to the newly-created instance.
\begin{method}{Initializing instances of \ct{DieMorph}.}
DieMorph>>>initialize
	super initialize.
	self extent: 50 @ 50.
	self useGradientFill; borderWidth: 2; useRoundedCorners.
	self setBorderStyle: #complexRaised.
	self fillStyle direction: self extent.
	self color: Color green.
	dieValue := 1.
	faces := 6.
	isStopped := false
\end{method}

We use a few methods of \ct{BorderedMorph} to give a nice appearance to the die: a thick border with a raised effect, rounded corners, and a color gradient on the visible face.
We define the instance method \ct{faces:} to check for a valid parameter as follows:
\begin{method}{Setting the number of faces of the die.}
DieMorph>>>faces: aNumber
	"Set the number of faces"
	(aNumber isInteger
			and: [aNumber > 0]
			and: [aNumber <= 9])
		ifTrue: [faces := aNumber]
\end{method}
\on{Why not make this a pre-condition, \ie an assertion?}

It may be good to review the order in which the messages are sent when a die is created. For instance, if we start by
evaluating \ct{DieMorph faces: 9}:
\begin{enumerate}
	\item The class method \ct{DieMorph class>>>faces:} sends \ct{new} to \ct{DieMorph class}.
	\item The method for \ct{new} (inherited by \ct{DieMorph class} from \ct{Behavior}) creates the new instance and sends it the \ct{initialize} message.
	\item The \ct{initialize} method in \ct{DieMorph} sets \ct{faces} to an initial value of 6.
	\item \ct{DieMorph class>>>new} returns to the class method \ct{DieMorph class>>>faces:}, which then sends the message \ct{faces: 9} to the new instance.
	\item The instance method \ct{DieMorph>>>faces:} now executes, setting the \ct{faces} instance variable to 9.
\end{enumerate}

Before defining \ct{drawOn:}, we need a few methods to place the dots on the displayed face:
\begin{methods}{Nine methods for placing points on the faces of the die.}
DieMorph>>>face1
	^{0.5@0.5}
DieMorph>>>face2
	^{0.25@0.25 . 0.75@0.75}
DieMorph>>>face3
	^{0.25@0.25 . 0.75@0.75 . 0.5@0.5}
DieMorph>>>face4
	^{0.25@0.25 . 0.75@0.25 . 0.75@0.75 . 0.25@0.75}
DieMorph>>>face5
	^{0.25@0.25 . 0.75@0.25 . 0.75@0.75 . 0.25@0.75 . 0.5@0.5}
DieMorph>>>face6
	^{0.25@0.25 . 0.75@0.25 . 0.75@0.75 . 0.25@0.75 . 0.25@0.5 . 0.75@0.5}
DieMorph>>>face7
	^{0.25@0.25 . 0.75@0.25 . 0.75@0.75 . 0.25@0.75 . 0.25@0.5 . 0.75@0.5 . 0.5@0.5}
DieMorph >>>face8
	^{0.25@0.25 . 0.75@0.25 . 0.75@0.75 . 0.25@0.75 . 0.25@0.5 . 0.75@0.5 . 0.5@0.5 . 0.5@0.25}
DieMorph >>>face9
	^{0.25@0.25 . 0.75@0.25 . 0.75@0.75 . 0.25@0.75 . 0.25@0.5 . 0.75@0.5 . 0.5@0.5 . 0.5@0.25 . 0.5@0.75}
\end{methods}
\on{kind of ugly boilerplate code -- should be a nice way to map these more elegantly to coordinates.}

These methods define collections of the coordinates of dots for each face. The coordinates are in a square of size $1\times1$; we will simply need to scale them to place the actual dots.

The \ct{drawOn:} method does two things: it draws the die background with the \ct{super}-send, and then draws the dots.
\begin{method}{Drawing the die morph.}
DieMorph>>>drawOn: aCanvas
	super drawOn: aCanvas.
	(self perform: ('face' , dieValue asString) asSymbol)
		do: [:aPoint | self drawDotOn: aCanvas at: aPoint]
\end{method}

The second part of this method uses the reflective capacities of \st.
Drawing the dots of a face is a simple matter of iterating over the collection given by the \ct{faceX} method for that face, sending the \ct{drawDotOn:at:} message for each coordinate. To call the correct \ct{faceX} method, we use the \mthind{Object}{perform:} method which sends a message built from a string, here \lct{('face', dieValue asString) asSymbol}. You will encounter this use of \ct{perform:} quite regularly.
\index{reflection}
\begin{method}{Drawing a single dot on a face.}
DieMorph>>>drawDotOn: aCanvas at: aPoint
	aCanvas
		fillOval: (Rectangle
			center: self position + (self extent * aPoint)
			extent: self extent / 6)
		color: Color black
\end{method}
\ew{I would not use reflection to call faceX on p.234. A statically filled Dictionary would be more appropriate. Get the point set with "MyDict at: X".}

Since the coordinates are normalized to the $[0{:}1]$ interval, we scale them to the dimensions of our die: \ct{self extent * aPoint}.

\dothis{We can already create a die instance from a workspace:}
\begin{code}{}
(DieMorph faces: 6) openInWorld.
\end{code}

To change the displayed face, we create an accessor that we can use as \ct{myDie dieValue: 4}:
\begin{method}{Setting the current value of the die.}
DieMorph>>>dieValue: aNumber
	(aNumber isInteger
			and: [aNumber > 0]
			and: [aNumber <= faces])
		ifTrue:
			[dieValue := aNumber.
			self changed]
\end{method}

Now we will use the animation system to show quickly all the faces:
\needlines{3}
\index{Morphic!animation}
\begin{methods}{Animating the die.}
DieMorph>>>stepTime
	^ 100

DieMorph>>>step
	isStopped ifFalse: [self dieValue: (1 to: faces) atRandom]
\end{methods}
Now the die is rolling!

To start or stop the animation by clicking, we will use what we learned previously about mouse events.
First, activate the reception of mouse events:

\begin{methods}{Handling mouse clicks to start and stop the animation.}
DieMorph>>>handlesMouseDown: anEvent
	^ true

DieMorph>>>mouseDown: anEvent
	anEvent redButtonPressed
		ifTrue: [isStopped := isStopped not]
\end{methods}
Now the die will roll or stop rolling when we click on it.

% That's all for the essentials of Morphic!

% Most of the work on \ct{DieMorph} was done with an instance of it living in the environment; this is quite nice when to tweak programs.

%=================================================================
\section{More about the canvas}

The \ct{drawOn:} method has an instance of \clsindmain{Canvas} as its sole argument;
the canvas is the area on which the morph draws itself.
By using the graphics methods of the canvas you are free to give the appearance you want to a morph.
If you browse the inheritance hierarchy of the \ct{Canvas} class, you will see that it has several variants.
The default variant of \ct{Canvas} is \clsind{FormCanvas}; you will find the key graphics methods in \ct{Canvas} and \ct{FormCanvas}.
These methods can draw points, lines, polygons, rectangles, ellipses, text, and images with rotation and scaling.

It is also possible to use other kinds of canvas, to obtain transparent morphs, more graphics methods, antialiasing, and so on.
To use these features you will need an \clsind{AlphaBlendingCanvas} or a \clsind{BalloonCanvas}.
But how can you obtain such a canvas in a \ct{drawOn:} method, when \ct{drawOn:} receives an instance of \ct{FormCanvas} as its argument?
Fortunately, you can transform one kind of canvas into another.

\dothis{To use a canvas with a 0.5 alpha-transparency in \ct{DieMorph}, redefine \ct{drawOn:} like this:}
\needlines{7}
\begin{method}{Drawing a translucent die.}
DieMorph>>>drawOn: aCanvas
	| theCanvas |
	theCanvas := aCanvas asAlphaBlendingCanvas: 0.5.
	super drawOn: theCanvas.
	(self perform: ('face' , dieValue asString) asSymbol)
		do: [:aPoint | self drawDotOn: theCanvas at: aPoint]
\end{method}
\noindent
That's all you need to do!

\begin{figure}[ht]
	\centerline{\includegraphics[scale=0.7]{multiMorphs}}
	\caption{The die displayed with alpha-transparency.
		\figlabel{multiMorphs}}
\end{figure}

% ON: This appears to be broken:

%If you're curious, have a look at the \mthind{Canvas}{asAlphaBlendingCanvas:} method.
%You can also get antialiasing by using \clsind{BalloonCanvas} and transforming the die drawing methods as shown in \mthsref{aadie}.

%\needlines{6}
%\begin{methods}[aadie]{Drawing an antialiased die.}
%DieMorph>>>drawOn: aCanvas
%	| theCanvas |
%	theCanvas := aCanvas asBalloonCanvas aaLevel: 3.
%	super drawOn: aCanvas.
%	(self perform: ('face' , dieValue asString) asSymbol)
%		do: [:aPoint | self drawDotOn: theCanvas at: aPoint]

%DieMorph>>>drawDotOn: aCanvas at: aPoint
%	aCanvas
%		drawOval: (Rectangle
%			center: self position + (self extent * aPoint)
%			extent: self extent / 6)
%		color: Color black
%		borderWidth: 0
%		borderColor: Color transparent
%\end{methods}

%=================================================================
\section{Chapter summary}

Morphic is a graphical framework in which graphical interface elements can be dynamically composed.

\begin{itemize}
  \item You can convert an object into a morph and display that morph on the screen by sending it the messages \ct{asMorph openInWorld}.
  \item You can manipulate a morph by \metaclick{ing} on it and using the handles that appear. (Handles have help balloons that explain what they do.)
  \item You can compose morphs by embedding one onto another, either by drag and drop or by sending the message \ct{addMorph:}.
  \item You can subclass an existing morph class and redefine key methods, like \ct{initialize} and \ct{drawOn:}.
  \item You can control how a morph reacts to mouse and keyboard events by redefining the methods \ct{handlesMouseDown:}, \ct{handlesMouseOver:}, \etc
  \item You can animate a morph by defining the methods \ct{step} (what to do) and \ct{stepTime} (the number of milliseconds between steps).
  \item Various pre-defined morphs, like \ct{PopUpMenu} and \ct{FillInTheBlank}, are available for interacting with users.
\end{itemize}

%=================================================================
\ifx\wholebook\relax\else\end{document}\fi
%=================================================================

%-----------------------------------------------------------------
