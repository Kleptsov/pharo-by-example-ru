% $Author: oscar $
% $Date: 2009-09-28 17:41:15 +0600 (пн, 28 сен 2009) $
% $Revision: 29309 $

% HISTORY:
% 2006-12-01 - Andrew edited (split from FirstApp?)
% 2006-12-03 - Andrew first draft
% 2006-12-06 - Stef edit
% 2007-06-11 - Oscar edit
% 2007-07-03 - Stef review
% 2007-08-22 - Andrew corrections
% 2007-09-11 - Marcus review
% 2007-09-11 - Orla review
% 2009-07-04 - Oscar migrated to Pharo

%=================================================================
\ifx\wholebook\relax\else
% --------------------------------------------
% Lulu:
	\documentclass[a4paper,10pt,twoside]{book}
	\usepackage[
		papersize={6.13in,9.21in},
		hmargin={.75in,.75in},
		vmargin={.75in,1in},
		ignoreheadfoot
	]{geometry}
	\input{../common.tex}
	\pagestyle{headings}
	\setboolean{lulu}{true}
% --------------------------------------------
% A4:
%	\documentclass[a4paper,11pt,twoside]{book}
%	\input{../common.tex}
%	\usepackage{a4wide}
% --------------------------------------------
    \graphicspath{{figures/} {../figures/}}
	\begin{document}
	% \renewcommand{\nnbb}[2]{} % Disable editorial comments
	\sloppy
\fi
%=================================================================
\newcommand{\clover}{%
	\raisebox{-0.8ex}[0pt][0pt]{%
		\includegraphics[width=1em]{cloverleafKey}}}
%=================================================================
% \chapter{A quick tour of \pharo}
\chapter{Краткий обзор \pharo}
\chalabel{quick}

% In this chapter we will give you a high-level tour of \pharo to help you get
% comfortable with the environment.
% There will be plenty of opportunities to try things out, so it would be a good
% idea if you have a computer handy when you read this chapter.

В этой главе мы постараемся провести содержательный обзор по Pharo
с целью помочь Вам познакомиться с удобной средой.
Будет много возможностей попробовать разные вещи,
поэтому было бы неплохо,
если в процессе чтения у Вас был бы компьютер. 

% We will use this icon: \dothisicon{} to mark places in the text where you
% should try something out in \pharo.
% In particular, you will fire up \pharo, learn about the different ways of
% interacting with the system, and discover some of the basic tools.
% You will also learn how to define a new method,
% create an object and send it messages.

Мы будем использовать этот символ: \dothisicon{} в качестве отметок
в тексте, где Вы должны попробовать что-то в Pharo.
В частности,  используя Pharo, Вы узнаете различные способы
взаимодействия с системой и изучите основные инструменты.
Вы также узнаете, как определить новый метод, создать
объект и как послать му сообщения. 

%=================================================================
% \section{Getting started}
\section{Приступая к работе}

% \pharo is available as a free \ind{download} from \pharoweb.
% There are three parts that you will need to download,
% consisting of four files (see \figref{download}).

\pharo доступен для загрузки с http://pharo-project.org.
И состоит из трех частей, которые вам необходимо скачать и
которые включают в себя 4 файла (см. \figref{download}).

\begin{figure}[htb]
\centerline {\includegraphics[width=\textwidth]{annotatedDownload-flat}}
% \caption{The \pharo download files for one of the supported platforms.
\caption{Загружаемые файлы Pharo для одной из поддерживаемых
платформ \figlabel{download}}
\end{figure}

\begin{enumerate}

% \item The \emphind{virtual machine} (VM) is the only part of the system
% that is different for each operating system and processor.  Pre-compiled
% virtual machines are available for all the major computing environments.
% In \figref{download} we see the VM for the selected platform is called
% \textit{Pharo.exe}.

\item Виртуальная машина (VM) является единственной частью системы,
которая отличается для каждой операционной системы  и процессора.
Предварительно скомпилированные виртуальные машины доступны
для всех основных вычислительных сред. На \figref{download} мы
видим, что VM для выбранной платформы  называется \textit{Pharo.exe}. 

%  \item The \emphind{sources} file contains the source code for all of the
% parts of \pharo that don't change very frequently. In \figref{download} it is
% called \emph{SqueakV39.sources}.\footnote{\pharo is derived from \squeak
% 3.9, and presently shares the VM with \squeak.}

\item Файл с расширением .source
содержит исходный код для всех частей Pharo и
изменяется очень редко. На \figref{download} он называется
\emph{SqueskV39.sources}
\footnote{Pharo унаследован от  Squeak3.9, и теперь делит VM со Squeak.}

%  \item The current \emph{system \ind{image}} is a snapshot of a running
% \pharo system, frozen in time.  It consists of two files: an
% \emph{.}\emphind{image} file, which contains the state of all of the objects
% in the system (including classes and methods, since they are objects too),
% and a \emph{.}\emphind{changes} file, which contains a log of all of the
% changes to the source code of the system.
% In \figref{download}, these files are called \emph{pharo.image} and
% \emph{pharo.changes}.

\item Текущий системный образ является снимком исполняемой системы
и изменяется очень часто. Он состоит из двух файлов: \emph{.}
\emphind{image} файл, который хранит состояение всех объектов в
системе (в том числе классы и методы, т.к. тоже являются объектами),
и \emph{.}\emphind{changes} файл содержит журнал всех изменений
в исходном коде системы. На \figref{download} эти файлы названы
\emph{pharo.image} и \emph{pharo.changes} соответственно.

\end{enumerate}

% \dothis{Download and install \pharo on your computer.}
% We recommend that you use the image provided on the \pharo by Example
% web page.\footnote{\pbe}
% \index{download}
% \seclabel{sbeImage}

\dothis{Загрузите и установите \pharo на Ваш компьютер.}
Мы рекомендуем Вам использовать образ представленный на странице
\pharo by example.\footnote{\pbe}
\index{загрузка}
\seclabel{sbeImage}

% Most of the introductory material in this book will work with any version,
% so if you already have one installed, you may as well continue to use it.
% However, if you notice differences between the appearance or behaviour of
% your system and what is described here, do not be surprised. 

%%% On the other hand, if you are about to download \pharo for the first time, you may as well grab the \emph{\pharo by Example} image.

Большая часть вводного материала этой книги может работать
с любой версией \pharo,  так, если у Вас уже была установлена среда,
то можете успешно продолжить ее изучение. Однако,
не стоит удивляться, если Вы заметите различия в интерфейсе или
исполнении своей среды от той, что здесь описана. 

% As you work in \pharo, the image and changes files are modified,
% so you need to make sure that they are writable.
% Always keep these two files together.
% Never edit them directly with a text editor, as \pharo uses them to store the
% objects you work with and to log the changes you make to the source code.
% It is a good idea to keep a backup copy of the downloaded image and
% changes files so you can always start from a fresh image and reload
% your code.

При работе с \pharo  \emph{.}image и \emph{.}сhanges  файлы будут
изменяться, поэтому Вам необходимо  всякий раз убеждаться
в их читаемости (они должны быть доступны для записи).
Всегда держите оба файла вместе. Никогда не
изменяйте  их непосредственно с помощью текстового редактора,
поскольку \pharo использует их для хранения объектов,
с которыми Вы работаете,  и регистрирует изменения,
совершенные Вами в исходном коде.  Неплохо было бы делать резервные
копии загруженных .image и .changes файлов с той целью,
чтобы впоследствии Вы смогли восстановить образ и
перезагрузить свой код. 

% The \emphind{sources} file and the VM can be read-only\,---\,they can be
% shared between different users.
% All of these files can be placed in the same directory, but it is also possible
% to put the Virtual Machine and sources file in separate directory where
% everyone has access to them.
% Do whatever works best for your style of working and your operating
% system.

\emph{.}\emphind{sources} файл и VM могут быть использованы
только для чтения и могут разделяются между несколькими
пользователями. Все эти файлы могут лежать в одной и той же
директории, но так же возможно разместить виртуальную машину и
\emph{.}\emphind{sources} файлы в различных директориях,
где у каждого есть доступ к ним. Делайте все для
улучшения Вашего стиля работы и операционной системы.

%\sd{it would be really nice to have a setup and startup section on PC, Mac and Linux}
%\ab{I agree entirely; the reason that this is not here is because I could do only the Mac section.  Damien can probably do Windoze.  Perhaps we can ask on the list for a Linux volunteer?}

%-----------------------------------------------------------------
\begin{figure}[htb]
% \centerline {\includegraphics[width=0.6\textwidth]{download}}
\centerline {\includegraphics[width=\textwidth]{startup}}
% \caption{A fresh \pbe image.\figlabel{startup}
\caption{Новый образ загруженный из \pbe. \figlabel{startup}}
\end{figure}

% \index{launching \pharo}
% \paragraph{Launching.} To start \pharo, do whatever your operating
% system expects: drag the \emph{.}\emphind{image} file onto the icon
% of the virtual machine, or double-click the \emph{.image} file,
% or  at the command line type the name of the virtual machine followed
% by the path to the \emph{.image} file. (When you have multiple VMs
% installed on your machine the operating system may not automatically
% pick the right one; in this case it is safer to drag and drop the image
% onto the virtual machine, or to use the command line.)

\index{Запуск \pharo}
\paragraph{Запуск.} Для старта \pharo сделайте следующее:
перетащите \emph{.}\emphind{image} файл на иконку виртуальной
машины или дважды кликнете по этому файлу или еще один вариант:
пропишите в командной строке имя виртуальной машины и путь
к \emph{.}\emphind{image} файлу. (В случае если на
вашем компьютере установлено несколько виртуальных машин,
операционная система возможно не сможет определить нужную из них.
В этом случае наиболее безопасным будет перетащить
образ на VM или воспользоваться командной строкой.)

% Once \pharo is running, you should see a single large window,
% possibly containing some open workspace windows (see \figref{startup}),
% and it's not obvious how to proceed!
% You might notice a menu bar, but \pharo mainly makes use
% of context-dependent pop-up menus.
%%% You will notice that there is no menu bar, or at least not a useful one.  
%%% Instead,  \pharo makes heavy use of context-dependent pop-up menus.

Как только \pharo начнет работу, Вы увидите одно большое окно,
возможно содержащее некоторые открытые окна рабочей области
(workspace). Но как же двигаться дальше? Это неясно.
Вы могли бы заметить строку-меню, но в большинстве своем
\pharo использует контекстно-зависимые всплывающие меню.

% \dothis{Start \pharo. You can dismiss any open workspaces by \click{ing}
% on the red button in the top left corner of the workspace window.}

\dothis{Запуск \pharo.  Вы можете закрыть открытые рабочие окна,
кликнув по красной кнопке в левом верхнем углу рабочей области окна.}

% You can minimize windows (so that they move to the dock on the bottom
% of the screen) by \click{ing} on the orange button.  Clicking on the green
% button will cause the window to take up the entire screen.

Вы можете свернуть окна, нажав на оранжевую кнопку,
или развернуть их на весь экран, кликнув по зеленой кнопке.

%-----------------------------------------------------------------
% \paragraph{First Interaction.}
\paragraph{Первые действия.}

%A good place to get started is the \ind{world menu} shown in
% \figref{threeButtons:click}.

Рассмотрим меню, изображенные на \figref{threeButtons:click}. 

%\dothis{Click with the mouse on the background of the main window
% to show the world menu, then choose \menu{Workspace} to create
% a new workspace.}

\dothis{Кликнете мышью на фоне главного окна для просмотра
world-меню, затем выберите \menu{Workspace}
для создания новой рабочей области.}

%\begin{figure}[tbh]
%	\centering
%	\subfigure[The world menu]{\figlabel{threeButtons:click}% click
%		\includegraphics[width=0.25\linewidth]{worldMenu}}\hfill
%	\subfigure[The contextual menu]{\figlabel{threeButtons:actclick}% action click
%		\includegraphics[width=0.35\linewidth]{yellowButtonMenuOnWorkspace}}\hfill
%	\subfigure[The morphic halo]{\figlabel{threeButtons:metaclick}% meta click
%		\includegraphics[width=0.35\linewidth]{morphicHaloOnWorkspace}}% these braces needed (else no whitespace at end of line)
%	\caption{The world menu (brought up by \click{ing}), a contextual menu (\actclick{ing}), and a morphic \subind{Morphic}{halo} (\metaclick{ing}).\figlabel{threeButtons}}
%\end{figure}

\begin{figure}[tbh]
	\centering
	% \subfigure[The world menu]{\figlabel{threeButtons:click}% click
	\subfigure[Главное меню]{\figlabel{threeButtons:click}
		\includegraphics[width=0.40\linewidth]{worldMenu}}\hfill
	%\subfigure[The contextual menu]{\figlabel{threeButtons:actclick}% action click
	\subfigure[Контекстное меню]{\figlabel{threeButtons:actclick}
		\includegraphics[width=0.55\linewidth]{yellowButtonMenuOnWorkspace}}\hfill
	%\subfigure[The morphic halo]{\figlabel{threeButtons:metaclick}% meta click
	\subfigure[Окружение (ореол) морфа]{\figlabel{threeButtons:metaclick}
		\includegraphics[width=0.60\linewidth]{morphicHaloOnWorkspace}}% these braces needed (else no whitespace at end of line)
	%\caption{The world menu (brought up by \click{ing}),
	%a contextual menu (\actclick{ing}), and a morphic \subind{Morphic}
	% {halo} (\metaclick{ing}).\figlabel{threeButtons}}
	\caption{Главное меню, контекстное меню, ореол морфа
(элементы управления окружают морф)}
\end{figure}
\seeindex{morphic halo}{Morphic}

%% ON: I had to shrink this and move it up to avoid
%% it running over the end of the page.
%\begin{wrapfigure}[19]{r}{0.25\linewidth}
%% The parameters are the number of narrow lines to the right of the figure [19],
%% the placement {r} for right, and the width of the figure. Capital R will allow some float.
%% Inside the wrapfig environment, linewidth is special --- the width of the figure.
%\includegraphics[width=0.95\linewidth]{colouredMouse}
%\caption{The author's mouse. \click{ing} the scroll wheel activates the blue button.}
%\figlabel{colouredMouse}
%\end{wrapfigure}

% \st was originally designed for a computer with a \ind{three button mouse}.
% If your mouse has fewer than three buttons, you will have to press extra
% keys while \click{ing} the mouse to simulate the extra buttons.
% A two-button mouse works quite well with \pharo, but if you have only a
% single-button mouse, you should seriously consider buying a two-button
% mouse with a clickable scroll wheel: it will make working with \pharo much
% more pleasant.

\st был первоначально разработан для компьютера с мышью,
имеющей три кнопки. Если у Вашей мыши меньше трёх кнопок,
то Вам придется нажимать дополнительные комбинации клавишей мыши,
чтобы имитировать нужные комбинации. Двухкнопочная мышь
хорошо справляется с задачами \pharo (владельцам однокнопочной
мыши советуем приобрести мышь с двумя кнопками и с кликабельным
колесиком прокрутки, так вы облегчите себе работу с \pharo).

%\pharo avoids terms like ``left mouse click'' because different computers,
% mice, keyboards and personal configurations mean that different users will
% need to press different physical buttons to achieve the same effect.
% Originally \st introduced colours to stand for the different mouse buttons.
% \footnote{The button colours were \emph{red}, \emph{yellow} and
% \emph{blue}. The authors of this book could never remember which colour
% referred to which button.}
% \index{red button}
% \index{yellow button}
% \index{blue button}
%Since many users will use various modifiers keys (\emph{control},
% \emph{ALT}, \emph{meta} \etc) to achieve the same effect, we will instead
% use the following terms:

\pharo избегает таких словосочетаний как <<щелкните левой кнопкой
мыши>>, потому что разные компьютеры, мыши, клавиатуры и
персональные конфигурации означают, что разным пользователям нужно
нажимать различные кнопки для достижения одного и
того же эффекта.
Первоначально Smalltalk вводил цвета для обозначения различных кнопок
мыши.\footnote{Цвета кнопок были красными, желтыми и синими.
Авторы этой
книги уже и  не помнят, какая кнопка какой цвет обозначает.}
\index{красная кнопка} \index{желтая кнопка} \index{голубая кнопка}
С тех пор на помощь многим пользователям приходят различные кнопки
(Ctrl, Alt и т.д.), мы же,  в свою
очередь, будем использовать следующие термины: 

\begin{description}
%\item [\click:] this is the most often used mouse button, and is normally
% equivalent to \click{ing} a single-mouse button without any modifier key;
% \click{} on the image to bring up the ``World'' menu
% (\figref{threeButtons:click}).

\item[клик:] (или щелчек) чаще всего используемая кнопка мыши и,
как правило, эквивалентна клику на однокнопочной мыши без
использования клавиш модификаторов; кликнете по образу,
чтобы открыть меню Главное меню (\figref{threeButtons:click}). 
 
% \item [\actclick:] this is the next most used button; it is used to bring up a
% contextual menu, that is, a menu that offers different sets of actions
% depending on where the mouse is pointing; see
% \figref{threeButtons:actclick}. If you do not have a multi-button mouse,
% then normally you will configure the \emph{control} modifier key to
% \actclick with the mouse button.

\item[действие-клик:] это следующая по популярности кнопка;
она используется для создания контекстного меню,
т.е. меню, которое предлагает различные наборы
действий в зависимости оттого, на что указывает мышь.
См. \figref{threeButtons:actclick}. Если Вы работаете не с
многокнопочной (с тремя кнопками) мышью, то Вам следует настроить
модификатор-клавишу Ctrl, как если бы
это был щелчок-действие кнопкой мыши. 
 
% \item [\metaclick:] Finally, you may \metaclick on any object displayed in the
% image to activate the ``morphic \subind{Morphic}{halo}'', an array of
% handles that are used to perform operations on the on-screen objects
% themselves, such as rotating them or resizing them; see
% \figref{threeButtons:metaclick}.\footnote{Note that the morphic handles are
% inactive by default in \pharo, but you can turn them on using the
%Preferences Browser, which we will see shortly.}
% If you let the mouse linger over a handle,
% a help balloon will explain its function.
% In \pharo, how you \metaclick depends on your operating system:
% either you must hold {\sc shift} \emph{ctrl} or {\sc shift} \emph{option}
% while clicking.

\item[мета-клик:] Наконец, Вы можете мета-щелкнуть по любому объекту,
отображаемому в образе для активации <<области морфа (morphic)>>,
множество ручек, которые используются для выполнения операций
вращения или изменения самих объектов, расположенных на экране.
См. \figref{threeButtons:metaclick}.\footnote{Обратите внимание,
что morphic-ручки по умолчанию неактивны,
но чтобы это исправить необходимо изменить Настройки Браузера,
что мы впоследствии  и увидим.}
Если Вы позволите мыши задержаться на ручке,
то откроется необходимая справка по функции.
В \pharo Ваш мета-клик зависит от Вашей операционной системы: либо Вы
должны нажать shift ctrl, либо shift.

% \ab{This makes it sound like either {\sc shift} \emph{ctrl} or {\sc shift} \emph{alt} will work.  On my (Mac OS) system, only the latter works.  Perhaps we want to say: In \pharo, how you meta-click depends on your operating system. On Linux \ldots}
% Typically you will use a third modifier key, such as \emph{command} or \emph{meta} to \metaclick.
\end{description}

% \dothis{Type \ct{Time now} in the workspace.
% Now \actclick in the workspace.
% Select \menu{print it}.}

\dothis{Напечатайте \ct{Time now} в рабочей области.
Далее действие-клик в рабочей обасти. Выберите \menu{print it}.}

%Now we will activate \metaclick{ing}.

%\dothis{Open the preference browser (\menu{System {\ldots\go} Preferences {\ldots\go} Preference Browser\ldots}) and find the \menu{halosEnabled} option using the search box.}

%\begin{figure}[htb]
%\centerline{\includegraphics[width=\textwidth]{PreferenceBrowser}}
%\caption{The Preference Browser.\figlabel{prefBrowser}}
%\end{figure}

%\dothis{Now you should be able to \metaclick on the workspace. (See \figref{threeButtons:metaclick}.)
%Grab the blue \raisebox{-0.4ex}{\includegraphics[width=1em]{morphicRotate}} handle near the bottom left corner and drag it to rotate the workspace.}

% We recommend that right-handed people configure their mouse to
% \click with the left button, \actclick with the right button, and \metaclick with
% the  clickable scroll wheel, if one is available.

%%% If you don't have a clickable scroll wheel, then you can get the Morphic
%%% halo by holding down the \ct{alt} or \ct{option} key while \click{ing}. 
%%% \ab{This doesn't work any more.  This sentence either repeats or
%%% contradicts the meta-click item above; neither is a good idea.}

% If you are using a Macintosh without a second mouse button, you can
% simulate one by holding down the \clover{} key while \click{ing} the mouse.
% However, if you are going to be using \pharo at all often, we recommend
% investing in a mouse with at least two buttons.

Правшам можно рекомендовать настраивать свои мыши так: щелчок
осуществлять левой кнопкой мыши, щелчок-действие\,---\,правой,
а meta-click\,---\,колесиком прокрутки.
Если вы используете Macintosh без второй кнопки мыши,
то можете удерживать нажатой клавишу и кликать мышкой.
Однако мы еще раз советуем Вам приобрести
мышь, по крайней мере, с двумя кнопками.

% You can configure your mouse to work the way you want by using the
% preferences of your operating system and mouse driver.
% \ab{How can I get meta-click without a three-finger salute?  Is this a
% secret?}
% \pharo has some preferences for customising the mouse and the meta keys
% on your keyboard.
%In the preference browser (\menu{System {\ldots\go} Preferences
% {\ldots\go} Preference Browser\ldots}), the \menu{keyboard} category
% contains an option \menu{swapControlAndAltKeys} that switches the
% \actclick and \metaclick functions.
%There are also options to duplicate the various command keys.

Вы можете самостоятельно настроить конфигурацию своей мыши,
но и у Pharo есть свои настройки мыши и мета-клавиш на клавиатуре.
В настройках браузера \menu{System {\ldots\go} Preferences {\ldots\go}
Preference Browser {\ldots}}, категория \menu{keyboard} содержит
опцию \menu{swapControlAndAltKeys}, которая переключает
действие-клик и мета-клик функции. Также есть опции, дублирующие
варианты командных клавиш. 

\begin{figure}[htb]
\centerline{\includegraphics[width=\textwidth]{PreferenceBrowser}}
%\caption{The Preference Browser.\figlabel{prefBrowser}}
\caption{Окно настроек браузера.\figlabel{prefBrowser}}
\end{figure}


%=================================================================
% \section{The World menu}
% \index{world menu}

\section{Главное меню}
\index{главное меню}

% \dothis{\Click again on the \pharo background.}
% You will see the \menu{World} menu again.
% Most \pharo menus are not modal; you can leave them on the screen
% for as long as you wish by \click{ing} the push pin icon in the top-right
% corner. Do this.

%% Also, notice that menus appear when you click the mouse, but do not
%% disappear when you release it; they stay visible until you make a
%% selection,
%% or until you click outside of the menu. You can even move the menu
%% around by grabbing its title bar.

\dothis{Кникните еще раз по фону \pharo}
Вы снова увидите World-меню. Большинство \pharo-меню
не являются модальными; вы можете оставить меню на экране
ровно столько, сколько вам понадобится, для этого нажмите на
красную иконку в верхнем правом углу. Проделайте это. 

%The world menu provides you a simple means to access many of
%the tools that \pharo offers.

World-меню предоставляет вам простые способы (средства) для доступа ко
многим инструментам, которые может предложить \pharo.

% \dothis{Have a closer look at the \menu{World} and \menu{{}Tools \ldots}
% menus. (\figref{threeButtons:click})}

\dothis{Подробнее ознакомьтесь с \menu{World} и \menu{{}Tools \ldots}
меню. См. \figref{threeButtons:click}}

% You will see a list of several of the core tools in \pharo, including the
% browser and the workspace.
% We will encounter most of them in the coming chapters.

Вы увидите список некоторых основных инструментов \pharo,
включая браузер и рабочее пространство. Мы столкнёмся с большей
частью инструментария в последующих главах.

%=================================================================
% \section{Sending messages}
\section{Посылка сообщений}
\index{посылка сообщений}

% \dothis{Open a workspace. Type in the following text:}

\dothis{Откройте рабочую область. Напечатайте текст:}

\begin{code}{}
BouncingAtomsMorph new openInWorld
\end{code}

%\dothis{Now \actclick. A menu should appear. Select \menu{do it (d)}.
% (See \figref{doit}.)}

\dothis{Далее клик-действие. Должно появиться меню. Выберите
\menu{do it (d)}. См. \figref{doit}.}

\begin{figure}[htb]
\centerline {\includegraphics[width=0.8\textwidth]{Doit}}
% \caption{``Doing'' an expression\figlabel{doit}}
\caption{<<Выполнение>> выражения\figlabel{doit}}
\end{figure}

% A window containing a large number of bouncing atoms should open
% in the top left of the \pharo image.

Окно, содержащее большое количество bouncing atoms,
должно открываться в левом верхнем углу образа (image) Pharo. 

% You have just evaluated your first \st expression!
% You just sent the message \ct{new} to the \bam class, resulting
%in a new \bam instance, followed by the message \ct{openInWorld}
%to this instance.
% The \bam class decided what to do with the \ct{new} message, that is,
% it looked up its \emph{methods} for handling \ct{new} message and
% reacted appropriately.
% Similarly the \bam instance looked up its method for responding to
% \ct{openInWorld} and took appropriate action.

Вы только что вычислили свое первое Smalltalk-выражение!
Вы отправили сообщение \ct{new} классу BouncingAtomsMorph,
обработанное в новом экземпляре класса BouncingAtomsMorph,
затем следует сообщение \ct{openInWorld} к этому экземпляру. Сам класс
BouncingAtomsMorph решает, что делать с сообщением \ct{new}, то есть,
класс ищет метод для обработки сообщения \ct{new} и реагирует
нужным образом. Похожим образом экземпляр BouncingAtomsMorph ищет
метод для ответа на \ct{openInWorld} и выполняет
соответствующие действия. 

% If you talk to Smalltalkers for a while, you will quickly notice that they
% generally do not use expressions like ``call an operation'' or ``invoke a
% method'', but instead they will say ``send a message''.
% This reflects the idea that objects are responsible for their own actions. 
% You never \emph{tell} an object what to do\,---\,instead you politely
% \emph{ask} it to do something by sending it a message. 
%The object, not you, selects the appropriate method for responding
% to your message.

Если Вы немного пообщаетесь со смолтокерами, вы быстро заметите,
что они, в общем,  не используют такие выражения как
<<вызов операции>> или <<вызвать метод>>, но вместо этого
они будут говорить <<послать сообщение>>. Это отражает идею, что
объекты несут ответственность за свои действия.
Вы никогда не скажите объекту что делать, вместо этого вы вежливо
попросите сам объект что-либо сделать, посылая ему сообщение.
И объект, а никак не Вы, выбирает подходящий метод для ответа
на Ваше сообщение.

%=================================================================
% \section{Saving, quitting and restarting a \pharo session}
\section{Сохрание, выход и перезагрузка \pharo-сессии}

%\dothis{Now \click on the bouncing atoms window and drag it anywhere
% you like. You now have the demo ``in hand''. Put it down by \click{ing}
% anywhere.}

\dothis{Теперь кликните на bouncing atoms окна и перетащите
в любое место, которое Вам по нраву. И сейчас у Вас есть демонстрация
<<in hand>>.  Поместите ее, просто где-нибудь кликнув.}

\begin{figure}[htb]
\begin{minipage}[b]{0.48\textwidth}
\centerline {\includegraphics[width=0.7\textwidth]{atoms}}
% \caption{A \bam.\figlabel{atoms}}
\caption{Атомы.\figlabel{atoms}}
\end{minipage}
\hfill
\begin{minipage}[b]{0.48\textwidth}
\centerline {\includegraphics[width=0.7\textwidth]{saveAs}}
%\caption{The \menu{save as \ldots} dialogue.\figlabel{saveas}}
\caption{Окно диалога меню \menu{save as \ldots}.\figlabel{saves}}
\end{minipage}
\end{figure}

%\dothis{Select \menu{World\go{}Save as \ldots}, enter the name
% ``myPharo'', and \click on the \button{OK} button.
% Now select \menu{World\go{}Save and quit}.}

\dothis{Выберите \menu{World\go{}Save as\ldots}, введите название
<<myPharo>>, и нажмите на кнопку \button{OK}.
Затем выберите \menu{World\go{}Save and quit}.}

% Now if you go to the location where the original image and changes
% files were, you will find two new files called ``myPharo.\ind{image}''
% and ``myPharo.\ind{changes}'' that represent the working state of the
% \pharo image at the moment before you told \pharo to \menu{Save and
% quit}.
% If you wish, you can move these two files anywhere that you like on your
% disk, but if you do so you may (depending on your operating system
% need to also move, copy or link to the virtual machine and the
% \emph{sources} file.

Если Вы перейдете в то место, где у вас хранится образ
и там измените файлы, вы обнаружите два новых файла
с названиями <<myPharo.image>> и <<myPharo.changes>>,
которое отражает работающее состояние образа Pharo в момент перед
тем, как Вы выполнили команду в \pharo \menu{Save and quit}.
Если Вам захочется, то  можете переместить эти два файла
в любое место на Вашем диске, однако, если сделаете это,
то может понадобиться  (в зависимости от Вашей операционной системы)
также перемещать, копировать или линковаться к виртуальной машине
и исходным файлам (\emph{sources}). 

% \dothis{Start up \pharo from the newly created ``myPharo.image'' file.}

\dothis{Запустите Pharo из только что созданного файла
<<myPharo.image>>.}

% Now you should find yourself in precisely the state you were when you
% quit \pharo. The \bam is there again and the atoms continue to bounce
% from where they were when you quit.

Теперь точно представьте себе то, что было перед выходом из \pharo.
BouncingAtomsMorph снова перед Вами и атомы продолжают двигаться
с того же места как и было до того как  нажали кнопку выхода.

% When you start \pharo for the first time, the \pharo \ind{virtual machine}
% loads the image file that you provide. This file contains a snapshot of a
% large number of objects, including a vast amount of pre-existing code and
% a large number of programming tools (all of which are objects). As you
% work with \pharo, you will send messages to these objects, you will create
% new objects, and some of these objects will die and their memory will be
% reclaimed (\ie garbage-collected).

Когда вы запускаете Pharo в первый раз, виртуальная машина \pharo
загружает image файл, который вы используете. Этот файл содержит
снимок большого количества объектов, включая огромное количество уже
существующего кода и большое количество программных инструментов
(каждый из которых является объектом).  Поскольку Вы работаете с
\pharo, то будете посылать сообщения этим объектом, создавать новые
объекты, а некоторые из этих объектов умрут и память, которая была
использована для них, будет освобождена (т.е. так называемое
собирание мусора\,---\,garbage-collected). 

% When you quit \pharo, you will normally save a snapshot that contains all
% of your objects.  If you save normally, you will overwrite your old image
% file with the new snapshot. Alternatively, you may save the image unde
% a new name, as we just did.

Когда Вы выходите из \pharo, вы в обычном порядке сохраняете снимок,
который содержит все Ваши объекты. Если нормально сохранили,
то перезаписывается Ваш старый image file новым снимком.
Или, по-другому,
можете сохранить образ под новым именем, ровно так,
как мы только что делали. 

% In addition to the \emph{.image} file, there is also a \emph{.changes} file.
% This file contains a log of all the changes to the source code that you
% have made using the standard tools.
% Most of the time you do not need to worry about this file at all.
%As we shall see, however, the \emph{.changes} file can be very useful for
% recovering from errors, or replaying lost changes.
% More about this later!

В дополнение к \emph{.image} файлу так же существует \emph{.change}
файл. Этот файл содержит журнал (log) всех изменений с исходным
кодом, который Вы создали, используя стандартные инструменты.
Большую часть времени Вам не потребуется следить за этим файлом
совсем. Но как мы увидим, как бы то ни было, \emph{.changes} файл
может быть очень полезен для отладки или восстановления
потерянных изменений. Однако, подробнее об том позже.

% The image that you have been working with is a descendant of the original
% \st-80 image created in the late 1970s.
% Some of these objects have been around for decades!

Образ, с которым Вы сейчас непосредственно работаете,
является потомком оригинального образа \st-80,
созданного в конце 70-ых годов. Некоторые из теперешних
объектов существуют на протяжении десятилетий!

% You might think that the image is the key mechanism for storing and
% managing software projects, but you would be wrong.
% As we shall see very soon, there are much better tools for managing code
% and sharing software developed by teams.
% Images are very useful, but you should learn to be very cavalier about
% creating and throwing away images, since tools like Monticello offer much
% better ways to manage versions and share code amongst developers.

Вы можете подумать, что образ\,---\,это ключевой механизм для
хранения и управления программными проектами, но вы ошибаетесь.
Как мы сможем очень скоро увидеть, есть намного лучшие инструменты
для управления кодом и совместного использования программного
обеспечивания, разработанные командами. Образы очень полезны, но Вы
должны научиться быть крайне непринуждёнными в создании и удалении
образов, ровно с тех самых пор, когда инструменты по типу Monticello
предлагают намного более лучшие пути для управления версиями и
разделением кода между разработчиками. 

% \dothis{Using the mouse (and the appropriate modifier keys), \metaclick
% on the \bam.\footnote{Remember, you may have to set the
% \ct{halosEnabled} option in the Preferences Browser.}}
% You will see a collection of colored circles that are collectively called the
% \bam's morphic \subind{Morphic}{halo}.
% Each circle is called a \emph{handle}.
% Click in the pink handle containing the cross; the \bam should go away. 

\dothis{Используйте мышку (и соответствующие клавиши-модфикаторы),
мета-кликните на BouncingAtomsMorph\footnote{Запомните, Вы
можете выбрать \ct{halosEnabled} (разрешить нимб у морфа)}}
Вы увидите целую коллекцию разноцветных кружков,
которые все вместе называются morphic halo.
Каждый кружок называется <<ручкой>>. Клик на розовую ручку
содержит в себе крест; BouncingAtomsMorph будет свернут. 

%=================================================================
% \section{Workspaces and Transcripts}
% \seclabel{transcript}

\section{Рабочая область и Транскрипт}
\seclabel{transcript}

% \dothis{Close all open windows. Open a \ind{transcript} and a
% \ind{workspace}. (The transcript can be opened from the
% \menu{World{\go}Tools ...} submenu.)}

\dothis{Закройте все открытые окна. Откройте \ind{transcript} 
и \ind{workspace}. Transcript можно открыть через
\menu{World{\go}Tools ... }}

% \dothis{Position and resize the transcript and workspace windows so
% that the workspace just overlaps the transcript.}
% You can resize windows either by dragging one of the corners, or by
% \metaclick{ing} the window to bring up the morphic halo, and dragging
% the yellow (bottom right) handle.

\dothis{Установите (измените размеры) transcript и workspace окон  так,
чтобы рабочая область перекрывала transcript.}
Вы можете изменить размеры окна  либо потянув за один из углов,
либо воспользоваться  мета-кликом, чтобы вызвать morphic halo
и уже на нём потянуть  желтую <<ручку>> правой кнопкой. 

% At any time only one window is active; it is in front and has its border
% highlighted.
%% The mouse cursor must be in the window in which you wish to type.

В любое время активно только одно окно; оно находится
на переднем плане и имеет выделенные границы. 

% The transcript is an object that is often used for logging system messages.
% It is a kind of ``system console''.

%%Note that the transcript is terribly slow, so if you keep it open and write
%% to it certain operations can become 10 times slower.
%%In addition the transcript is not thread-safe so you may experience
%% strange problems if multiple objects write concurrently to the transcript.
%% ON: I think the transcript has been made thread-safe now, right?

Transcript\,---\,это объект, который часто используется для регистрации
системных сообщений, своего рода, <<системная консоль>>. 

% Workspaces are useful for typing snippets of \st code that you would
% like to experiment with.
% You can also use workspaces simply for typing arbitrarily text that you
% would like to remember, such as to-do lists or instructions for anyone
% who will use your image.
% Workspaces are often used to hold documentation about a captured image,
% as is the case with the standard image that we downloaded earlier
% (see \figref{startup}).

Рабочая область полезна для набора фрагментов кода на  \st,
с которыми впоследствии вы  бы хотели поэкспериментировать.
Вы можете в этой рабочей области просто набирать произвольный текст,
который хотите не забыть, например, список дел или инструкций для
того, кто потом будет пользоваться вашим образом \pharo.
Рабочие области часто используют для документации снимков образов,
как и в случае стандартного образа, который мы загрузили ранее.
(См. \figref{startup}).

% \dothis{Type the following text into the workspace:}
\dothis{Напечатайте текст в рабочей области:}
\begin{code}{}
Transcript show: 'hello world'; cr.
\end{code}

% Try double-\click{ing} in the workspace at various points in the text you
% have just typed.
% Notice how an entire word, entire string, or the whole text is selected,
% depending on whether you \click within a word, at the end of the string,
% or at the end of the entire expression.

Попробуйте дважды щелкнуть на рабочей области в различных местах
набранного текста. Стоит заметить, что будет выбрано слово,
строка или целый текст в зависимости от того где вы щелкнули:
рядом со словом, в конце строки или в конце целого текста. 

% \dothis{Select the text you have typed and \actclick.
% Select \menu{do it (d)}.}

\dothis{Выберите текст, который вы набрали только что и действие-click.
Выберите пункт подменю \menu{do it (d)}.}

% Notice how the text ``hello world'' appears in the transcript window
% (\figref{helloworld}).
% Do it again.
% (The \menu{(d)} in the menu item \menu{do it (d)} tells you that the
% keyboard shortcut to \emph{do it} is \short{d}. More on this in the next
% section!)

Обратите внимание, как два слова <<hello world>> появляются
в transcript-окне. (\figref{helloworld}) Проделайте это снова.
(Буква \menu{(d)}, стоящая рядом с \emph{do it(d)} в подменю
говорит о том, что горячая клавиша \emph{do it}\,---\,это \short{d}.
Подробнее об этом в следующем разделе!) 

\begin{figure}[htb]
\centerline {\includegraphics[width=\textwidth]{HelloWorld}}
%\caption{Overlapping windows. The workspace is active.
% \figlabel{helloworld}}
\caption{Частично перекрытые окна. Активно окно рабочей
области. \figlabel{helloworld}}
\end{figure}

%=================================================================
% \section{Keyboard shortcuts}
\section{Клавиши быстрого вызова}

% If you want  to evaluate an expression, you do not always have to \actclick.
% Instead, you can use \ind{keyboard shortcuts}. These are the parenthesized
%  expressions in the menu.  Depending on your platform, you may have to
% press one of the modifier keys (control, alt, command, or meta).
% (We will indicate these generically as \short{\emph{key}}.)

Для выполнения выражений вам не всегда нужно использовать
действие-клик. Вместо этого, вы можете воспользоваться
<<горячими клавишами>> (клавишами быстрого вызова).
Это зарезервированные выражения в меню. В зависимости от
вашей платформы, вы можете нажать одну из клавиш-модификаторов
(Ctrl, Alt, command, или meta). (Мы будем
указывать их в общем как \short{\emph{key}}/CMD-ключ). 

% \dothis{Evaluate the expression in the workspace again, but using the
% keyboard shortcut: \short{d}.}
% \index{keyboard shortcut!do it}

\dothis{Теперь снова вычислите выражение, но уже с
использованием горячей клавиши \short{d}.}
\index{горячие клавиши!do it}

% In addition to \menu{do it}, you will have noticed \menu{print it},
% \menu{inspect it} and \menu{explore it}. Let's have a quick look
% at each of these.

Перед тем, как это сделать, обратите внимание на \menu{print it},
\menu{inspect it} и \menu{explore it}.
Давайте рассмотрим каждое из этих действий.

% \dothis{Type the expression \ct{3 + 4} into the workspace.
% Now \menu{do it} with the keyboard shortcut.}

\dothis{Наберите выражение \ct{3 + 4} в рабочей области.
Теперь сделайте это с помощью горячих клавиш.}

% Do not be surprised if you saw nothing happen! What you just did is send
% the message \ct{+} with argument \ct{4} to the number \ct{3}. Normally
% the result \ct{7} will have been computed and returned to you, but since
% the workspace did not know what to do with this answer, it simply threw
% the answer away.  If you want to see the result, you should \menu{print it}
% instead. \menu{print it} actually compiles the expression, executes it, sends
% the message \ct{printString} to the result, and displays the resulting string.

Не удивляйтесь, если увидите, что ничего не произошло!
Вы только что послали сообщение \ct{+} с аргументом \ct{4} номеру
\ct{3}. Обычно результат 7 вычисляется и возвращается вам,
но так как рабочая область попросту не знает, что с ним делать
дальше,и отбрасывает его. Если захотите увидеть результат,
распечатайте его.\menu{Print it} 
обычно компилирует выражение, выполняет его, посылает
сообщение \ct{printString} результату и отображает
результирующую строку.

% \dothis{Select \ct{3+4} and \menu{print it} (\short{p}).}
% This time we see the result we expect (\figref{printit}).
% \index{keyboard shortcut!print it}

\dothis{Выделите \ct{3+4} и выполните \menu{print it} (\short{p}).}
На этот раз мы увидим ожидаемый результат (\figref{printit}).
\index{горячие клавиши!print it}

\begin{figure}[htb]
% \centerline {\includegraphics[width=0.4\textwidth]{PrintIt}}
\centerline {\includegraphics[width=0.8\textwidth]{PrintIt}}
% \caption{``Print it'' rather than ``do it''. \figlabel{printit}}
\caption{<<Print it>> лучше чем <<do it>>.\figlabel{printit}}
\end{figure}

\needlines{3}
\begin{code}{@TEST}
3 + 4 --> 7
\end{code}

%\noindent
%We use the notation \ct{-->} as a convention in this book to indicate tha
% a particular \pharo expression yields a given result when you
% \menu{print it}.

\noindent
Мы используем обозначение \ct{-->},
чтобы показать, к чему приведет \pharo-выражение после того,
как вы получите результат, применяя \menu{print it}. 

% \dothis{Delete the highlighted text ``\ct{7}'' (\pharo should have selected
% it for you, so you can just press the delete key). Select \ct{3+4} again and
% this time \menu{inspect it} (\short{i}).}

\dothis{Удалите выделенный текст <<\ct{7}>> (\pharo должен сделать
 это за вас так, чтобы вы смогли потом нажать только delete).
Наберите 3 + 4 снова и выберите \menu{inspect it} (\short{i}).}

% \noindent
% Now you should see a new window, called an \emphind{inspector},
% with the heading \ct{SmallInteger: 7} (\figref{inspectit}).
% The inspector is an extremely useful tool that will allow you to browse and
% interact with any object in the system.
% The title tells us that \ct{7} is an instance of the class \clsind{SmallInteger}.
% The left panel allows us to browse the instance variables of an object,
% the values of which are shown in the right panel.
% The bottom panel can be used to write expressions to send messages
% to the object.

\noindent
После этого появится новое окно, которое называется инспектор,
с заголовком SmallInteger: 7 (См. \figref{inspectit}).
Инспектор\,---\,очень полезный инструмент, который позволит
вам просматривать и взаимодействовать с любым объектом в системе.
Название говорит нам о том, что 7\,---\,это экземпляр класса SmallInteger.
Левая панель предоставляет нам возможность просматривать переменные
экземпляра объекта/класса/, значения которых выведены на правой
панели. Нижняя панель может быть  использована  для написания
выражений чтобы отправить их потом объекту.

\begin{figure}[htb]
\centerline {\includegraphics[width=\textwidth]{InspectIt}}
% \caption{Inspecting an object. \figlabel{inspectit}}
\caption{Инспектирование объекта. \figlabel{inspectit}}
\end{figure}

% \dothis{Type \ct{self squared} in the bottom panel of the inspector on
% \ct{7} and \menu{print it}.}

\dothis{Наберите \ct{self squared} в нижней панели инспектора на \ct{7}
и выберите в меню \menu{print it}}

% \needlines{2}
% \dothis{Close the inspector. Type the expression \ct{Object} in
% a workspace and this time \menu{explore it} (\short{I}, uppercased i).}
% \index{keyboard shortcut!explore it}
% \index{explorer}

\needlines{2}
\dothis{Закройте инспектор. Наберите выражение \ct{Object} в рабочей
области и выберите \menu{explore it} (\short{I}, i в верхнем регистре).}
\index{горячие клавиши!explore it}
\index{explorer}

% This time you should see a window labelled \clsind{Object}
% containing the text
% \mbox{$\triangleright$ \ct{root: Object}}.
% Click on the triangle to open it up (\figref{exploreit}).

В этот раз вы увидите окно, помеченное Object,
содержащее текст \mbox{$\triangleright$ \ct{root: Object}}.
Кликнете по треугольнику, чтобы раскрыть его (См. \figref{exploreit}).

\begin{figure}[htb]
\centerline {\includegraphics[width=0.7\textwidth]{ExploreIt}}
% \caption{Exploring \ct{Object}. \figlabel{exploreit}}
\caption{Обзор \ct{Object}. \figlabel{exploreit}}
\end{figure}

% The explorer is similar to the inspector, but it offers a tree view of a
% complex object.
% In this case the object we are looking at is the \ct{Object} class.
% We can see directly all the information stored in this class, and we
% can easily navigate to all its parts.

Explorer похож на inspector, за исключением древовидной структуры
сложного объекта. В этом случае, объект, который мы наблюдаем,
это класс Object. Мы можем видеть всю информацию, которая хранится
в этом классе и легко переходить ко всем его частям.

%=================================================================
% \section{The Class Browser}
\section{Браузер классов}

% The class \ind{browser}\footnote{Confusingly, this is variously referred to
% as the ``system browser'' or the ``code browser''. \pharo uses the
% \ind{OmniBrowser} implementation of the browser, which may also be
% variously known as ``OB'' or the ``Package browser''.
% In this book we will simply use the term ``browser'', or, in case of
% ambiguity, the ``class browser''.} is one of the key tools used for
% programming. As we shall see, there are several interesting browsers
% available for \pharo, but this is the basic one you will find in any image.
% \seeindex{class browser}{browser}

Браузер классов\footnote{Почему-то это наименование может
употребляться как <<системный браузер>> или как <<кодовый
браузер>>.
В Pharo используется OmniBrowser, который может быть известен под
названием <<OB>> или <<Pachage browser>>. В этой книге мы просто
используем термин <<браузер>> или, в случае двусмысленности,
<<браузер классов>>.}\,---\,один из ключевых инструментов,
используемых в
программировании. Как мы увидим далее, несколько интересных
браузеров доступны в  Pharo, но это основной, который вы найдёте в
любом образе.
\seeindex{браузер классов}{браузер}
 

% \dothis{Open a browser by selecting \menu{World \go Class browser}.
% \footnote{If the browser you get does not look like the one shown in
% \figref{classBrowser}, then you may need to change the default browser. 
% See \faqref{packagebrowser}.}}

\dothis{Откройте браузер и выберите \menu{World \go Class browser.}
\footnote{Если вид у браузера будет отличаться от \figref{classBrowser},
в таком случае нужно изменить браузер по умолчанию. Смотрите
\faqref{packagebrowser}.}}

\begin{figure}[htb]
\ifluluelse
	{\centerline {\includegraphics[width=\textwidth]{ClassBrowser1}}}
	{\centerline {\includegraphics[width=0.7\textwidth]{ClassBrowser1}}}
% \caption{The browser showing the \ct{printString} method of class object.
\caption{Браузер демонстрирует метод printString класса Object.
\figlabel{classBrowser}}
\end{figure}

% We can see a browser in \figref{classBrowser}.
% The title bar indicates that we are browsing the class \clsind{Object}.
%% \footnote{If the browser you have seems to differ from the one described
%% in this book, you may be using an image with a different default browser.
%% See \faqref{omnibrowser}.}

Мы можем видеть браузер на \figref{classBrowser} Заголовок окна
указывает, что мы просматриваем класс \clsind{Object}. 

% When the browser first opens, all panes are empty but the leftmost one.
% This first pane lists all known \emph{packages}, which contain groups of
% related classes.
% \index{category}

При первом открытии браузера все панели пусты, за исключением
одной\,---\,крайней левой. Эта первая панель перечисляет все известные
пакеты, которые содержат группы связанных классов.
\index{category}
 

% \dothis{Click on the \scatind{Kernel} package.}
% This causes the second pane to show a list of all of the classes in the
% selected package.

\dothis{Кликнете по  Kernel пакету.}
Вторая панель покажет нам список всех классов выбранного пакета. 

% \dothis{Select the class \clsind{Object}.}
% Now the remaining two panes will be filled with text.
% The third pane displays the \emph{protocols} of the currently selected
% class. These are convenient groupings of related methods.
% If no \ind{protocol} is selected you should see all methods in
% the fourth pane.

\dothis{Выберите класс Object.}
Теперь в оставшихся панелях появится текст. Третья панель выводит
протоколы текущего выбранного класса. Это удобно при группировке
связанных методов. Если вы не выбрали ни одного протокола, то
четвертая панель покажет все методы. 

% \dothis{Select the \protind{printing} protocol.}
% You may have to scroll down to find it.
% Now you will see in the fourth pane only methods related to printing.

\dothis{Выберите протокол printing.}
Вы можете прокрутить вниз для его поиска.  Теперь вы будете видеть в
четвертом блоке только те методы, которые связаны с печатью. 

% \dothis{Select the \mthind{Object}{printString} method.}
% Now we see in the bottom pane the source code of the \ct{printString}
% method, shared by all objects in the system (except those that override it).

\dothis{Выберите метод printString.}
Теперь вы увидите на нижней панели исходный код метода printString,
используемый всеми объектами системы (кроме тех, которые
переопределили его). 

%=================================================================
% \section{Finding classes}
\section{Поиск классов}

% There are several ways to find a class in \pharo.  The first, as we have just
% seen above, is to know (or guess) what category it is in, and to navigate
% to it using the browser.
% \index{browser}
% \seeindex{browser!finding classes}{class, finding}
% \index{class!finding}

Есть несколько способов найти класс в Pharo. Во-первых, как мы видели
только что выше, нужно знать (или предполагать), к какой категории он
принадлежит и ориентироваться при помощи браузера.
\index{браузер}
\seeindex{браузер!поиск классов}{класс, поиск}
\index{класс!поиск}

% A second way is to send the \ct{browse} message to the class, asking it to
% open a browser on itself.  Suppose we want to browse the class
% \clsind{Boolean}.

Второй способ заключается в том, чтобы послать browse сообщение
классу, попросив при этом открыть результат в браузере.
Предположим, что мы хотим просмотреть класс Boolean. 

% \dothis{Type \ct{Boolean browse} into a workspace and \menu{do it}.}
% A browser will open on the Boolean class (\figref{browseBoolean}).
% There is also a \ind{keyboard shortcut} \short{b} (browse) that you can
% use in any tool where you find a class name;
% \index{keyboard shortcut!browse it}
% select the name and type \short{b}.

\dothis{Наберите \ct{Boolean browse} в рабочей области и \menu{do it}.}
Браузер откроет класс Boolean (figref{browseBoolean}).
Есть также горячая клавиша \short{b}(browser), которую можете
использовать в любом инструменте где вы найдете имя класса; выберите имя и наберите \short{b}. 
\index{Горячие клавиши!browse it}


% \dothis{Use the keyboard shortcut to browse the class \ct{Boolean}.}

\dothis{Используйте горячую клавишу, чтобы просмотреть класс Boolean.}

\begin{figure}[hbt]
\centerline {\includegraphics[width=\textwidth]{Kernel-objects-boolean}}
% \caption{The browser showing the definition of class Boolean.
% \figlabel{browseBoolean}}
\caption{Браузер отображает содержимое класса Boolean
\figlabel{browseBoolean}}
\end{figure}

% Notice that when the \ct{Boolean} class is selected but no protocol or
% method is selected, instead of the source code of a method, we see a
% \emph{class definition} (\figref{browseBoolean}).
% This is nothing more than an ordinary \st message that is sent to the parent
% class, asking it to create a subclass.
% Here we see that the class \ct{Object} is being asked to create a subclass
% named \ct{Boolean} with no instance variables, class variables or ``pool
% dictionaries'', and to put the class \ct{Boolean} in the
% \scatind{Kernel-Objects} category.
%% The lower pane shows the \emph{class comment} --- a piece of plain
%% text describing the class.
% If you \click on the \button{?} at the bottom of the class pane, you can see
% the class \subind{class}{comment} in a dedicated pane (see
% \figref{classComment}).

Заметим, что когда класс \ct{Boolean} класс выбран, а протоколы или
методы\,---\,нет, то вместо исходного кода метода мы видим
определение класса (\figref{browseBoolean}). Это не что иное как обычная
посылка Smalltalk-сообщения родительскому классу с просьбой создать
подкласс. Здесь мы видим, как  класс \ct{Object} просят создать подкласс
с именем \ct{Boolean} без переменных экземпляра, переменных класса или
<<общих словарей>> и поместить класс \ct{Boolean} в категорию
Kernel-Objects. Если вы кликнете по \button {?} в нижней части второй
панели (отвечающей за классы), то увидите комментарии к классу в
специальной области (смотрите \figref{classComment}) 

\begin{figure}[hbt]
\centerline {\includegraphics[width=\textwidth]{classComment}}
% \caption{The class comment for \ct{Boolean}.
\caption{Комментарий класса \ct{Boolean}.
\figlabel{classComment}}
\end{figure}

% Often, the fastest way to find a class is to search for it by name.
% For example, suppose that you are looking for some unknown class that
% represents dates and times.

Часто самый быстрый способ найти класс это искать его по имени.
Для примера, предположим, что вам нужно найти какой-либо известный
класс, представляющий даты и время. 

% \dothis{Put the mouse in the package pane of the browser and type
% \short{f}, or select \menu{find class \ldots (f)} by \actclick{ing}.  Type
% ``time'' in the dialog box and accept it.} 
% \noindent
% You will be presented with a list of classes whose names contain ``time''
% (see \figref{findit}).  Choose one, say, \ct{Time}, and the browser will
% show it, along with a class comment that suggests other classes that might
% be useful.  If you want to browse one of the others, select its name (in any
% text pane), and type \short{b}.
% \index{keyboard shortcut!find ...}
% \index{keyboard shortcut!browse it}

\dothis{Наведите мышь на пакет панель браузера и наберите \short{f}
или выберите \menu{find class \ldots (f)}.
Наберите <<time>> в диалоговом окне и accept it.}
\noindent
Вам будет предоставлен список классов, имена которых содержат
<<time>> (\figref{findit}). Выберите один, скажем, Time и браузер
покажет его наряду с другими возможными вариантами.
Если хотите просмотреть другой класс, выберите его имя,
и наберите \short{b}.
\index{горячая клавиша!найти...}
\index{горячая клавиша!browse it}
 

\begin{figure}[hbt]
\centerline{
	\includegraphics[width=0.45\textwidth]{FindIt}
	\hspace{1cm}
	\includegraphics[width=0.45\textwidth]{TimeClasses}
}
% \caption{Searching for a class by name.
\caption{Поиск классов по имени.
\figlabel{findit}}
\end{figure}

% Note that if you type the complete (and correctly capitalized) name of a
% class in the find dialog, the browser will go directly to that class without
% showing you the list of options.

Обратите внимание, что если вы наберете полное (и с заглавной буквы)
имя класса в диалоговом окне, то браузер перейдет непосредственно 
к этому классу, не показывая вам список вариантов. 

%=================================================================
% \section{Finding methods}
% \seclabel{quick:methodFinder}

\section{Поиск методов}
\seclabel{quick:methodFinder}

Sometimes you can guess the name of a method, or at least part of the name of a method, more easily than the name of a class.  For example, if you are interested in the current time, you might expect that there would be a method called ``now'', or containing ``now'' as a substring.   But where might it be?
The \emphind{method finder} can help you.
\seeindex{browser!finding methods}{method, finding}
\index{method!finding}

\dothis{Select \menu{World \go Tools ... \go Method finder}.
Type ``now'' in the top left pane, and \menu{accept} it (or just press the \textsc{return} key).}
The method finder will display a list of all the method names that contain the substring ``now''.  
To scroll to \ct{now} itself, move the cursor to the list and type ``\ct{n}''; this trick works in all scrolling windows.  Select ``now'' and the right-hand pane shows you the classes that define a method with this name, as shown in \figref{MethodFinder}.  Selecting any one of them will open a browser on it.

\begin{figure}[hbt]
\centerline {\includegraphics[width=0.7\textwidth]{methodFinder-now}}
\caption{The method finder showing all classes defining a method named \ct{now}.
\figlabel{MethodFinder}}
\end{figure}

At other times you may have a good idea that a method exists, but will have no idea what it might be called.
The method finder can still help!  For example, suppose that you would like to find a method that turns a string into upper case, for example, it would translate \ct{'eureka'} into \ct{'EUREKA'}.

\dothis{Type \ct{'eureka' . 'EUREKA'} into the method finder and press
  the \textsc{return} key, as shown in
  \figref{methodFinder-example1}.}
\noindent
The method finder will suggest a method that does what you want.\footnote{If a window pops up with a warning about a deprecated method, don't panic --- the method finder is simply trying out all likely candidates, including deprecated methods. Just \click ~\button{Proceed}.}

An asterisk at the beginning of a line in the right pane of the method finder indicates that this method is the one that was actually used to obtain the requested result. 
So, the asterisk in front of \ct{String asUppercase} lets us know that the method \mthind{String}{asUppercase} defined on the class \clsind{String} was executed and returned the result we wanted. The methods that do not have an asterisk are just the other methods that have the same name as the ones that returned the expected result. So \cmind{Character}{asUppercase} was not executed on our example, because \ct{'eureka'} is not a \clsind{Character} object.

\begin{figure}[hbt]
\centerline {\includegraphics[width=\textwidth]{MethodFinder-example1}}
\caption{Finding a method by example.
\figlabel{methodFinder-example1}}
\end{figure}

You can also use the method finder for methods with arguments; for example, if you are looking for a method that will find the greatest common factor of two integers, you might try \ct{25. 35. 5} as an example.  You can also give the method finder multiple examples to narrow the search space; the help text in the bottom pane explains how.

%=================================================================
\section{Defining a new method}

The advent of \ind{Test Driven Development}\cite{Beck03a} (TDD) has changed the way that we write code.  
The idea behind TDD is that we write a test that defines the desired behaviour of our code \emph{before} we write the code itself.
Only then do we write the code that satisfies the test.
\seeindex{Behavior Driven Development}{Test Driven Development}
% \orla{describe the technique where we write a test hat ... subsequently we write ...}

Suppose that our assignment is to write a method that ``says something loudly and with emphasis''.  What exactly could that mean?  What would be a good name for such a method?  How can we make sure  that programmers who may have to maintain our method in the future have an unambiguous description of what it should do?   We can answer all of these questions by giving an example:

\begin{quote}
When we send the message \ct{shout} to the string ``Don't panic'' the result should be ``DON'T PANIC!''.
\end{quote}

\noindent
To make this example into something that the system can use, we turn it into a test method:
\index{testing}
\index{SUnit}

\needlines{3}
\begin{method}[testShout]{A test for a shout method}
testShout
	self assert: ('Don''t panic' shout = 'DON''T PANICBANG')
\end{method} % BANG is the escape for !

How do we create a new method in \pharo?   First, we have to decide which class the method should belong to.
In this case, the \ct{shout} method that we are testing will go in class \clsind{String}, so the corresponding test will, by convention, go in a class called \clsind{StringTest}.

\begin{figure}[hbt]
\centerline {\includegraphics[width=\textwidth]{StringTest-newMethodTemplate}}
\caption{The new method template in class \ct{StringTest}.
\figlabel{newMethodTemplate}}
\end{figure}

\dothis{Open a browser on the class \ct{StringTest}, and select an appropriate protocol for our method, in this case \menu{tests - converting}, as shown in \figref{newMethodTemplate}.
The highlighted text in the bottom pane is a template that reminds you what a \st method looks like.
Delete this and enter the code from  \mthref{testShout}.}
Once you have typed the text into the browser, notice that the bottom pane is outlined in red.  This is a reminder that the pane contains unsaved changes.
So select \menu{accept (s)} by \actclick{ing} in the bottom pane, or just type \short{s}, to compile and save your method.
\index{keyboard shortcuts}
\index{keyboard shortcut!accept}
\seeindex{accept it}{keyboard shortcut, accept}

If this is the first time you have accepted any code in your image, you will likely be prompted to enter your name. Since many people have contributed code to the image, it is important to keep track of everyone who creates or modifies methods. Simply enter your first and last names, without any spaces, or separated by a dot.

%\begin{figure}[hbt]
%\centerline {\includegraphics[width=0.35\textwidth]{initials}}
%\caption{Entering your initials.
%\figlabel{initials}}
%\end{figure}

Because there is as yet no method called \ct{shout}, the browser will ask you to confirm that this is the name that you really want\,---\,and it will suggest some other names that you might have intended (\figref{testShoutConfirm}).
This can be quite useful if you have merely made a typing mistake, but in this case, we really \emph{do} mean \ct{shout}, since that is the method we are about to create, so we have to confirm this by selecting the first option from the menu of choices, as shown in \figref{testShoutConfirm}. 


%\begin{figure}[htb]
%\begin{minipage}[b]{0.48\textwidth}
%\centerline {\includegraphics[width=0.9\textwidth]{name}}
%\caption{Entering your name.\figlabel{name}}
%\end{minipage}
%\hfill
%\begin{minipage}[b]{0.48\textwidth}
%\centerline {\includegraphics[width=\textwidth]{testShoutConfirm}}
%\caption{Accepting the \ct{StringTest} method \ct{testShout}.\figlabel{testShoutConfirm}}
%\end{minipage}
%\end{figure}

\begin{figure}[htb]
\centerline {\includegraphics[width=0.6\textwidth]{name}}
\caption{Entering your name.\figlabel{name}}
\end{figure}

\begin{figure}[htb]
\centerline {\includegraphics[width=\textwidth]{testShoutConfirm}}
\caption{Accepting the \ct{StringTest} method \ct{testShout}.\figlabel{testShoutConfirm}}
\end{figure}


%\begin{figure}[hbt]
%\ifluluelse
%	{\centerline {\includegraphics[width=\textwidth]{testShoutConfirm}}}
%	{\centerline {\includegraphics[width=0.7\textwidth]{testShoutConfirm}}}
%\caption{Accepting the \ct{StringTest} method \ct{testShout}.
%\figlabel{testShoutConfirm}}
%\end{figure}

\dothis{Run your newly created test: open the \ind{SUnit} \emphind{TestRunner} from the \menu{World} menu.}

The leftmost two panes are a bit like the top panes in the browser.  The left pane contains a list of categories, but it's restricted to those categories that contain test classes.

\dothis{Select \scat{CollectionsTests-Text} and the pane to the right will show all of the test classes in that category, which includes the class \ct{StringTest}.  The names of the classes are already selected, so \click \menu{Run Selected} to run all these tests.}

\begin{figure}[hbt]
\centerline {\includegraphics[width=\textwidth]{testRunnerOnStringTest}}
\caption{Running the String tests.
\figlabel{testRunnerTestShout}}
\end{figure}

You should see a message like that shown in \figref{testRunnerTestShout}, which indicates that there was an error in running the tests.  The list of tests that gave rise to errors is shown in the bottom right pane; as you can see, \ct{StringTest>>>#testShout} is the culprit.
(Note that \ct{StringTest>>#testShout} is the Smalltalk way of identifying the \mthind{StringTest}{testShout} method of the \ct{StringTest} class.)
If you \click on that line of text, the erroneous test will run again, this time in such a way that you see the error happen: ``\ct{MessageNotUnderstood: ByteString>>>shout}''.
\seeindex{\ct{>>}}{Behavior, \ct{>>}}
\cmindex{Behavior}{>>}

The window that opens with the error message is the \st debugger (see \figref{predebugger}).
% \ab{Well, it's actually the \emph{pre-}debugger.  Does this matter?}\damien{I don't think it's important at this point.}
We will look at the \ind{debugger} and how to use it in \charef{env}.

\begin{figure}[hbt]
\centerline {\includegraphics[width=\textwidth]{Predebugger}}
\caption{The (pre-)debugger.}
\figlabel{predebugger}
\end{figure}

The error is, of course, exactly what we expected:  running the test generates an error because we haven't yet written a method that tells strings how to \ct{shout}.  
Nevertheless, it's good practice to make sure that the test fails because this confirms that we have set up the testing machinery correctly and that the new test is actually being run.
Once you have seen the error, you can \button{Abandon} the running test, which will close the debugger window.
Note that often with Smalltalk you can define the missing method using the \button{Create} button, edit the newly-created method in the debugger, and then \button{Proceed} with the test.

Now let's define the method that will make the test succeed!

\dothis{Select class \clsind{String} in the browser, select the \menu{converting} protocol, type the text in \mthref{shout} over the method creation template, and \menu{accept} it.
(Note: to get a \mbox{\ct{^}}, type \caret). }
\begin{method}[shout]{The shout method}
shout
	^ self asUppercase, 'BANG'
\end{method}

The comma is the string concatenation operation, so the body of this method appends an exclamation mark to an upper-case version of whatever \ct{String} object the \ct{shout} message was sent to.
The $\uparrow$ tells \pharo that the expression that follows is the answer to be returned from the method, in this case the new concatenated string.
\seeindex{comma}{Collection, comma operator}
\index{Collection!comma operator}

Does this method work?  Let's run the tests and see.

\dothis{Click on \menu{Run Selected} again in the test runner, and this time you should see a green bar and text indicating that all of the tests ran with no failures and no errors.}
When you get to a green bar\footnotemark, it's a good idea to save your work and take a break.  
So do that right now!
%\footnotetext{Actually, you might not get a green bar since some images contains tests for bugs that still need to be fixed.
%Don't worry about this.
%\pharo is constantly evolving.
%}

\begin{figure}[hbt]
\ifluluelse
	{\centerline{\includegraphics[width=\textwidth]{String-Shout}}}
	{\centerline{\includegraphics[width=0.7\textwidth]{String-Shout}}}
\caption{The \ct{shout} method defined on class \ct{String}.
\figlabel{String-shout}}
\end{figure}

%=================================================================
\section{Chapter summary}
This chapter has introduced you to the \pharo environment and shown you how to use some of the major tools, such as the browser, the method finder, and the test runner.   You have also seen a little of \pharo's syntax, even though you may not understand it all yet.

\begin{itemize}
  \item A running \pharo system consists of a \emph{virtual machine}, a \emph{sources} file, and \emph{image} and \emph{changes} files. Only these last two change, as they record a snapshot of the running system.
  \item When you restore a \pharo image, you will find yourself in exactly the same state\,---\,with the same running objects\,---\,that you had when you last saved that image.
  \item \pharo is designed to work with a three-button mouse to \click, \actclick or \metaclick.  If you don't have a three-button mouse, you can use modifier keys to obtain the same effect.
  \item You \click on the \pharo background to bring up the \emph{World menu} and launch various tools.
  \item A \emph{workspace} is a tool for writing and evaluating snippets of code. You can also use it to store arbitrary text.
  \item You can use \ind{keyboard shortcuts} on text in the workspace, or any other tool, to evaluate code. The most important of these are \menu{do it} (\short{d}), \menu{print it} (\short{p}), \menu{inspect it} (\short{i}), \menu{explore it} (\short{I}) and \menu{browse it} (\short{b}).
%  \item \sqmap is a tool for loading useful packages from the Internet.
  \item The \emph{browser} is the main tool for browsing \pharo code, and for developing new code.
  \item The \emph{test runner} is a tool for running unit tests. It also supports Test Driven Development.
\end{itemize}

%=================================================================
\ifx\wholebook\relax\else 
   \bibliographystyle{jurabib}
   \nobibliography{scg}
   \end{document}
\fi
%=================================================================

%%% Local Variables:
%%% coding: utf-8
%%% mode: latex
%%% TeX-master: t
%%% TeX-PDF-mode: t
%%% ispell-local-dictionary: "english"
%%% End:
