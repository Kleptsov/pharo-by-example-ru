% $Author: oscar $
% $Date: 2009-09-18 15:57:20 +0600 (пт, 18 сен 2009) $
% $Revision: 29170 $

% HISTORY:
% 2007-03-30 - Cassou splits of Streams from Collection chapter
% 2007-07-05 - Cassou partial draft complete?
% 2007-08-02 - Stef pass
% 2007-08-16 - Cassou continues
% 2007-08-21 - Oscar edit
% 2007-08-21 - Cassou review
% 2009-07-07 - Oscar migrate to Pharo; fixed broken tests

%=================================================================
\ifx\wholebook\relax\else
% --------------------------------------------
% Lulu:
	\documentclass[a4paper,10pt,twoside]{book}
	\usepackage[
		papersize={6.13in,9.21in},
		hmargin={.75in,.75in},
		vmargin={.75in,1in},
		ignoreheadfoot
	]{geometry}
	\input{../common.tex}
	\pagestyle{headings}
	\setboolean{lulu}{true}
% --------------------------------------------
% A4:
%	\documentclass[a4paper,11pt,twoside]{book}
%	\input{../common.tex}
%	\usepackage{a4wide}
% --------------------------------------------
    \graphicspath{{figures/} {../figures/}}
	\begin{document}
	% \renewcommand{\nnbb}[2]{} % Disable editorial comments
	\sloppy
\fi
%=================================================================
%\chapter{Streams}\chalabel{streams}
\chapter{Потоки}\chalabel{streams}

%\ew{Streams are presented as a way to navigate collection. From my point of view, stream are important not to navigate collection, but to produce/consume data:
(a)	memory constraint. Data can not hold into memory and must be processed in a stream fashion, e.g: encryption
(b)	blocking IO. A stream is a nice abstraction to deal with, and the stream manages internally data availability, buffering, etc. to simplify the consumption/production of data
Only few streams have random access capability.}

\clsindexmain{Stream}
%Streams are used to iterate over sequences of elements such as sequenced collections, files, and network streams.
Потоки используются для итерации по элементам последовательностей, таких, как упорядоченные коллекции, файлы, \ugh{сетевые потоки ввода-вывода}.
%Streams may be either readable, or writeable, or both.
Потоки могут быть для чтения, для записи или для чтения и записи одновременно.
%Reading or writing is always relative to the current position in the stream.
Чтение и запись всегда производятся относительно текущей позиции в потоке.
%Streams can easily be converted to collections, and vice versa.
Потоки моут быть легко преобразованы в коллекции и наоборот.

\lr{"Streams can easily be converted into collections." I wouldn't say it like this, because it is not true for all streams (infinite streams). According to Kent Beck we should only talk about conversion when the same protocol is supported. Collections and Streams do not support the same protocol. (p. 249)}

%=============================================================
%\section{Two sequences of elements}
\section{Две последовательности элементов}
%A good metaphor to understand a stream is the following: A stream can be represented as two sequences of elements: a past element sequence and a future element sequence. 
Подходящей метафорой для потока будет следующая. Поток можно рассматривать как две последовательности элементов: \ugh{последовательность прошлых элементов и последовательность элеметнов будущих}.
%The stream is positioned between the two sequences. 
Поток расположен между двумя этими последовательностями.
%Understanding this model is important since all stream operations in Smalltalk rely on it.
Понимание этой модели важно, т.к. все операции над потоками в \st основываются на ней.
%For this reason, most of the \clsind{Stream} classes are subclasses of \clsind{PositionableStream}.
По этой причине, большинство классов потоков являются наследниками \clsind{PositionableStream}.
%\figref{_abcde} presents a stream which contains five characters. This stream is in its original position, \ie there is no element in the past. You can go back to this position using the message \mthind{PositionableStream}{reset}.
\figref{_abcde} изображён поток, состоящий из пяти букв. Этот поток находится в начальной позиции, \ie <<в прошлом>> нет элементов. Вы можете вернуться к этой позиции с помощью сообщения \mthind{PositionableStream}{reset}.

\begin{figure}[ht]
\centerline{\includegraphics[scale=0.5]{_abcdeStef}}
%\caption{A stream positioned at its beginning.}
\caption{Поток находится в начальной позиции.}
\figlabel{_abcde}
\vspace{.2in}
\end{figure}

%Reading an element conceptually means removing the first element of the future element sequence and putting it after the last element in the past element sequence. After having read one element using the message \ct{next}, the state of your stream is that shown in \figref{a_bcde}.
Чтение элемента из потока концептуально можно представить как удаление первого элемента <<будущей>> последовательности и добавление их в конец <<прошлой>> последовательности. На \figref{a_bcde} показано состояние потока после прочтения одного элемента с помощью сообщения \ct{next}.

\begin{figure}[ht]
\centerline{\includegraphics[scale=0.5]{a_bcdeStef}}
%\caption{The same stream after the execution of the method \ct{next}: the character \ct{a} is ``in the past'' whereas \ct{b}, \ct{c}, \ct{d} and \ct{e} are ``in the future''.}
\caption{Тот же поток после выполнения метода \ct{next}: буква \ct{a} <<в прошлом>>, а \ct{b}, \ct{c}, \ct{d} и \ct{e} -- <<в будущем>>.}
\figlabel{a_bcde}
\vspace{.2in}
\end{figure}

%Writing an element means replacing the first element of the future sequence by the new one and moving it to the past. \figref{ax_cde} shows the state of the same stream after having written an \ct{x} using the message \mthind{Stream}{nextPut:} \ct{anElement}.
Запись элемента подразумевает замену первого элемента <<будущей>> последовательности и перемещение его <<в прошлое>>. На \figref{ax_cde} показано состояние потока после записи буквы \ct{x} при помощи сообщения \mthind{Stream}{nextPut:} \ct{anElement}.

\begin{figure}[h!t]
\centerline{\includegraphics[scale=0.5]{ax_cdeStef}}
%\caption{The same stream after having written an \ct{x}.}
\caption{Поток после записи в него буквы \ct{x}.}
\figlabel{ax_cde}
\vspace{.2in}
\end{figure}

%=============================================================
%\section{Streams vs. collections}
\section{Потоки и коллекции}

%The collection protocol supports the storage, removal and enumeration of the elements of a collection, but does not allow these operations to be intermingled. For example, if the elements of an \clsind{OrderedCollection} are processed by a \mthind{OrderedCollection}{do:} method, it is not possible to add or remove elements from inside the \ct{do:} block.
Протокол коллекций поддерживает сохранение, удаление и перечисление элементов коллекции, но не позволяет совмещать эти операции. Например, при обработке элементов \clsind{OrderedCollection} с помощью метода \mthind{OrderedCollection}{do:} внутри блока \ct{do:} невозможно ни добавлять, ни удалять элементы.
%Nor does the collection protocol offer ways to iterate over two collections at the same time, choosing which collection goes forward and which does not. Procedures like these require that a traversal index or position reference is maintained outside of the collection itself: this is exactly the role of \clsind{ReadStream}, \clsind{WriteStream} and \clsind{ReadWriteStream}.
Также не позволяет протокол коллекций производить итерацию сразу по двум коллекциям, выбирая, по какой из них идёт продвижение, а по какой нет. Операции, подобные этой, требуют чтобы индекс обхода или указатель позиции поддерживался вне самой коллекции: это в точности то, что позволяют \clsind{ReadStream}, \clsind{WriteStream} и \clsind{ReadWriteStream}.

%These three classes are defined to \emph{stream over} some collection.
Эти три класса существуют, чтобы \ugh{\emph{stream over}} коллекции.
%For example, the following snippet creates a stream on an interval, then it reads two elements.
Например, следующий фрагмент кода создаёт поток над интервалом, а затем читает из него два элемента.
\needlines{5}
\begin{code}{@TEST |r|}
r := ReadStream on: (1 to: 1000).
r next.   --> 1
r next.   --> 2
r atEnd.--> false
\end{code}

%\ct{WriteStream}s can write data to the collection:
\ct{WriteStream} может записывать данные в коллекцию:
\begin{code}{@TEST |w|}
w := WriteStream on: (String new: 5).
w nextPut: $a.
w nextPut: $b.
w contents. -->  'ab'
\end{code}

%It is also possible to create \ct{ReadWriteStream}s that support both the reading and writing protocols.
Можно также использовать \ct{ReadWriteStream}, который поддерживает оба протокола: и чтение, и запись.

%The main problem with \ct{WriteStream} and \ct{ReadWriteStream} is that they only support arrays and strings in \pharo. This is currently being changed by the development of a new library named Nile, but for now if you try to stream over another kind of collection, you will get an error:
Основная проблема с \ct{WriteStream} и \ct{ReadWriteStream} заключается в том, что они поддерживают только массивы и строки в \pharo. Эта проблема адресована в новой библиотеке Nile, но сейчас, если вы попытаетесь обернуть в поток другой тип коллекции, вы получите ошибку:

\needlines{3}
\begin{code}{}
w := WriteStream on: (OrderedCollection new: 20).
w nextPut: 12. -->  выдаёт ошибку
\end{code}

%Streams are not only meant for collections, they can be used for files or sockets too. The following example creates a file named \ct{test.txt}, writes two strings to it, separated by a carriage return, and closes the file.
Потоки предназначены не только для коллекций. Они также могут быть использованы для файлов и сокетов. В следующем примере создаётся файл \ct{test.txt}, в него записываются две строки, разделённые символом возврата каретки, а затем файл закрывается.

\needlines{3}
\begin{code}{}
StandardFileStream
  fileNamed: 'test.txt'
  do: [:str | str
                nextPutAll: '123';
                cr;
                nextPutAll: 'abcd'].
\end{code}
\cmindex{FileStream class}{fileNamed:do:}

%The following sections present the protocols in more depth.
Последующие главы в деталях рассказывают об этих протоколах.

%=============================================================
%\section{Streaming over collections}
\section{Streaming over collections}

%Streams are really useful when dealing with collections of elements. They can be used for reading and writing elements in collections. We will now explore the stream features for the collections.
Потоки особенно полезны при использовании совместно с коллекциями. Они могут использоваться для чтения и записи элеметов в коллекции, с чем мы сейчас и познакомимся.

%-----------------------------------------------------------------
%\subsection{Reading collections}
\subsection{Чтение из коллекций}

%This section presents features used for reading collections. Using a stream to read a collection essentially provides you a pointer into the collection. That pointer will move forward on reading and you can place it wherever you want. The class \clsindmain{ReadStream} should be used to read elements from collections.
В этом разделе рассказывается о возможностях потоков касающихся чтения из коллекций. \ugh{Использование потока для чтения из коллекции по сути предоставляет вам указатель на эту коллекцию.} Этот указатель будет продвигаться вперёд по мере чтения, но вы можете поместить его в любое место, в какое пожелаете. За это поведение отвечает класс \clsindmain{ReadStream}.

%Methods \mthind{ReadStream}{next} and \mthind{ReadStream}{next:} are used to retrieve one or more elements from the collection.
Методы \mthind{ReadStream}{next} и \mthind{ReadStream}{next:} используются для получения одного или более элементов коллекции.

\begin{code}{@TEST |stream|}
stream := ReadStream on: #(1 (a b c) false).
stream next. -->   1
stream next. -->   #(#a #b #c)
stream next. -->   false
\end{code}
\cmindex{PositionableStream class}{on:}

\begin{code}{@TEST |stream|}
stream := ReadStream on: 'abcdef'.
stream next: 0. -->   ''
stream next: 1. -->   'a'
stream next: 3. -->   'bcd'
stream next: 2. -->   'ef'
\end{code}

%The message \mthind{PositionableStream}{peek} is used when you want to know what is the next element in the stream without going forward.
Используйте сообщение \mthind{PositionableStream}{peek}, когда нужно получить следующий элемент коллекции без продвижения указателя вперёд.

\begin{code}{@TEST |stream negative number|}
stream := ReadStream on: '-143'.
negative := (stream peek = $-).    "получить следующий элемент, не читая его"
negative. --> true
negative ifTrue: [stream next].       "игнорирует знак минус"
number := stream upToEnd.
number. --> '143'
\end{code}
\noindent
%This code sets the boolean variable \ct{negative} according to the sign of the number in the stream and \ct{number} to its absolute value. The method \mthind{ReadStream}{upToEnd} returns everything from the current position to the end of the stream and sets the stream to its end. This code can be simplified using \mthind{PositionableStream}{peekFor:}, which moves forward if the following element equals the parameter and doesn't move otherwise.
Этот код устанавливает логическую переменную \ct{negative} в соответствии со знаком числа в потоке, а в переменную \ct{number} записывает его абсолютное значение. Метод \mthind{ReadStream}{upToEnd} возвращает всё содержимое потока, начиная с текущей позиции и до конца, устанавливая указатель потока в его конец. Этот пример можно записать проще, используя метод \mthind{PositionableStream}{peekFor:}, который продвигает указатель вперёд, если следующий элемент равен переданному параметру, и не продвигает в противном случае.

\begin{code}{@TEST |stream negative number|}
stream := '-143' readStream.
(stream peekFor: $-) --> true
stream upToEnd         --> '143'
\end{code}
\noindent
%\ct{peekFor:} also returns a boolean indicating if the parameter equals the element.
\ct{peekFor:} возвращает логическое значение, показывающее совпадает ли параметр с элементом.

%You might have noticed a new way of constructing a stream in the above example: one can simply send \mthind{SequenceableCollection}{readStream} to a sequenceable collection to get a reading stream on that particular collection.
Вы могли обратить внимание на ещё один способ создания потока в последнем примере: достаточно отправить сообщение \mthind{SequenceableCollection}{readStream} упорядоченной коллекции, чтобы получить поток для её чтения. 

%\paragraph{Positioning.} There are methods to position the stream pointer. If you have the index, you can go directly to it using \mthind{PositionableStream}{position:}. You can request the current position using \mthind{PositionableStream}{position}. Please remember that a stream is not positioned on an element, but between two elements. The index corresponding to the beginning of the stream is 0.
\paragraph{Позиционирование.} Существуют методы для установки указателя потока в определённую позицию. Вы можете, имея некоторый индекс, перейти прямо к нему, используя \mthind{PositionableStream}{position:}. Вы можете получить текущую позицию, используя \mthind{PositionableStream}{position}. Помните, что поток указывает не на элемент, а между двумя элементами. Индекс 0 соответствует началу потока.

%You can obtain the state of the stream depicted in \figref{ab_cde} with the following code:
Вы можете получить состояние потока, изображённое на \figref{ab_cde} с помощью следующего кода:

\begin{code}{@TEST |stream|}
stream := 'abcde' readStream.
stream position: 2.
stream peek --> $c
\end{code}

\begin{figure}[h!t]
\centerline{\includegraphics[scale=0.5]{ab_cdeStef}}
%\caption{A stream at position 2}
\caption{Поток в позиции 2}
\figlabel{ab_cde}
\vspace{.2in}
\end{figure}

%To position the stream at the beginning or the end, you can use \mthind{PositionableStream}{reset} or \mthind{PositionableStream}{setToEnd}. \mthind{PositionableStream}{skip:} and \mthind{PositionableStream}{skipTo:} are used to go forward to a location relative to the current position: \ct{skip:} accepts a number as argument and skips that number of elements whereas \ct{skipTo:} skips all elements in the stream until it finds an element equal to its parameter. Note that it positions the stream after the matched element.
Чтобы установить поток в конечную позицию, используйте \mthind{PositionableStream}{reset} или \mthind{PositionableStream}{setToEnd}. \mthind{PositionableStream}{skip:} и \mthind{PositionableStream}{skipTo:} служат для того, чтобы продвинуть указатель вперёд относительно текущей позиции: \ct{skip:} принимает в качестве аргумента количество элементов, которое нужно пропустить, в то время как \ct{skipTo:} пропускает элементы до тех пор, пока не встретит элемент совпадающий с переданным параметром, и помещает указатель за этот элемент.

\begin{code}{@TEST |stream|}
stream := 'abcdef' readStream.
stream next.        --> $a    "поток установен в позицию сразу после a"
stream skip: 3.                           "теперь после d"
stream position.  -->   4
stream skip: -2.                          "теперь после b"
stream position.  --> 2
stream reset.
stream position.  --> 0
stream skipTo: $e.                      "а теперь сразу за e"
stream next.        --> $f
stream contents. --> 'abcdef'
\end{code}

%As you can see, the letter \ct{e} has been skipped.
Как вы можете видеть, буква \ct{e} была пропущена.

%The method \mthind{PositionableStream}{contents} always returns a copy of the entire stream.
Метод \mthind{PositionableStream}{contents} всегда возвращает полную копию потока.

%\paragraph{Testing.} Some methods allow you to test the state of the current stream: \mthind{PositionableStream}{atEnd} returns true if and only if no more elements can be read whereas \mthind{PositionableStream}{isEmpty} returns true if and only if there is no element at all in the collection.
\paragraph{Тестирование.} Некоторые методы позволяют проверять текущее состояние потока: \mthind{PositionableStream}{atEnd} возвращает истину только если не осталось больше не осталось элементов для чтения, а \mthind{PositionableStream}{isEmpty} возвращает истину только если в коллекции вообще нет элементов.

%Here is a possible implementation of an algorithm using \ct{atEnd} that takes two sorted collections as parameters and merges those collections into another sorted collection:
Ниже представлена возможная реализация алгоритма, использующего \ct{atEnd}, который принимает две отсортированные коллекции в качестве параметров и объединяет их в другую отсортированную коллекцию:

\needlines{4}
\begin{code}{@TEST |stream1 stream2 result|}
stream1 := #(1 4 9 11 12 13) readStream.
stream2 := #(1 2 3 4 5 10 13 14 15) readStream.

"Переменная result будет содержать отсортированную коллекцию"
result := OrderedCollection new.
[stream1 atEnd not & stream2 atEnd not]
  whileTrue: [stream1 peek < stream2 peek
 	"Удалить наименьший элемент из обоих потоков и добавить его к результату"
    ifTrue: [result add: stream1 next]
    ifFalse: [result add: stream2 next]].

"One of the two streams might not be at its end. Copy whatever remains."
"Один из двух потоков может быть не в конечной позиции. Копировать всё, что осталось."
result
  addAll: stream1 upToEnd;
  addAll: stream2 upToEnd.

result. -->   an OrderedCollection(1 1 2 3 4 4 5 9 10 11 12 13 13 14 15)
\end{code}

%-----------------------------------------------------------------
\subsection{Writing to collections}

We have already seen how to read a collection by iterating over its
elements using a \ct{ReadStream}. We'll now learn how to create
collections using \clsindmain{WriteStream}{}s.

\ct{WriteStream}s are useful for appending a lot of data to a collection at various locations. They are often used to construct strings that are based on static and dynamic parts as in this example:

\begin{code}{NB: can't be tested due to the changing number of classes}
stream := String new writeStream.
stream
  nextPutAll: 'This Smalltalk image contains: ';
  print: Smalltalk allClasses size;
  nextPutAll: ' classes.';
  cr;
  nextPutAll: 'This is really a lot.'.

stream contents. --> 'This Smalltalk image contains: 2322 classes.
This is really a lot.'
\end{code}

This technique is used in the different implementations of the method
\ct{printOn:} for example. There is a simpler and more efficient way
of creating streams if you are only interested in the content of the
stream:

\begin{code}{@TEST |string|}
string := String streamContents:
		[:stream |
			stream
                 print: #(1 2 3);
                 space;
                 nextPutAll: 'size';
                 space;
                 nextPut: $=;
                 space;
                 print: 3.	].
string. -->   '#(1 2 3) size = 3'
\end{code}

The method \mthind{SequenceableCollection class}{streamContents:} \seclabel{streamContents} creates a collection and a stream on
that collection for you. It then executes the block you gave passing
the stream as a parameter. When the block ends, \ct{streamContents:}
returns the content of the collection.

The following \ct{WriteStream} methods are especially useful in this context:

\begin{description}
\item[\lct{nextPut:}] adds the parameter to the stream;
\item[\lct{nextPutAll:}] adds each element of the collection, passed as a
  parameter, to the stream;
\item[\lct{print:}] adds the textual representation of the parameter to the
  stream.
	\cmindex{Stream}{print:}
\end{description}

There are also methods useful for printing different kinds of characters to
the stream like \mthind{WriteStream}{space}, \mthind{WriteStream}{tab} and
\mthind{WriteStream}{cr} (carriage return). Another useful
method is \mthind{WriteStream}{ensureASpace} which ensures that the last character
in the stream is a space; if the last character isn't a space it adds one.

\paragraph{About Concatenation.}
Using \mthind{WriteStream}{nextPut:} and \mthind{WriteStream}{nextPutAll:} on a \ct{WriteStream} is often the best way to
concatenate characters. Using the comma concatenation operator (\ct{,}) is far
less efficient:
\index{Collection!comma operator}

\begin{code}{}
[| temp |
  temp := String new.
  (1 to: 100000)
    do: [:i | temp := temp, i asString, ' ']] timeToRun --> 115176 "(milliseconds)"

[| temp |
  temp := WriteStream on: String new.
  (1 to: 100000)
    do: [:i | temp nextPutAll: i asString; space].
  temp contents] timeToRun --> 1262 "(milliseconds)"
\end{code}

The reason that using a stream can be much more efficient is that 
comma creates a new string containing
the concatenation of the receiver and the argument, so it must copy both of them.
When you repeatedly concatenate onto the same receiver, it gets longer and longer each time,
so that the number of characters that must be copied goes up exponentially.
This also creates a lot of garbage, which must be collected. Using
a stream instead of string concatenation is a well-known optimization.
\lr{About Concatenation. This is not true for real world examples (the example benchmark is unrealistic). Most of the time concatenation is simpler, cleaner and much faster, for example when being used on a small number of longer strings. (p. 257)}
In fact, you can use \mthind{SequenceableCollection class}{streamContents:} (mentioned on page \pageref{sec:streamContents}) to help you do this:

\begin{code}{}
String streamContents: [ :tempStream |
  (1 to: 100000)
       do: [:i | tempStream nextPutAll: i asString; space]] 
\end{code}

%-----------------------------------------------------------------
\subsection{Reading and writing at the same time}

It's possible to use a stream to access a collection for reading and
writing at the same time.
Imagine you want to create an \ct{History} class which will manage
backward and forward buttons in a web browser.
A history would react as in figures from \ref{fig:emptyStream} to
\ref{fig:page4}.

\begin{figure}[!ht]
\centerline{\includegraphics[scale=0.5]{emptyStef}}
\caption{A new history is empty. Nothing is displayed in the web browser.}
\figlabel{emptyStream}
\vspace{.2in}
\end{figure}

\begin{figure}[!ht]
\centerline{\includegraphics[scale=0.5]{page1Stef}}
\caption{The user opens to page 1.}
\figlabel{page1}
\vspace{.2in}
\end{figure}

\begin{figure}[!ht]
\centerline{\includegraphics[scale=0.5]{page2Stef}}
\caption{The user clicks on a link to page 2.}
\figlabel{page2}
\vspace{.2in}
\end{figure}

\begin{figure}[!ht]
\centerline{\includegraphics[scale=0.5]{page3Stef}}
\caption{The user clicks on a link to page 3.}
\figlabel{page3}
\vspace{.2in}
\end{figure}

\begin{figure}[!ht]
\centerline{\includegraphics[scale=0.5]{page2_Stef}}
\caption{The user clicks on the back button. He is now viewing page 2 again.}
\figlabel{page2_}
\vspace{.2in}
\end{figure}

\begin{figure}[!ht]
\centerline{\includegraphics[scale=0.5]{page1_Stef}}
\caption{The user clicks again the back button. Page 1 is now displayed.}
\figlabel{page1_}
\vspace{.2in}
\end{figure}

\begin{figure}[!ht]
\centerline{\includegraphics[scale=0.5]{page4Stef}}
\caption{From page 1, the user clicks on a link to page 4. The history forgets pages 2 and 3.}
\figlabel{page4}
\vspace{.2in}
\end{figure}

This behaviour can be implemented using a \clsind{ReadWriteStream}.

\needlines{6}
\begin{code}{}
Object subclass: #History
  instanceVariableNames: 'stream'
  classVariableNames: ''
  poolDictionaries: ''
  category: 'PBE-Streams'

History>>initialize
    super initialize.
    stream := ReadWriteStream on: Array new.
\end{code}

Nothing really difficult here, we define a new class which contains a
stream. The stream is created during the \ct{initialize} method.

We need methods to go backward and forward:

\begin{code}{}
History>>goBackward
  self canGoBackward ifFalse: [self error: 'Already on the first element'].
  stream skip: -2.
  ^ stream next.

History>>goForward
  self canGoForward ifFalse: [self error: 'Already on the last element'].
  ^ stream next
\end{code}

Until then, the code was pretty straightforward. Now, we have to deal
with the \ct{goTo:} method which should be activated when the user
clicks on a link. A possible solution is:

\begin{code}{}
History>>goTo: aPage
    stream nextPut: aPage.
\end{code}

This version is incomplete however. This is because when the user
clicks on the link, there should be no more future pages to go to,
\ie the forward button must be deactivated. To do this, the simplest
solution is to write \ct{nil} just after to indicate the history end:

\begin{code}{}
History>>goTo: anObject
  stream nextPut: anObject.
  stream nextPut: nil.
  stream back.
\end{code}

Now, only methods \ct{canGoBackward} and \ct{canGoForward} have to be
implemented.

A stream is always positioned between two elements. To go backward,
there must be two pages before the current position: one page is the
current page, and the other one is the page we want to go to.

\begin{code}{}
History>>canGoBackward
  ^ stream position > 1

History>>canGoForward
  ^ stream atEnd not and: [stream peek notNil]
\end{code}

Let us add a method to peek at the contents of the stream:
\begin{code}{}
History>>contents
  ^ stream contents
\end{code}

And the history works as advertised:
\begin{code}{}
History new
	goTo: #page1;
	goTo: #page2;
	goTo: #page3;
	goBackward;
	goBackward;
	goTo: #page4;
	contents --> #(#page1 #page4 nil nil)
\end{code}

%=============================================================
\section{Using streams for file access}

You have already seen how to stream over collections of elements. It's
also possible to stream over files on your hard disk.
Once created, a stream on a file is really like a stream on a
collection: you will be able to use the same protocol to read, write
or position the stream.
The main difference appears in the creation of the stream.
There are several different ways to create file streams, as we shall now see.

%-----------------------------------------------------------------
\subsection{Creating file streams}
\seclabel{creat-file-stre}

To create file streams, you will have to use one of the following
instance creation methods offered by the class \clsindmain{FileStream}:

\begin{description}

\item[\lct{fileNamed:}] Open a file with the given name for reading and
  writing. If the file already exists, its prior contents may be
  modified or replaced, but the file will not be truncated on
  close. If the name has no directory part, then the file will be
  created in the default directory.
  \cmindex{FileStream class}{fileNamed:}
  
\item[\lct{newFileNamed:}] Create a new file with the given name,
  and answer a stream opened for writing on that file. If the file
  already exists, ask the user what to do.
  \cmindex{FileStream class}{newFileNamed:}
  
\item[\lct{forceNewFileNamed:}] Create a new file with the given
  name, and answer a stream opened for writing on that file. If the
  file already exists, delete it without asking before creating the
  new file.
  \cmindex{FileStream class}{forceNewFileNamed:}

\item[\lct{oldFileNamed:}] Open an existing file with the given
  name for reading and writing. If the file already exists, its prior
  contents may be modified or replaced, but the file will not be
  truncated on close. If the name has no directory part, then the file
  will be created in the default directory.
  \cmindex{FileStream class}{oldFileNamed:}

\item[\lct{readOnlyFileNamed:}] Open an existing file with the
  given name for reading.
  \cmindex{FileStream class}{readOnlyFileNamed:}

\end{description}

You have to remember that each time you open a stream on a file, you
have to close it too. This is done through the \mthind{FileStream}{close} method.

\begin{code}{@TEST |stream|}
stream := FileStream forceNewFileNamed: 'test.txt'.
stream
    nextPutAll: 'This text is written in a file named ';
    print: stream localName.
stream close.

stream := FileStream readOnlyFileNamed: 'test.txt'.
stream contents. --> 'This text is written in a file named ''test.txt'''
stream close.
\end{code}

% \on{need way to clean up test files afterwards}

%[:fileName | (FileDirectory forFileName: fileName)
%	deleteFileNamed: fileName
%	ifAbsent: [ 'Could not delete ', fileName ] ]
%	value: 'test.txt'

The method \mthind{FileStream}{localName} answers the last component of the name of the file. You can
also access the full path name using the method
\mthind{StandardFileStream}{fullName}.

You will soon notice that manually closing the file stream is
painful and error-prone. That's why \ct{FileStream} offers a message called \mthind{FileStream class}{forceNewFileNamed:do:} to
automatically close a new stream after evaluating a block that
sets its contents.

\begin{code}{@TEST |string|}
FileStream
    forceNewFileNamed: 'test.txt'
    do: [:stream |
        stream
            nextPutAll: 'This text is written in a file named ';
            print: stream localName].
string := FileStream
            readOnlyFileNamed: 'test.txt'
            do: [:stream | stream contents].
string --> 'This text is written in a file named ''test.txt'''
\end{code}

The stream-creation methods that take a block as an argument first
create a stream on a file, then execute the block with the stream
as an argument, and finally close the stream. These methods return what
is returned by the block, which is to say, the value of the last
expression in the block. This is used in the previous example to get
the content of the file and put it in the variable \ct{string}.

%-----------------------------------------------------------------
\subsection{Binary streams}
\seclabel{binary-streams}

By default, created streams are text-based which means you will read
and write characters. If your stream must be binary, you have to send
the message \mthind{FileStream}{binary} to your stream.

When your stream is in binary mode, you can only write numbers from 0
to 255 (1 Byte). If you want to use \ct{nextPutAll:} to write more
than one number at a time, you have to pass a \clsind{ByteArray} as
argument.

\begin{code}{@TEST}
FileStream
  forceNewFileNamed: 'test.bin'
  do: [:stream |
          stream
            binary;
            nextPutAll: #(145 250 139 98) asByteArray].

FileStream
  readOnlyFileNamed: 'test.bin'
  do: [:stream |
          stream binary.
          stream size.         --> 4
          stream next.         --> 145
          stream upToEnd. --> #[250 139 98]
      ].
\end{code}

Here is another example which creates a picture in a file named
``test.pgm'' (portable graymap file format). You can open this file with your favorite drawing program.

% The following does not assert anything, but @TEST is used to ensure
% that no error is thrown.
\begin{code}{@TEST}
FileStream
  forceNewFileNamed: 'test.pgm' 
  do: [:stream |
	stream
		nextPutAll: 'P5'; cr;
		nextPutAll: '4 4'; cr;
		nextPutAll: '255'; cr;
		binary;
		nextPutAll: #(255 0 255 0) asByteArray;
		nextPutAll: #(0 255 0 255) asByteArray;
		nextPutAll: #(255 0 255 0) asByteArray;
		nextPutAll: #(0 255 0 255) asByteArray
	]
\end{code}

This creates a 4x4 checkerboard as shown in \figref{checkerboard4x4}.

\begin{figure}[!ht]
\centerline{\includegraphics[width=0.25\textwidth]{checkerboard4x4}}
\caption{A 4x4 checkerboard you can draw using binary streams.}
\figlabel{checkerboard4x4}
\vspace{.2in}
\end{figure}

%=============================================================
\section{Chapter summary}

Streams offer a better way than collections to incrementally read and write a sequence of elements. There are easy ways to convert back and forth between streams and collections.

\begin{itemize}
  \item Streams may be either readable, writeable or both readable and writeable.
  \item To convert a collection to a stream, define a stream ``on'' a collection, \eg \ct{ReadStream on: (1 to: 1000)}, or send the messages \ct{readStream}, \etc to the collection.
  \item To convert a stream to a collection, send the message \ct{contents}.
  \item To concatenate large collections, instead of using the comma operator, it is more efficient to create a stream, append the collections to the stream with \ct{nextPutAll:}, and extract the result by sending \ct{contents}.
  \item File streams are by default character-based. Send \ct{binary} to explicitly make them binary.
\end{itemize}

%=================================================================
\ifx\wholebook\relax\else\end{document}\fi
%=================================================================

%-----------------------------------------------------------------

%%% Local Variables: 
%%% coding: utf-8
%%% mode: latex
%%% TeX-master: t
%%% TeX-PDF-mode: t
%%% ispell-local-dictionary: "english"
%%% End:
