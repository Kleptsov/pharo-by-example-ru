% $Author: oscar $
% $Date: 2009-08-17 00:11:18 +0600 (пн, 17 авг 2009) $
% $Revision: 28482 $
% $Id: Seaside.tex 28482 2009-08-16 18:11:18Z oscar $

% HISTORY:
% 2007-10-29 - Oscar started chapter
% 2007-11-30 - Oscar first draft
% 2007-12-07 - Orla Greevy reviewed
% 2007-12-09 - Lukas Renggli reviewed
% 2008-01-11 - Andrew revised
% 2009-04-17 - Fabrizio Perin reviewed
% 2009-04-18 - Jorge Ressia reviewed
% 2009-05-06 - Oscar converted to Pharo; fixed review comments

%=================================================================
\ifx\wholebook\relax\else
% --------------------------------------------
% Lulu:
	\documentclass[a4paper,10pt,twoside]{book}
	\usepackage[
		papersize={6.13in,9.21in},
		hmargin={.75in,.75in},
		vmargin={.75in,1in},
		ignoreheadfoot
	]{geometry}
	\input{../common.tex}
	\pagestyle{headings}
	\setboolean{lulu}{true}
% --------------------------------------------
% A4:
%	\documentclass[a4paper,11pt,twoside]{book}
%	\input{../co	mmon.tex}
%	\usepackage{a4wide}
% --------------------------------------------
    \graphicspath{{figures/} {../figures/}}
	\begin{document}
	% \renewcommand{\nnbb}[2]{} % Disable editorial comments
	\sloppy
\fi
%=================================================================

%\chapter{Seaside by Example}
\chapter{Seaside в примерах}

%\chalabel{seaside}
\chalabel{seaside}

%=================================================================

% \ind{Seaside} is a framework for building web applications in Smalltalk. It was
% originally developed by Avi Bryant \index{Bryant, Avi} in 2002; once mastered, Seaside
% makes web applications almost as easy to write as desktop applications.

\ind{Seaside} -- фреймворк для создания веб-приложений в Smalltalk.
Был разработан Avi Bryant \index{Bryant, Avi} в 2002 году.
После освоения, Seaside позволяет писать веб-приложения так же легко как и стандартные
приложения.

% Two of the better known applications built with Seaside are
% \ind{SqueakSource}\footnote{\url{http://SqueakSource.com}} and \ind{Dabble
% DB}\footnote{\url{http://DabbleDB.com}}.
% Seaside is unusual in that it is thoroughly object-oriented: there are no XHTML
% templates, no complicated control flows through web pages, and no encoding of state in
% URLs. Instead, you just send messages to objects. What a nice idea!

Наиболее известные приложения построенные при помощи \hbox{Seaside --}
\ind{SqueakSource}\footnote{\url{http://SqueakSource.com}}
и \ind{Dabble DB}\footnote{\url{http://DabbleDb.com}}.
Необычность Seaside заключается в том, что он является чистым объектно-ориентированным:
отсутствуют XHTML шаблоны, нет сложного управления через веб-страницы, отсутствует
человокочитабельный URL.
Вместо этого можно отправлять сообщения объектам.
Вот в чем главная идея!

% \section{Why do we need Seaside?}
\section{Почему нам нужен Seaside?}

% Modern web applications try to interact with the user in the same way as desktop
% applications: they ask the user questions and the user responds, usually by filling in a
% form or clicking a button.
% But the web works the other way around: the user's browser makes a request of the server,and the server responds with a new web page.
% So \ind{web application development} frameworks have to cope with a host of problems,
% chief among them being the management of this ``inverted'' control flow.
% Because of this, many web applications try to forbid the use of the browser's ``back''
% button due to the difficulty of keeping track of the state of a session.
% Expressing non-trivial control flows across multiple web pages is often cumbersome, and
% multiple control flows can be difficult or impossible to express.

Современные веб-приложения взаимодействуют с пользователем и ведут себя как обычные приложения:
они обращаются к пользователю с вопросами и отвечают на его запросы, используя при этом формы или нажатия на кнопки.
Но работа в сети имеет и другие стороны: браузер пользователя делает запросы к серверу, и получает от него ответ в виде веб-страницы.
Так \ind{среды разработки приложений} должны справлятся со множеством проблем,
где главной проблемой является управление обратными потоками.
Из-за этого многие приложения пытаются запретить использование кнопки <<предыдущая страница>> у браузера,
чтобы не усложнять текущее состояние сессии.
Управление тривиальными потоками через несколько веб-страниц часто бывает громоздко,
а управление множеством потоков становится сложным или невозможным.

% Seaside is a component-based framework that uses ``\ind{continuations}''\footnote{A
% \emph{continuation} represents ''the rest of the computation'' at any point in a
% computation. In Smalltalk, a continuation is just an object that captures the current
% state of the computation, and which can be resumed at any point. to keep track of
% multiple points in the control flow of web applications. Continuations are managed
% automatically by Seaside, so web developers do not even have to be aware of the
% underlying machinery. It just works.

Seaside -- компонентно ориентированный фреймворк который использует
\emph{продолжения} представляющие собой <<остаточные вычисления>> в какой-либо точке.
В Smalltalk, продолжение является объектом который запоминает текущее состояние вычислений,
и который может вернуться в любую точку. Продолжения управляются Seaside автоматически,
что даже веб-разработчику не надо быть в курсе их основных механизмов. И это работает.

\index{Seaside!backtracking state}
\index{Seaside!transactions}
\index{Seaside!components}

% Seaside is a component-based framework that makes web development easier in several
% ways.First, control flow can be expressed naturally using message sends.
% Seaside keeps track of which web page corresponds to which point in the execution of the
% web application.
% This means that the browser's ``back'' button works correctly.

Seaside -- фреймворк, который делает веб разработку проще в нескольких случаях.
Во-первых, управление потоком может быть выраженно посылкой сообщений.
Также Seaside отслеживает соответствие веб-страницы точке исполнения веб-приложения.
Это означает, что кнопка браузера <<назад>> будет работать корректно.


% Second, state is managed for you.
% As the developer, you have the choice of enabling 
% backtracking of state, so that navigation ``back'' in time will undo side-effects.
% Alternatively, you can use the transaction support built into Seaside to prevent users
% from undoing permanent side-effects when they use the back button.
% You do not have to encode state information in the URL\,---\,this too is managed
% automatically for you.

Во-вторых состояние может быть управляемо вами.
Как разработчик, вы имеете выбор разрешения возврата к предыдущему состоянию кнопкой <<назад>>, чтобы отменить ошибочные действия.
В качестве альтернативы, можно использовать механизм транзакций, встроенный в Seaside,
чтобы защитить пользователя от нежелательных побочных эффектов, при нажатии на кнопку <<назад>>.
Также не нужно кодировать информацию в URL адресе, все это уже автоматизированно для вас.


% Third, web pages are built up from nested components, each of which can support its own,
% independent control flow.
% There are no XHTML templates\,---\,instead valid XHTML is generated programmatically
% using a simple Smalltalk protocol.
% Seaside supports Cascading Style Sheets (\ind{CSS}), so content and layout are cleanly
% separated.
% \seeindex{Cascading Style Sheets}{CSS}

В третьих, веб-страницы постоены из вложенных компонентов,
каждый из которых может поддерживать свое собственное,
независимое управление потоком.
Отсутствуют временные шаблоны XHTML,
вместо это XHTML генерируется программно,
используюя простой Smalltalk протокол.
Seaside поддерживает Таблицы каскадных стилей (\ind{CSS}), так как контекст и оформление страницы разделены.
\seeindex{Cascading Style Sheets}{CSS}

% Finally, Seaside provides a convenient web-based development interface, making it easy todevelop applications iteratively, debug applications interactively, and recompile and
% extend applications while the server is running.

И в заключении, Seaside обеспечивает удобный веб интерфейс среды,
позволяющей интерактивно отлаживать приложение, рекомпилировать, и расширять, когда сервер запущен.

%=================================================================
% \section{Getting started}

\section{Начало работы с Seaside}

% The easiest way to get started is to download the ``Seaside \subind{Seaside}{One-Click
% Experience}'' from the Seaside \subind{Seaside}{web
% site}\footnote{\url{http://seaside.st}}.
% This is a prepackaged version of Seaside 2.8 for \ind{Mac OSX}, \ind{Linux} and
% \ind{Windows}.
% The same web site lists many pointers to additional resources, including documentation
% and tutorials.
% Be warned, however, that Seaside has evolved considerably over the years, and not all
% available material refers to the latest version of Seaside.

Самый простой путь, чтобы начать - это загрузить Seaside
\subind{Seaside}{One-Click Experience} из Seaside \subind{Seaside}{веб сайта} \footnote{\url{http://seaside.st}}.
Это версия Seaside 2.8 для \ind{Mac OSX}, \ind{Linux} и \ind{Windows}.
На этом же сайте много ссылок на дополнительные ресурсы, включающие документацию и руководства.
Необходимо иметь в виду, что Seaside претерпела значительные изменения на протяжении ряда лет,
и не все имеющиеся материалы относятся к последней версии Seaside.

% If you are feeling more adventurous, an alternative to the ``one-click'' image is to
% start with the latest \ind{\pharo web
% image}\footnote{\url{http://pharo-project.org/download}}, and install Seaside yourself byfollowing the manual \subind{Seaside}{installation} instructions on the Seaside web site.
% Seaside includes a web server; you can turn the server on, telling it to listen on port
% 8080, by evaluating \clsind{WAKom} \ct{startOn: 8080},
% and you can turn it off again by evaluating \ct{WAKom stop}.
% In the default installation, the default \subind{Seaside}{administrator login} is
% \lct{admin} and the default password is \lct{seaside}.
% To change them, evaluate: \clsind{WADispatcherEditor} \ct{initialize}.
% This will prompt you for a new name and password.

Если вы считаете себя продвинутым пользователем,
то в качестве альтернативы можно использовать <<one-click>> образ
с последнего \ind{\pharo веб образа}\footnote{\url{http://pharo-project.org/download}},
и установить Seaside самому следуя руководству
\subind{Seaside}{installation} на Seaside веб сайте.
Seaside включает в себя веб сервер; его можно запустить на прослушивание порта 8080,
выполнив \clsind{WAKom} \ct{startOn: 8080},
и можно остановить, выполнив \ct{WAKom stop}.
После установки, по умолчанию,
\subind{Seaside}{логин администратора} -- \lct{admin} и пароль -- \lct{seaside}.
Чтобы изменить, нужно выполнить: \clsind{WADispatcherEditor} \ct{initialize}.
И ввести необходимые имя и пароль.


\begin{figure}[tbh]
\begin{center}
\includegraphics[width=\textwidth]{seasideStartup}

% \caption{Starting up Seaside}
\caption{Запуск Seaside}

\figlabel{seasideStartup}
\end{center}
\end{figure}

% \dothis{Start the Seaside server and direct a web browser to
% \url{http://localhost:8080/seaside/}.}
\dothis{Запустите Seaside сервер и откройте через веб-браузер \url{http://localhost:8080/seaside/}.}

\noindent
% You should see a web page that looks like \figref{seasideStartup}.
Открывшаяся веб-страница будет выглядеть как \figref{seasideStartup}.

\noindent
% \dothis{Navigate to the \lct{examples{\go}counter} page. (\figref{counter})}
\dothis{Перейдите по ссылкам \lct{examples}{\go}counter. (\figref{counter})}

\begin{figure}[htb]
\begin{center}
\includegraphics[width=0.8\textwidth]{counter}
\caption{The counter.}
\figlabel{counter}
\end{center}
\end{figure}

\noindent
% This page is a small Seaside application: it displays a \subind{Seaside}{counter} that
% can be incremented or decremented by clicking on the \link{++} and \link{--\,--} links.

Эта страница является небольшим Seaside приложением,
в котором значение счетчика показанного на веб-странице можно изменять,
щелкая по ссылкам \link{++} и \link{--}.

% \noindent
% \dothis{Play with the counter by clicking on these links.
% Use your browser's ``back'' button to go back to a previous state, and then click on
% \link{++} again.
% Notice how the counter is correctly incremented with respect to the currently displayed
% state, rather than the state that the counter was in when you started using the ``back''
% button.}

\noindent
Проверте как работает приложение, щелкая по ссылкам на странице.
Далее нажмите кнопку браузера <<назад>>,
чтобы просмотреть предыдущую страницу,
и щелкните по сслылке \link{++}.
Стоит обратить внимание, что увеличение счетчика происходит правильно по отношению к отображаемому числу, а не к числу до использования кнопки <<назад>>.


% Notice the \subind{Seaside}{toolbar} at the bottom of the web page in
% \figref{seasideStartup}.
% Seaside supports a notion of ``sessions'' to keep track of the state of the application
% for different users.
% \button{New Session} will start a new session on the counter application.
% \button{Configure} allows you to configure the settings of your application through a
% convenient web-interface.
% (To close the \button{Configure} view, click on the \link{x} in the top right corner.)
% \button{Toggle Halos} provides a way to explore the state of the application running on
% the Seaside server.
% \button{Profiler} and \button{Memory} provide detailed information about the run-time
% performance of the application.
% \button{XHTML} can be used to validate the generated web page, but works only when the
% web page is publicly accessible from the Internet, because it uses the W3C validation
% service.
% \index{Seaside!halos}

Внизу окна расположена панель инструментов,
которую видно на \figref{seasideStartup}.
Seaside поддерживает сессии для пользователя,
предназначенные для отслеживания состояния приложения для различных пользователей.
\button{New Session} начинает новую сессию.
\button{Configure} позволяет конфигурировать настройки вашего приложения через удобный веб интерфейс.
(Чтобы закрыть окно \button{configure} щелкните по ссылке \link{x}
которая расположенна в правом верхнем углу.)
\button{Toggle Halos} обеспечивает возможность просмотра состояния приложения
исполняемого на Seaside сервере.
\button{Profiler} и \button{Memory} детализируют информацию о производительности приложения и потребляемой памяти.
\button{XHTML} может быть использовано для веб страниц, которые опубликованы в Интернете, так как используется сервис проверки W3C.
\index{Seaside!halos}


% Seaside applications are built up from pluggable ``components''.
% In fact, components are ordinary Smalltalk objects.
% The only thing that is special about them is that they should be instances of classes
% that inherit from the Seaside framework class \ct{WAComponent}.
% We can explore components and their classes from the \pharo image, or directly from the
% web interface using halos.

Seaside приложение создается из подключаемых <<компонентов>>.
Фактически, компонентами являются Smalltalk объекты.
Только одно важное условие должно соблюдаться для них --
все должны быть экзеплярами класса,
который наследует \ct{WAComponent}.
Есть возможность просматривать экземпляры и других классов,
расположенных в \pharo образе.

\begin{figure}[ht]
\begin{center}
\includegraphics[width=\textwidth]{counterHalos}
\caption{Halos}
\figlabel{counterHalos}
\end{center}
\end{figure}

\dothis{Select \button{Toggle Halos}. You should see a web page like
\figref{counterHalos}.
At the top left the text \ct{WACounter} tells us the class of the Seaside component that
implements the behavior of this web page.
Next to this are three clickable icons. 
The first, with the pencil, activates a Seaside class browser on this class.
The second, with the magnifying glass, opens an object inspector on the currently active
\ct{WACounter} instance.
The third, with the coloured circles, displays the \ind{CSS} style sheet for this
component.
At the top right, the \link{R} and \link{S} let you toggle between the rendered and
source views of the web page.
Experiment with all of these links. 
Note that the \link{++} and \link{--} links are also active in the source view.
Contrast the nicely-formatted source view provided by the Halos with the unformatted
source view offered by your browser.}

The Seaside class browser and object inspector can be very convenient when the server is
running on another computer, especially when the server does not have a display, or if itis in remote place.
However, when you are first developing a Seaside application, the server will be running
locally, and it is easy to use the ordinary \pharo development tools in the server image.
\begin{figure}[ht]
\begin{center}
\includegraphics[width=0.7\textwidth]{haltingCounter}
\caption{Halting the counter}
\figlabel{haltingCounter}
\end{center}
\end{figure}

\dothis{Using the object inspector link in the web browser, open an inspector on the
underlying Smalltalk counter object and evaluate \ct{self halt}.
The web page will stop loading.
Now switch to the Seaside image.
You should see a pre-debugger window (\figref{haltingCounter}) showing a \ct{WACounter}
object executing a \ct{halt}.
Examine this execution in the debugger, and then \button{Proceed}.
Go back to the web browser and notice that the counter application is running again.}

Seaside components can be instantiated multiple times and in different contexts.

\begin{figure}[ht]
\begin{center}
\includegraphics[width=\textwidth]{multiCounterHalos}
\caption{Independent subcomponents}
\figlabel{multiCounterHalos}
\end{center}
\end{figure}

\dothis{Point your web browser to
\url{http://localhost:8080/seaside/examples/multicounter}.
You will see an application built out of a number of independent instances of the countercomponent.
Increment and decrement several of the counters.
Verify that they behave correctly even if you use the ``back'' button.
Toggle the halos to see how the application is built out of nested components.
Use the Seaside class browser to view the implementation of \ct{WAMultiCounter}.
You should see three methods on the class side (\ct{canBeRoot}, \ct{description}, and
\ct{initialize}) and three on the instance side (\ct{children}, \ct{initialize}, and
\ct{renderContentOn:}).
Note that an application is simply a component that is willing to be at the root of the
component containment hierarchy; this willingness is indicated by defining a class-side
method \ct{canBeRoot}
to answer \ct{true}.}
\index{Seaside!multi-counter}

You can use the Seaside web interface to configure, copy or remove individual
applications (\ie root-level components). Try making the following configuration change.

\dothis{Point your web browser to \url{http://localhost:8080/seaside/config}.
Supply the login and password (\ct{admin} and \ct{seaside} by default).
Select \link{Configure} next to ``examples.''
Under the heading ``Add entry point'', enter the new name ``counter2'' for the type
\emph{Application} and click on \button{Add} (see \figref{counter2}).
On the next screen, set the \emph{Root Component} to \clsind{WACounter}, then click
\button{Save} and \button{Close}.
Now we have a new counter installed at
\url{http://localhost:8080/seaside/examples/counter2}.
Use the same configuration interface to remove this entry point.
}
\index{Seaside!configuration}


\begin{figure}[ht]
\begin{center}
\includegraphics[width=\textwidth]{counter2}
\caption{Configuring a new application}
\figlabel{counter2}
\end{center}
\end{figure}

Seaside operates in two modes: \emph{development} mode, which is what we have seen so
far, and \emph{deployment} mode, in which the toolbar is not available.
\index{Seaside!deployment mode}
\index{Seaside!development mode}
You can put Seaside into deployment mode using either the configuration page (navigate tothe entry for the application and click on the \link{Configure} link)
% \ab{How?  I couldn't find this}
or click the \button{Configure} button in the toolbar.
In either case, set the deployment mode to \emph{true}.
Note that this affects new sessions only.
You can also set the mode globally by evaluating
\clsind{WAGlobalConfiguration} \lct{setDeploymentMode}
or
\ct{WAGlobalConfiguration setDevelopmentMode}.
\index{Seaside!deployment mode}
\index{Seaside!development mode}

The configuration web page is just another Seaside application, so it too can be
controlled from the configuration page.
If you remove the ``config'' application, you can get it back by evaluating
\clsind{WADispatcherEditor} \ct{initialize}.

%=================================================================
\section{Seaside components}
\seclabel{components}

%\ab{This section was too long\,---\,18 pages. It also contained several self-references
(``see section 1.3''). So I broke into smaller sections, by promoting some of the
subsections and subsubsections.

As we mentioned in the previous section, Seaside applications are built out of
\emph{\subind{Seaside}{components}.}
Let's take a closer look at how Seaside works by implementing the \emph{Hello World}
component.

Every Seaside component should inherit directly or indirectly from \clsind{WAComponent},
as shown in \figref{WACounter}.

\dothis{Define a subclass of \ct{WAComponent} called \ct{WAHelloWorld}.}

Components must know how to render themselves.
Usually this is done by implementing the method \mthind{WAPresenter}{renderContentOn:},
which gets as its argument an instance of \clsind{WAHtmlCanvas}, which knows how to
render XHTML.
\index{Seaside!rendering}

\dothis{Implement the following method, and put it in a protocol called
\prot{rendering}:}
\needlines{2}
\begin{code}{}
WAHelloWorld>>>renderContentOn: html
	html text: 'hello world'
\end{code}

\noindent
Now we must inform Seaside that this component is willing to be a standalone application.\dothis{Implement the following method on the class side of \ct{WAHelloWorld}.}

\begin{code}{}
WAHelloWorld class>>>canBeRoot
	^ true
\end{code}

\noindent
We are almost done!

\dothis{Point your web browser at \url{http://localhost:8080/seaside/config}, add a new
entry point called ``hello'', and set its root component to be \ct{WAHelloWorld}.
Now point your browser to \url{http://localhost:8080/seaside/hello}.
That's it!  You should see a web page like \figref{WAHelloWorld}.}

\begin{figure}[htb]
\begin{center}
\includegraphics[width=\textwidth]{WAHelloWorld}
\caption{``Hello World'' in Seaside}
\figlabel{WAHelloWorld}
\end{center}
\end{figure}

%-----------------------------------------------------------------
\subsection{State backtracking and the ``Counter'' Application}
%{Simple and nested components}

The ``counter'' application is only slightly more complex than the ``hello world''
application.
\seclabel{backtracking}

\begin{figure}[ht]
\begin{center}
\includegraphics[width=\textwidth]{WACounter}
\caption{The \ct{WACounter} class, which implements the \emph{counter} application.
Methods with underlined names are on the class-side; those with plain-text names are on
the instance side.}
\figlabel{WACounter}
\end{center}
\end{figure}

The class \clsind{WACounter} is a standalone application, so \ct{WACounter class} must
answer \ct{true} to the \mthind{WAComponent class}{canBeRoot} message.
It must also register itself as an application; this is done in its class-side
\ct{initialize} method, as shown in \figref{WACounter}.

\ct{WACounter} defines two methods, \ct{increase} and \ct{decrease}, which will be
triggered from the \link{++} and \link{--\,--} links on the web page.
It also defines an instance variable \ct{count} to record the state of the counter.
However, we also want Seaside to synchronize the counter with the browser page:
when the user clicks on the browser's ``back'' button, we want seaside to ``backtrack''
the state of the \ct{WACounter} object.
Seaside includes a general mechanism for backtracking, but each application has to tell
Seaside which parts of its state to track.

A component enables backtracking by implementing the \ct{states} method on the instance
side:
% \ab{note that xspace messes up again, by inserting a space at the start of this line}
\ct{states} should answer an array containing all the objects to be tracked.
In this case, the \ct{WACounter} object adds itself to Seaside's table of backtrackable
objects by returning \ct{Array with: self}.

\paragraph{\emph{Caveat.}}
There is a subtle but important point to watch for when declaring objects for
backtracking.
Seaside tracks state by making a \emph{copy} of all the objects declared in the
\ct{states} array.
It does this using a \clsind{WASnapshot} object; \ct{WASnapshot} is a subclass of
\clsind{IdentityDictionary} that records the objects to be tracked as keys and shallow
copies of their state as values.
If the state of an application is backtracked to a particular snapshot, the state of eachobject entered into the snapshot dictionary is overwritten by the copy saved in the
snapshot.

Here is the point to watch out for:
In the case of \ct{WACounter}, you might think that the state to be tracked is a
number\,---\,the value of the \ct{count} instance variable.
However, having the \ct{states} method answer \ct{Array with: count} won't work.  
This is because the object named by \ct{count} is an integer, and integers are
immutable.The \ct{increase} and \ct{decrease} methods don't change the state of the
object \ct{0}
into \ct{1} or the object \ct{3} into \ct{2}.
Instead, they make \ct{count} name a different integer: 
every time the count is incremented or decremented, the object named by \ct{count} is
\emph{replaced} by another.
This is why \ct{WACounter>>>states} must return \ct{Array with: self}.
When the state of a \mbox{\ct{WACounter}} object is replaced by a previous state, the
\emph{value} of each of the instance variable in the object is replaced by a previous
value; this correctly replaces the current value of \ct{count} by a prior value.
\index{Seaside!backtracking state}
\index{WAPresenter!states@\ct{states}}

\section{Rendering XHTML}

The purpose of a web application is to create, or ``render'', web pages. As we mentioned
in \secref{components}, each Seaside component is responsible for rendering itself.
So, lets start our exploration of rendering by seeing how the counter component renders
itself.

\subsection{Rendering the Counter}

The rendering of the counter is relatively straightforward; the code is shown in
\figref{WACounter}.
The current value of the counter is displayed as an XHTML heading, and the increment and
decrement operations are implemented as html anchors (that is, links) with callbacks to
blocks that will send \ct{increase} and \ct{decrease} to the counter object.

We will have a closer look at the rendering protocol in a moment.
But before we do, let's have a quick look at the \subind{Seaside}{multi-counter}.

\subsection{From Counter to MultiCounter}

\ct{WAMultiCounter}, shown in \figref{WAMultiCounter} is also a standalone application,
so it overrides \mthind{WAComponent class}{canBeRoot} to answer \ct{true}.
In addition, it is a \emph{composite} component, so Seaside requires it to declare its
children by implementing a method \ct{children} that answers an array of all the
components it contains.
It renders itself by rendering each of its subcomponents, separated by a horizontal
rule.Aside from instance and class-side initialization methods, there is nothing else to
the
multi-counter!

\begin{figure}[bht]
\begin{center}
\includegraphics[width=\textwidth]{WAMultiCounter}
\caption{WAMultiCounter}
\figlabel{WAMultiCounter}
\end{center}
\end{figure}

%-----------------------------------------------------------------
\subsection{More about Rendering XHTML}

As you can see from these examples, Seaside does not use templates to generate web
pages.Instead it generates XTHML programmatically.
The basic idea is that every Seaside component should override the method
\mthind{WAPresenter}{renderContentOn:}; this message will be sent by the framework to
each component that needs to be rendered.
This \ct{renderContentOn:} message will have argument that is an \seeindex{canvas}{html
canvas} \emphind{html canvas} onto which the component should render itself. By
convention, the html canvas parameter is called \ct{html}.
An html canvas is analogous to the graphics canvas used by Morphic (and most other
drawing frameworks) to abstract away from the device-dependent details of drawing.
 

Here are some of the most basic rendering methods:
\begin{code}{}
html text: 'hello world'.  "render a plain text string"
html html: '&ndash;'.     "render an XHTML incantation"
html render: 1.              "render any object"
\end{code}

The message \ct{render: anyObject} can be sent to an html canvas to render
\ct{anyObject}; it is normally used to render subcomponents. \lct{anyObject} will itself
be sent the message \ct{renderContentOn:}
this is what happens in the multi-counter (see \figref{WAMultiCounter}).

\subsection{Using Brushes}
\seclabel{brushes}

A canvas provides a number of \emphind{brushes} that can be used to render (\ie
``paint'') content on the canvas.
There are brushes for every kind of XHTML element\,---\,paragraphs, tables, lists, and soon.
To see the full protocol of brushes and convenience methods, you should browse the class
\clsind{WACanvas} and its subclasses.
The argument to \ct{renderContentOn:} is actually an instance of the subclass
\clsind{WARenderCanvas}.

We have already seen the following brush used in the counter and multi-counter examples:
\needlines{2}
\begin{code}{}
html horizontalRule.
\end{code}

\begin{figure}[ht]
\begin{center}
\includegraphics[width=\textwidth]{RenderingDemo}
\caption{RenderingDemo}
\figlabel{RenderingDemo}
\end{center}
\end{figure}

In \figref{RenderingDemo} we can see the output of many of the basic brushes offered by
Seaside.\footnote{The source code for \mthref{renderdemo} is in the package
\ct{PBE-SeasideDemo} in the project \url{http://www.squeaksource.com/PharoByExample}.}
The root component \ct{SeasideDemo} simply renders its subcomponents, which are instancesof \ct{SeasideHtmlDemo}, \ct{SeasideFormDemo}, \ct{SeasideEditCallDemo} and
\ct{SeasideDialogDemo}, as shown in \mthref{renderdemo}.

\needspace{7ex}
\begin{method}[renderdemo]{\lct{SeasideDemo>>renderContentOn:}}
SeasideDemo>>>renderContentOn: html
	html heading: 'Rendering Demo'.
	html heading
		level: 2;
		with: 'Rendering basic HTML: '.
	html div
		class: 'subcomponent';
		with: htmlDemo.
	"render the remaining components ..."
\end{method}

\noindent
Recall that a root component must always declare its children, or Seaside will refuse to
render them.
\begin{code}{}
SeasideDemo>>>children
	^ { htmlDemo . formDemo . editDemo . dialogDemo }
\end{code}

Notice that there are two different ways of instantiating the \ct{heading} brush.
The first way is to set the text directly by sending the message \ct{heading:}.
The second way is instantiate the brush by sending \ct{heading}, and then to send a
cascade of messages to the brush to set its properties and render it.
Many of the available brushes can be used in these two ways.

\important{If you send a \ind{cascade} of messages to a brush including the message
\mthind{WABrush}{with:}, then \ct{with:} should be the \emph{final} message.
\ct{with:}  both sets the content and renders the result.}


In \mthref{renderdemo}, the first heading is at level 1, since this is the default.
We explicitly set the level of the second heading to 2.
The subcomponent is rendered as an XHTML \emph{div} with the \ind{CSS} class
``subcomponent''.
(More on CSS in \secref{css}.)
Also note that the argument to the \ct{with:} keyword message need not be a literal
string: it can be another component, or even\,---\,as in the next example\,---\,a block
containing further rendering actions.

The \ct{SeasideHtmlDemo} component demonstrates many of the most basic brushes.
Most of the code should be self-explanatory.

\begin{code}{}
SeasideHtmlDemo>>>renderContentOn: html 
	self renderParagraphsOn: html.
	self renderListsAndTablesOn: html.
	self renderDivsAndSpansOn: html.
	self renderLinkWithCallbackOn: html
\end{code}

It is common practice to break up long rendering methods into many helper methods, as we
have done here.

\important{Don't put all your rendering code into a single method. 
Split it into helper methods named using the pattern \ct{render*On:}.
All rendering methods go in the \prot{rendering} protocol.
Don't send \ct{renderContentOn:} from your own code, use \ct{render:} instead.}

Look at the following code.  
The first helper method, \ct{SeasideHtmlDemo>>>renderParagraphsOn:}, shows you how to
generate XHTML paragraphs, plain and emphasized text, and images.
Note that in Seaside simple elements are rendered by specifying the text they contain
directly, whereas complex elements are specified using blocks.
This is a simple convention to help you structure your rendering code.

\begin{code}{}
SeasideHtmlDemo>>>renderParagraphsOn: html 
	html paragraph: 'A plain text paragraph.'.
	html paragraph: [
		html
			text: 'A paragraph with plain text followed by a line break. ';
			break;
			emphasis: 'Emphasized text ';
			text: 'followed by a horizontal rule.';
			horizontalRule;
			text: 'An image URI: '.
		html image
			url: self squeakImageUrl;
			width: '50']
\end{code}

The next helper method, \ct{SeasideHtmlDemo>>>renderListsAndTablesOn:}, shows you how to
generate lists and tables.
A table uses two levels of blocks to display each of its rows and the cells within the
rows.

\begin{code}{}
SeasideHtmlDemo>>>renderListsAndTablesOn: html 
	html orderedList: [
		html listItem: 'An ordered list item'].
	html unorderedList: [
		html listItem: 'An unordered list item'].
	html table: [
		html tableRow: [
			html tableData: 'A table with one data cell.']]
\end{code}

The next example shows how we can specify CSS \emph{div}s and \emph{span}s with
\emph{class} or \emph{id} attributes.
Of course, the messages \ct{class:} and \ct{id:} can also be sent to the other brushes,
not just to \emph{div}s and \emph{span}s.
The method \ct{SeasideDemoWidget>>>style} defines how these XHTML elements should be
displayed (see \secref{css}).

\begin{code}{}
SeasideHtmlDemo>>>renderDivsAndSpansOn: html 
	html div
		id: 'author';
		with: [
			html text: 'Raw text within a div with id ''author''. '.
			html span
				class: 'highlight';
				with: 'A span with class ''highlight''.']
\end{code}

Finally we see a simple example of a link, created by binding a simple
\subind{Seaside}{callback} to an ``anchor'' (\ie a link).
Clicking on the link will cause the subsequent text to toggle between ``true'' and
``false'' by toggling the instance variable \ct{toggleValue}.

\needlines{3}
\begin{code}{}
SeasideHtmlDemo>>>renderLinkWithCallbackOn: html 
	html paragraph: [
		html text: 'An anchor with a local action: '.
		html span with: [
			html anchor
				callback: [toggleValue := toggleValue not];
				with: 'toggle boolean:'].
		html space.
		html span
			class: 'boolean';
			with: toggleValue ]
\end{code}

\important{Note that actions should appear only in callbacks.
The code executed while rendering should not change the state of the application!}

%-----------------------------------------------------------------
\subsection{Forms}

Forms are rendered just like the other examples that we have already seen.
Here is the code for the \ct{SeasideFormDemo} component in \figref{RenderingDemo}.
\index{Seaside!XHTML forms}

\begin{code}{}
SeasideFormDemo>>>renderContentOn: html
	| radioGroup |
	html heading: heading.
	html form: [
		html span: 'Heading: '.
		html textInput on: #heading of: self.
		html select
			list: self colors;
			on: #color of: self.
		radioGroup := html radioGroup.
		html text: 'Radio on:'.
		radioGroup radioButton
			selected: radioOn;
			callback: [radioOn := true].
		html text: 'off:'.
		radioGroup radioButton
			selected: radioOn not;
			callback: [radioOn := false].
		html checkbox on: #checked of: self.
		html submitButton
			text: 'done' ]
\end{code}{}

Since a form is a complex entity, it is rendered using a block.
Note that all the state changes happen in the callbacks, not as part of the rendering.

There is one Seaside feature used here that is worth special mention, namely the message
\mthind{WAAnchorTag}{on:of:}.
In the example, this message is used to bind a text input field to the variable
\ct{heading}.
Anchors and buttons also support this message.
The first argument is the name of an instance variable for which accessors have been
defined; the second argument is the object to which this instance variable belongs.
Both observer (\ct{heading}) and mutator (\ct{heading:}) accessor messages must be
understood by the object, with the usual naming convention.
In the case here of a text input field, this saves us the trouble of having to define a
callback that updates the field as well as having to bind the default contents of the
html input field to the current value of the instance variable.
Using \ct{on: #heading of: self}, the \ct{heading} variable is updated automatically
whenever the user updates the text input field.

The same message is used twice more in this example, to cause the selection of a colour
on the html form to update the \ct{color} variable, and to bind the result of the
checkbox to the \ct{checked} variable.
Many other examples can be found in the functional tests for Seaside.
Have a look at the category \scat{Seaside-Tests-Functional}, or just point your browser
to \url{http://localhost:8080/seaside/tests/alltests}.
Select \menu{WAInputTest} and click on the \button{Restart} button to see most of the
features of forms.

Don't forget, if you \button{Toggle Halos}, you can browse the source code of the
examples directly using the Seaside class browser.

%-----------------------------------------------------------------
\section{CSS: Cascading style sheets}
\seclabel{css}

%\ab{I think that it just needs a few paragraphs telling the reader the key ideas behind
CSS, and the new terminology that the CSS folks introduce, before going in to the detailsof how you define their "thingies". Now I have forgotten what they call their "thingies"
--- I know that there are effectively paragraph styles (divs) and character styles
(spans), but I've forgotten what they call them. So, I think that the text needs to tell
the reader, for each thingie, (1) the CSS concept behind the thingie, (2) what it looks
like in a CSS style sheet , (3) what it looks like in html, and (4) how to do it in
Seaside. Maybe (3) can be omitted, because it's not needed to use Seaside.
% \on{I think we do most of that already.}

Cascading Style Sheets\footnote{\url{http://www.w3.org/Style/CSS/}}, or \ind{CSS} for
short, have emerged as a standard way for web applications to separate style from
content.
Seaside relies on CSS to avoid cluttering your rendering code with layout considerations.You can set the CSS style sheet for your web components by defining the method
\ct{style}, which should return a string containing the CSS rules for that component.
The styles of all the components displayed on a web page are joined together, so each
component can have its own style.
A better approach can be to define an abstract class for your web application that
defines a common style for all its subclasses.

Actually, for deployed applications, it is more common to define style sheets as externalfiles.
This way the look and feel of the component is completely separate from its
functionality.
(Have a look at \clsind{WAFileLibrary}, which provides a way to serve static files
without the need for a standalone server.)

If you already are familiar with CSS, then that's all you need to know.
Otherwise, read on for a very brief introduction to CSS.

Instead of directly encoding display attributes in the paragraph and text elements of
your web pages, with CSS you will define different classes of elements and place all
display considerations in a separate style sheet.
Paragraph-like entities are called \emph{div}s and text-like entities are \emph{span}s.
You would then define symbolic names, like ``highlight'' (see example below) for text to
be highlighted, and specify how highlighted text is to be displayed in your style sheet.

Basically a CSS style sheet consists of a set of rules that specify how to format given
XHTML elements.
Each rule consists of two parts.
There is a \emph{selector} that specifies which XHTML elements the rule applies to, and
there is a \emph{declaration} which sets a number of attributes for that element.

\begin{figure}[tb]
\begin{code}{}
SeasideDemoWidget>>>style
	^ '
body {
	font: 10pt Arial, Helvetica, sans-serif, Times New Roman;
}
h2 {
	font-size: 12pt;
	font-weight: normal;
	font-style: italic;
}
table { border-collapse: collapse; }
td {
	border: 2px solid #CCCCCC;
	padding: 4px;
}
#author {
	border: 1px solid black;
	padding: 2px;
	margin: 2px;
}
.subcomponent {
	border: 2px solid lightblue;
	padding: 2px;
	margin: 2px;
}
.highlight { background-color: yellow; }
.boolean { background-color: lightgrey; }
.field { background-color: lightgrey; }
'
\end{code}
\caption{\lct{SeasideDemoWidget} common style sheet.
\figlabel{democss}}
\end{figure}
\figref{democss} illustrates a simple style sheet for the rendering demo shown earlier in\figref{RenderingDemo}.
The first rule specifies a preference for the fonts to use for the \ct{body} of the web
page.
The next few rules specify properties of second-level headings (\ct{h2}), tables
(\ct{table}), and table data (\ct{td}).

The remaining rules have selectors that will match XHTML elements that have the given
``class'' or ``id'' attributes.
CSS selectors for class attributes start with a ``\ct{.}'' and those for id attributes
with ``\ct{#}''.
The main difference between class and id attributes is that many elements may have the
same class, but only one element may have a given id (\ie an \emph{identifier}).
So, whereas a class attribute, such as \ct{highlight}, may occur multiple times on any
page, an id must identify a \emph{unique} element on the page, such as a particular menu,the modified date, or author.
Note that a particular XHTML element may have multiple classes, in which case all the
applicable display attributes will be applied in sequence.

% This style sheet expects at most one element to specify the \emph{author} of the web
page.

Selector conditions may be combined, so the selector \ct{div.subcomponent} will only
match an XHTML element if it is both a div \emph{and} it has a class attribute equal to
``subcomponent''.

It is also possible to specify nested elements, though this is seldom necessary.
For example, the selector ``\ct{p span}'' will match a span within a paragraph but not
within a div.

There are numerous books and web sites to help you learn CSS.
For a dramatic demonstration of the power of CSS, we recommend you to have a look at the
CSS Zen Garden\footnote{\url{http://www.csszengarden.com/}}, which shows how the same
content can be rendered in radically different ways simply by changing the CSS style
sheet.

%-----------------------------------------------------------------
\section{Managing control flow}

Seaside makes it particularly easy to design web applications with non-trivial control
flow.
There are basically two mechanisms that you can use:

\begin{enumerate}
\item A component can \emph{call} another component by sending \ct{caller call: callee}.
The caller is temporarily replaced by the callee, until the callee returns
control by
sending \ct{answer:}.
The caller is usually \ct{self}, but could also be any other currently visible component.  \item A workflow can be be defined as a \emphsubind{Seaside}{task}.
This is a special kind of component that subclasses \clsind{WATask} (instead of
\clsind{WAComponent}). \seclabel{task}
Instead of defining \ct{renderContentOn:}, it defines no content of its own, but rather
defines a \ct{go} method that sends a series of \ct{call:} messages to activate various
subcomponents in turn.
\end{enumerate}
\index{Seaside!control flow}

%-----------------------------------------------------------------
\subsection{Call and answer}

Call and answer are used to realize simple dialogues.

There is a trivial example of \ct{call:} and \ct{answer:} in the rendering demo of
\figref{RenderingDemo}.
The component \ct{SeasideEditCallDemo} displays a text field and an \emph{edit} link.
The callback for the edit link calls a new instance of \ct{SeasideEditAnswerDemo}
initialized to the value of the text field.
The callback also updates this text field to the result which is sent as an answer.

(We underline the \ct{call:} and \ct{answer:} sends to draw attention to them.)

\begin{code}{}
SeasideEditCallDemo>>>renderContentOn: html 
	html span
		class: 'field';
		with: self text.
	html space.
	html anchor
callback: [self text: (self !\underline{call:}! (SeasideEditAnswerDemo new text: self
text))];
		with: 'edit'
\end{code}{}

What is particularly elegant is that the code makes absolutely no reference to the new
web page that must be created.
At run-time, a new page is created in which the \ct{SeasideEditCallDemo} component is
replaced by a \ct{SeasideEditAnswerDemo} component; the parent component and the other
peer components are untouched.

\important{\mthind{WAComponent}{call:} and \mthind{WAComponent}{answer:} should never be
used while rendering.
They may safely be sent from within a \subind{Seaside}{callback}, or from within the
\mthind{WATask}{go} method of a task.}

The \ct{SeasideEditAnswerDemo} component is also remarkably simple.
It just renders a form with a text field.
The submit button is bound to a callback that will answer the final value of the text
field.

\begin{code}{}
SeasideEditAnswerDemo>>>renderContentOn: html
	html form: [
		html textInput
			on: #text of: self.
		html submitButton
			callback: [ self !\underline{answer:}! self text ];
			text: 'ok'.
		]
\end{code}{}

That's it.

Seaside takes care of the control flow and the correct rendering of all the components.
Interestingly, the ``back'' button of the browser will also work just fine (though side
effects are not rolled back unless we take additional steps).

%-----------------------------------------------------------------
\subsection{Convenience methods}

Since certain call--answer dialogues are very common, Seaside provides some convenience
methods to save you the trouble of writing components like \ct{SeasideEditAnswerDemo}.
The generated dialogues are shown in \figref{dialogs}.
We can see these convenience methods being used within
\ct{SeasideDialogDemo>>>renderContentOn:}
\index{Seaside!convenience methods}

\begin{figure}[b]
\begin{center}
\includegraphics[width=\textwidth]{dialogs}
\caption{Some standard dialogs}
\figlabel{dialogs}
\end{center}
\end{figure}

The message \mthind{WAComponent}{request:} performs a call to a component that will let
you edit a text field.
The component answers the edited string.
An optional label and default value may also be specified.

\needlines{3}
\begin{code}{}
SeasideDialogDemo>>>renderContentOn: html
	html anchor
		callback: [ self request: 'edit this' label: 'done' default: 'some text' ];
		with: 'self request:'.
...
\end{code}

The message \mthind{WAComponent}{inform:} calls a component that simply displays the
argument message and waits for the user to click ``ok''.
The called component just returns \ct{self}.

\begin{code}{}
...
	html space.
	html anchor
		callback: [ self inform: 'yesBANG' ];
		with: 'self inform:'.
...
\end{code}

The message \mthind{WAComponent}{confirm:} asks a questions and waits for the user to
select either ``Yes'' or ``No''.
The component answers a boolean, which can be used to perform further actions.

\begin{code}{}
...
	html space.
	html anchor
		callback: [
			(self confirm: 'Are you happy?')
				ifTrue: [ self inform: ':-)' ]
				ifFalse: [ self inform: ':-(' ]
			];
		with: 'self confirm:'.
\end{code}

A few further convenience methods, such as \mthind{WAComponent}{chooseFrom:caption:}, aredefined in the \prot{convenience} protocol of \clsind{WAComponent}.

%-----------------------------------------------------------------
\subsection{Tasks}

A \subind{Seaside}{task} is a component that subclasses \clsind{WATask}.
It does not render anything itself, but simply calls other components in a control flow
defined by implementing the method \mthind{WATask}{go}.

\clsind{WAConvenienceTest} is a simple example of a task defined in the category
\scat{Seaside-Tests-Functional}.
To see its effect, just point your browser to
\url{http://localhost:8080/seaside/tests/alltests}, select \menu{WAConvenienceTest} and
click \button{Restart}.

\begin{code}{}
WAConvenienceTest>>>go
	[ self chooseCheese.
	  self confirmCheese ] whileFalse.
	self informCheese
\end{code}

This task calls in turn three components.
The first, generated by the convenience method \mthind{WAComponent}{chooseFrom:
caption:}, is a \clsind{WAChoiceDialog} that asks the user to choose a cheese.

\begin{code}{}
WAConvenienceTest>>>chooseCheese
	cheese := self
		chooseFrom: #('Greyerzer' 'Tilsiter' 'Sbrinz')
		caption: 'What''s your favorite Cheese?'.
	cheese isNil ifTrue: [ self chooseCheese ]
\end{code}

% \alex{Is there a situation where cheese may be nil? Maybe if a browser authorizes an
empty selection...

The second is a \clsind{WAYesOrNoDialog} to confirm the choice (generated by the
convenience method \mthind{WAComponent}{confirm:}).

\begin{code}{}
WAConvenienceTest>>>confirmCheese
	^self confirm: 'Is ', cheese,  ' your favorite cheese?'
\end{code}

Finally a \clsind{WAFormDialog} is called (via the convenience method
\mthind{WAComponent}{inform:}).

\begin{code}{}
WAConvenienceTest>>>informCheese
	self inform: 'Your favorite cheese is ', cheese, '.'
\end{code}

The generated dialogues are shown in \figref{chooseCheese}.

\begin{figure}[ht]
\begin{center}
\includegraphics[width=\textwidth]{chooseCheese}
\caption{A simple task}
\figlabel{chooseCheese}
\end{center}
\end{figure}

%-----------------------------------------------------------------
\subsection{Transactions}

We saw in \secref{backtracking} that Seaside can keep track of the correspondence betweenthe state of components and individual web pages by having components register their
state for backtracking:
all that a component need do is implement the method \ct{states} to answer an array of
all the objects whose state must be tracked.

Sometimes, however, we do not want to backtrack state: instead we want to \emph{prevent}
the user from accidentally undoing effects that should be permanent.
This is often referred to as ``the shopping cart problem''.
Once you have checked-out your shopping cart and paid for the items you have purchased,
it should not be possible to go ``back'' with the browser and add more items to the
shopping cart!

Seaside allows you to prevent this by defining a task within which certain actions are
grouped together as \emph{transactions}.
You can backtrack within a transaction, but once a transaction is complete, you can no
longer go back to it.
The corresponding pages are \emph{invalidated}, and any attempt to go back to them will
cause Seaside to generate a warning and redirect the user to the most recent valid page.

\begin{figure}[ht]
\begin{center}
\includegraphics[width=\textwidth]{sushiStore}
\caption{The Sushi Store}
\figlabel{sushiStore}
\end{center}
\end{figure}

The Seaside \emphsubind{Seaside}{Sushi Store} is sample application that illustrates manyof the features of Seaside, including transactions.
This application is bundled with your installation of Seaside, so you can try it out by
pointing your browser at
\url{http://localhost:8080/seaside/examples/store}.\footnote{If you cannot find it in
your image, there is a version of the sushi store available on SqueakSource from
\url{http://www.squeaksource.com/SeasideExamples/}.}

The sushi store supports the following workflow:
\begin{enumerate}[itemsep=0pt]
  \item Visit the store.
  \item Browse or search for sushi.
  \item Add sushi to your shopping cart.
  \item Checkout.
  \item Verify your order.
  \item Enter shipping address.
  \item Verify shipping address.
  \item Enter payment information.
  \item Your fish is on its way!
\end{enumerate}

If you toggle the \subind{Seaside}{halos}, you will see that the top-level component of
the sushi store is an instance of \clsind{WAStore}.
It does nothing but render the title bar, and then it renders \ct{task}, an instance of
\clsind{WAStoreTask}.

\begin{code}{}
WAStore>>>renderContentOn: html
	"... render the title bar ..."
	html div id: 'body'; with: task
\end{code}

\clsind{WAStoreTask} captures this workflow sequence. At a couple of points it is
critical that the user not be able to go back and change the submitted information.

\dothis{\,``Purchase'' some sushi and then use the ``back'' button to try to put more
sushi into your cart.
You will get the message ``That page has expired.''}

Seaside lets the programmer say that a certain part of a workflow act like a transaction:once the transaction is complete, the user cannot go back and undo it.
You say this by sending \mthind{WAComponent}{isolate:} to a task with the transactional
block as its argument.
We can see this in the sushi store workflow as follows:

\begin{code}{}
WAStoreTask>>>go
	| shipping billing creditCard |
	cart := WAStoreCart new.
	self isolate:
		[[self fillCart.
		self confirmContentsOfCart]
			whileFalse].

	self isolate:
		[shipping := self getShippingAddress.
		billing := (self useAsBillingAddress: shipping)
					ifFalse: [self getBillingAddress]
					ifTrue: [shipping].
		creditCard := self getPaymentInfo.
		self shipTo: shipping billTo: billing payWith: creditCard].

	self displayConfirmation.
\end{code}

Here we see quite clearly that there are two transactions.
The first fills the cart and closes the shopping phase.
(The helper methods \ct{fillCart} \etc take care of instantiating and calling the right
subcomponents.)
Once you have confirmed the contents of the cart you cannot go back without starting a
new session.
The second transaction completes the shipping and payment data.
You can navigate back and forth within the second transaction until you confirm payment.
However, once both transactions are complete, any attempt to navigate back will fail.

Transactions may also be nested.
A simple demonstration of this is found in the class \clsind{WANestedTransaction}.
The first \ct{isolate:} takes as argument a block that contains another, nested
\ct{isolate:}

\begin{code}{}
WANestedTransaction>>>go
	self inform: 'Before parent txn'.
	self isolate:
			[self inform: 'Inside parent txn'.
			self isolate: [self inform: 'Inside child txn'].
			self inform: 'Outside child txn'].
	self inform: 'Outside parent txn'
\end{code}

\dothis{Go to \url{http://localhost:8080/seaside/tests/alltests}, select
\menu{WATransactionTest} and click on \button{Restart}.
Try to navigate back and forth within the parent and child transaction by clicking the
\button{back} button and then clicking \button{ok}.
Note that as soon as a transaction is complete, you can no longer go back inside the
transaction without generating an error upon clicking \button{ok}.}

%=================================================================
\section{A complete tutorial example}

% ON: Should take about two hours

Let's see how we can build a complete Seaside application from scratch.\footnote{The
exercise should take at most a couple of hours. If you prefer to just look at the
completed source code, you can grab it from the SqueakSource project
\url{http://www.squeaksource.com/PharoByExample}.
The package to load is \scat{PBE-SeasideRPN}. The tutorial that follows uses slightly
different class names so that you can compare your implementation with ours.}
We will build a RPN (Reverse Polish Notation) calculator as a Seaside application that
uses a simple stack machine as its underlying model.
Furthermore, the Seaside interface will let us toggle between two displays\,---\,one
which just shows us the current value on top of the stack, and the other which shows us
the complete state of the stack.
The calculator with the two display options is shown in \figref{stackMachine}.

\begin{figure}[ht]
\begin{center}
\includegraphics[width=\textwidth]{stackMachine}
\caption{RPN calculator and its stack machine}
\figlabel{stackMachine}
\end{center}
\end{figure}

We begin by implementing the stack machine and its tests.

\dothis{Define a new class called \ct{MyStackMachine} with an instance variable
\ct{contents} initialized to a new \ct{OrderedCollection}.}

\begin{code}{}
MyStackMachine>>>initialize
	super initialize.
	contents := OrderedCollection new.
\end{code}

The stack machine should provide operations to \ct{push:} and \ct{pop} values, view the
\ct{top} of the stack, and perform various arithmetic operations to add, subtract,
multiply and divide the top values on the stack.

\dothis{Write some tests for the stack operations and then implement these operations.
Here is a sample test:}

\needlines{4}
\begin{code}{}
MyStackMachineTest>>>testDiv
	stack
		push: 3;
		push: 4;
		div.
	self assert: stack size = 1.
	self assert: stack top = (4/3).
\end{code}

You might consider using some helper methods for the arithmetic operations to check that
there are two numbers on the stack before doing anything, and raising an error if this
precondition is not fulfilled.\footnote{It's a good idea to use \ct{Object>>>assert:} to
specify the preconditions for an operation.
This method will raise an \ct{AssertionFailure} if the user tries to use the stack
machine in an invalid state.}
If you do this, most  of your methods will just be one or two lines long.

You might also consider implementing \ct{MyStackMachine>>>printOn:} to make it easier to
debug your stack machine implementation with the help of an object inspector.
(Hint: just delegate printing to the \ct{contents} variable.)
\index{Object!printOn:@\ct{printOn:}}

\dothis{Complete the \ct{MyStackMachine} by writing operations \ct{dup} (push a duplicateof the top value onto the stack), \ct{exch} (exchange the top two values), and \ct{rotUp}(rotate the entire stack contents up\,---\,the top value will move to the bottom).}

Now we have a simple stack machine implementation.
We can start to implement the Seaside RPN Calculator.

We will make use of 5 classes:
\begin{itemize}
\item \ct{MyRPNWidget}\,---\,this should be an abstract class that defines the common CSSstyle sheet for the application, and other common behavior for the components of the RPN
calculator.
It is a subclass of \ct{WAComponent} and the direct superclass of the following four
classes.
  
    \item \ct{MyCalculator}\,---\,this is the root component.
It should register the application (on the class side), it should instantiate and render
its subcomponents, and it should register any state for backtracking.
\item \ct{MyKeypad}\,---\,this displays the keys that we use to interact with the
calculator.
\item \ct{MyDisplay}\,---\,this component displays the top of the stack and provides a
button to call another component to display the detailed view.
\item \ct{MyDisplayStack}\,---\,this component shows the detailed view of the stack and
provides a button to answer back.
  It is a subclass of \lct{MyDisplay}.
\end{itemize}

\dothis{Define \ct{MyRPNWidget} in the category \ct{MyCalculator}.
Define the common \ct{style} for the application.}

Here is a minimal CSS for the application.
You can make it more fancy if you like.
\begin{code}{}
MyRPNWidget>>>style
	^ 'table.keypad { float: left; }
td.key {
	border: 1px solid grey;
	background: lightgrey;
	padding: 4px;
	text-align: center;
}
table.stack { float: left; }
td.stackcell {
	border: 2px solid white;
	border-left-color: grey;
	border-right-color: grey;
	border-bottom-color: grey;
	padding: 4px;
	text-align: right;
}
td.small { font-size: 8pt; }'
\end{code}

\dothis{Define \ct{MyCalculator} to be a root component and register itself as an
application (\ie implement \ct{canBeRoot} and \ct{initialize} on the class side).
Implement \ct{MyCalculator>>>renderContentOn:} to render something trivial (such as its
name), and verify that the application runs in a browser.
}

\ct{MyCalculator} is responsible for instantiating \ct{MyStackMachine}, \ct{MyKeypad} and\ct{MyDisplay}.

\dothis{
Define \ct{MyKeypad} and \ct{MyDisplay} as subclasses of \lct{MyRPNWidget}.
All three components will need access to a common instance of the stack machine, so
define the instance variable \ct{stackMachine} and an initialization method
\ct{setMyStackMachine:} in the common parent, \ct{MyRPNWidget}.
Add instance variables \ct{keypad} and \ct{display} to \ct{MyCalculator} and initialize
them in \ct{MyCalculator>>>initialize}.
(Don't forget to send \lct{super initialize}!)}

\dothis{
Pass the shared instance of the stack machine to the keypad and the display in the same
initialize method.
Implement \ct{MyCalculator>>>renderContentOn:} to simply render in turn the keypad and
the display.
To correctly display the subcomponents, you must implement \ct{MyCalculator>>>children}
to return an array with the keypad and the display.
Implement placeholder rendering methods for the keypad and the display and verify that
the calculator now displays its two subcomponents.
}

%\ab{Too long!}

Now we will change the implementation of the display to show the top value of the stack.

\dothis{
Use a table with class ``keypad'' containing a row with a single table data cell with
class ``stackcell''.
Change the rendering method of the keypad to ensure that the number 0 is pushed on the
stack in case it is empty.
(Define and use \ct{MyKeypad>>>ensureMyStackMachineNotEmpty}.)
Also make it display an empty table with class ``keypad''.
Now the calculator should display a single cell containing the value 0.
If you toggle the halos, you should see something like this:
}

\begin{figure}[ht]
\begin{center}
\includegraphics[width=0.8\textwidth]{firstStackDisplay}
\caption{Displaying the top of the stack}
\figlabel{firstStackDisplay}
\end{center}
\end{figure}

Now let's implement an interface to interact with the stack.

\dothis{
First define the following helper methods, which will make it easier to script the
interface:
}

\needlines{3}
\begin{code}{}
MyKeypad>>>renderStackButton: text callback: aBlock colSpan: anInteger on: html 
	html tableData
		class: 'key';
		colSpan: anInteger;
		with: 
				[html anchor
					callback: aBlock;
					with: [html html: text]]
\end{code}


\begin{code}{}
MyKeypad>>>renderStackButton: text callback: aBlock on: html 
	self 
		renderStackButton: text
		callback: aBlock
		colSpan: 1
		on: html
\end{code}

We will use these two methods to define the buttons on the keypad with appropriate
callbacks.
Certain buttons may span multiple columns, but the default is to occupy just one column.

\dothis{
Use the two helper methods to script the keypad as follows:
(Hint: start by getting the digit and ``Enter'' keys working, then the arithmetic
operators.)
}

\needlines{4}
\begin{code}{}
MyKeypad>>>renderContentOn: html 
  self ensureStackMachineNotEmpty.
  html table
    class: 'keypad';
    with: [
      html tableRow: [
          self renderStackButton: '+' callback: [self stackOp: #add] on: html.
          self renderStackButton: '&ndash;' callback: [self stackOp: #min] on: html.
          self renderStackButton: '&times;' callback: [self stackOp: #mul] on: html.
          self renderStackButton: '&divide;' callback: [self stackOp: #div] on: html.
          self renderStackButton: '&plusmn;' callback: [self stackOp: #neg] on: html ].
        html tableRow: [
          self renderStackButton: '1' callback: [self type: '1'] on: html.
          self renderStackButton: '2' callback: [self type: '2'] on: html.
          self renderStackButton: '3' callback: [self type: '3'] on: html.
          self renderStackButton: 'Drop' callback: [self stackOp: #pop]
          	colSpan: 2 on: html ].
" and so on ... "
        html tableRow: [
          self renderStackButton: '0' callback: [self type: '0'] colSpan: 2 on: html.
          self renderStackButton: 'C' callback: [self stackClearTop] on: html.
          self renderStackButton: 'Enter'
          	callback: [self stackOp: #dup. self setClearMode]
			colSpan: 2 on: html ]]
\end{code}

Check that the keypad displays properly.
If you try to click on the keys, however, you will find that the calculator does not workyet ...

\dothis{
Implement \ct{MyKeypad>>>type:} to update the top of the stack by appending the typed
digit.
You will need to convert the top value to a string, update it, and convert it back to an
integer, something like this:
}
\begin{code}{}
MyKeypad>>>type: aString
	stackMachine push: (stackMachine pop asString, aString) asNumber.
\end{code}
Now when you click on the digit keys the display should be updated.
(Be sure that \ct{MyStackMachine>>>pop} returns the value popped, or this will not work!)
\dothis{Now we must implement \ct{MyKeypad>>>stackOp:}
Something like this will do the trick:}

\begin{code}{}
MyKeypad>>>stackOp: op
	[ stackMachine perform: op ] on: AssertionFailure do: [ ].
\end{code}

The point is that we are not sure that all operations will succeed, for example, additionwill fail if we do not have two numbers on the stack.
For the moment we can just ignore such errors.
If we are feeling more ambitious later on, we can provide some user feedback in the errorhandler block.

\dothis{The first version of the calculator should be working now.
Try to enter some numbers by pressing the digit keys, hitting \menu{Enter} to push a copyof the current value, and entering \menu{+} to sum the top two values.}

You will notice that typing digits does not behave the way you might expect.
Actually the calculator should be aware of whether you are typing a \emph{new} number, orappending to an existing number.

\dothis{Adapt \ct{MyKeypad>>>type:} to behave differently depending on the current typingmode.
Introduce an instance variable \ct{mode} which takes on one of the three values
\lct{typing} (when you are typing), \lct{push} (after you you have performed a calculatoroperation and typing should force the top value to be pushed), or \lct{clear} (after you
have performed \menu{Enter} and the top value should be cleared before typing).
The new \ct{type:} method might look like this:
}

\begin{code}{}
MyKeypad>>>type: aString
	self inPushMode ifTrue: [
		stackMachine push: stackMachine top.
		self stackClearTop ].
	self inClearMode ifTrue: [ self stackClearTop ].
	stackMachine push: (stackMachine pop asString, aString) asNumber.
\end{code}

Typing might work better now, but it is still frustrating not to be able to see what is
on the stack.

\dothis{
Define \ct{MyDisplayStack} as a subclass of \ct{MyDisplay}.
Add a button to the rendering method of \ct{MyDisplay} which will call a new instance of
\ct{MyDisplayStack}.
You will need an html anchor that looks something like this:
}

\begin{code}{}
html anchor
	callback: [ self call: (MyDisplayStack new setMyStackMachine: stackMachine)];
	with: 'open'
\end{code}

The callback will cause the current instance of \ct{MyDisplay} to be temporarily replacedby a new instance of \ct{MyDisplayStack} whose job it is to display the complete stack.
When this component signals that it is done (\ie by sending \ct{self answer}), then
control will return to the original instance of \ct{MyDisplay}.

\dothis{
Define the rendering method of \ct{MyDisplayStack} to display all of the values on the
stack.
(You will either need to define an accessor for the stack machine's \ct{contents} or you
can define \ct{MyStackMachine>>>do:} to iterate over the stack values.)
The stack display should also have a button labelled ``close'' whose callback will simplyperform \ct{self answer}.
}

\begin{code}{}
html anchor
	callback: [ self answer];
	with: 'close'
\end{code}

Now you should be able to \emph{open} and \emph{close} the stack while you are using the
calculator.

There is, however, one thing we have forgotten.
Try to perform some operations on the stack.
Now use the ``back'' button of your browser and try to perform some more stack
operations.
(For example, \menu{open} the stack, type \menu{1}, \menu{Enter} twice and \menu {+}.
The stack should display ``2'' and ``1''.
Now hit the ``back'' button.
The stack now shows three times ``1'' again.
Now if you type \menu{+} the stack shows ``3''.
Backtracking is not yet working.

\dothis{
Implement \ct{MyCalculator>>>states} to return the contents of the stack machine.
Check that backtracking now works correctly!
}

Sit back and enjoy a tall glass of something cool!

%=================================================================
\section{A quick look at AJAX}

% Original text by Lukas Renggli

\ind{AJAX} (Asynchronous \ind{JavaScript} and \ind{XML}) is a technique to make web
applications more interactive by exploiting JavaScript functionality on the client side.

Two well-known JavaScript libraries are \ind{Prototype}
(\url{http://www.prototypejs.org}) and \ind{script.aculo.us}
(\url{http://script.aculo.us}).
Prototype provides a framework to ease writing JavaScript.
script.aculo.us provides some additional features to support animations and drag-and-dropon top of Prototype.
Both frameworks are supported in Seaside through the package ``Scriptaculous''.

All ready-made images have the Scriptaculous package extensions already loaded.
The latest version is available from \url{http://www.squeaksource.com/Seaside}.
An online demo is available at \url{http://scriptaculous.seasidehosting.st}.
Alternatively, if you have a enabled image running, simply go to
\url{http://localhost:8080/seaside/tests/scriptaculous}.

The Scriptaculous extensions follow the same approach as Seaside itself\,---\,simply
configure Smalltalk objects to model your application, and the needed Javascript code
will be generated for you.

Let us look at a simple example of how client-side Javascript support can make our RPN
calculator behave more naturally.
Currently every keystroke to enter a digit generates a request to refresh the page.
We would like instead to handle editing of the display on the client-side by updating thedisplay in the existing page.

\dothis{To address the display from JavaScript code we must first give it a unique id.
Update the calculator's rendering method as follows:\footnote{If you have not implementedthe tutorial example yourself, you can simply load the complete example (PBE-SeasideRPN)
from \url{http://www.squeaksource.com/PharoByExample} and apply the suggested changes to
the classes \ct{RPN*} instead of \ct{My*}.}}

\begin{code}{}
MyCalculator>>>renderContentOn: html
	html div id: 'keypad'; with: keypad.
	html div id: 'display'; with: display.	
\end{code}
				
\dothis{To be able to re-render the display when a keyboard button is pressed, the
keyboard needs to know the display component.
Add a \ct{display} instance variable to \ct{MyKeypad}, an initialize method
\ct{MyKeypad>>>setDisplay:}, and call this from \ct{MyCalculator>>initialize}.
Now we are able to assign some JavaScript code to the buttons by updating
\ct{MyKeypad>>>renderStackButton:callback:colSpan:on:} as follows:}

\begin{code}{}
MyKeypad>>>renderStackButton: text callback: aBlock colSpan: anInteger on: html 
	html tableData
		class: 'key';
		colSpan: anInteger;
		with: [
			html anchor
				callback: aBlock;
				onClick:				"handle Javascript event"
					(html updater
						id: 'display';
						callback: [ :r |
							aBlock value.
							r render: display ];
						return: false);
				with: [ html html: text ] ]
\end{code}

\mthind{WATagBrush}{onClick:} specifies a \ind{JavaScript} event handler.
\ct{html updater} returns an instance of \ct{SUUpdater}, a Smalltalk object representing
the JavaScript Ajax.Updater object (\url{http://www.prototypejs.org/api/ajax/updater}).
This object performs an AJAX request and updates a container's contents based on the
response text.
\ct{id:} tells the updater what XHTML DOM element to update, in this case the contents ofthe div element with the id 'display'.
\ct{callback:} specifies a block that is triggered when the user presses the button.
The block argument is a new renderer \ct{r}, which we can use to render the display
component.
(Note: Even though html is still accessible, it is not valid anymore at the time this
callback block is evaluated).
Before rendering the display component we evaluate \ct{aBlock} to perform the desired
action.

\ct{return: false} tells the JavaScript engine to not trigger the original link callback,which would cause a full refresh.
We could instead remove the original anchor \ct{callback:}, but like this the calculator
will still work even if JavaScript is disabled.

\dothis{Try the calculator again, and notice how a full page refresh is triggered every
time you press a digit key. (The URL of the web page is updated at each keystroke.)}

Although we have implemented the client-side behavior, we have not yet activated it.
Now we will enable the Javascript event handling.

\dothis{
Click on the \link{Configure} link in the toolbar of the calculator.
Select ``Add Library:'' \ct{SULibrary}, click the \button{Add} button and
\button{Close}.}

Instead of manually adding the library, you may also do it programmatically when you
register the application:
\begin{code}{}
MyCalculator class>>>initialize
	(self registerAsApplication: 'rpn')
		addLibrary: SULibrary}}
\end{code}

\begin{figure}[ht]
\begin{center}
\includegraphics[width=\textwidth]{ajax-processing}
\caption{Seaside AJAX processing (simplified)}
\figlabel{ajax-processing}
\end{center}
\end{figure}

\dothis{Try the revised application. Note that the feedback is much more natural. In
particular, a new URL is not generated with each keystroke.}

You may well ask, \emph{yes, but how does this work?}
\figref{ajax-processing} shows how the RPN applications would both without and with
AJAX.Basically AJAX short-circuits the rendering to \emph{only} update the display
component.
Javascript is responsible both for triggering the request and updating the corresponding
DOM element.
Have a look at the generated source-code, especially the JavaScript code:

\begin{code}{}
new Ajax.Updater(
	'display',
	'http://localhost/seaside/RPN+Calculator',
	{'evalScripts': true,
'parameters': ['UNDERSCOREs=zcdqfonqwbeYzkza',
'UNDERSCOREk=jMORHtqr','9'].join('&')});return false
\end{code}

For more advanced examples, have a further look at
\url{http://localhost:8080/seaside/tests/scriptaculous}.

\paragraph{\emph{Hints.}}
In case of server side problems use the Smalltalk debugger.
In case of client side problems use FireFox (\url{http://www.mozilla.com}) with the
JavaScript debugger FireBug (\url{http://www.getfirebug.com/}) plugin enabled.

%=================================================================
\section{Chapter summary}

\begin{itemize}
\item The easiest way to get started is to download the ``Seaside One-Click Experience''
from \url{http://seaside.st}
\item Turn the server on and off by evaluating \ct{WAKom startOn: 8080} and \ct{WAKom stop}.
\item Reset the administrator login and password by evaluating \ct{WADispatcherEditor initialize}.
\item \menu{Toggle Halos} to directly view application source code, run-time objects, CSSand XHTML.
  \item Send \ct{WAGlobalConfiguration setDeploymentMode} to hide the toolbar.
\item Seaside web applications are composed of components, each of which is an instance
of a subclass of \ct{WAComponent}.
\item Only a root component may be registered as a component. It should implement
\ct{canBeRoot} on the class side. Alternatively it may register itself as an application
in its class-side \ct{initialize} method by sending \ct{self registerAsApplication:}
\emph{application path}.
If you override \ct{description} it is possible to return a descriptive application name
that will be displayed in the configuration editor.
\item To backtrack state, a component must implement the \ct{states} method to answer an
array of objects whose state will be restored if the user clicks the browser's ``back''
button.
  \item A component renders itself by implementing \ct{renderContentOn:}.
The argument to this method is an XHTML rendering \emph{canvas} (usually called
\ct{html}).
\item A component can render a subcomponent by sending \ct{self render:}
\emph{subcomponent}.
\item XHTML is generated programmatically by sending messages to \emph{brushes}. A brush
is obtained by sending a message, such as \ct{paragraph} or \ct{div}, to the html canvas.\item If you send a cascade of messages to a brush that includes the message \ct{with:},
then \ct{with:} should be the last message sent.
  Thw \ct{with:} message sets the contents \emph{and} renders the result.
  \item Actions should appear only in callbacks.
You should not change the state of the application while you are rendering it.
\item You can bind various form widgets and anchors to instance variables with accessors
by sending the message \ct{on:} \emph{instance variable} \ct{of:} \emph{object} to the
brush.
\item You can define the CSS for a component hierarchy by defining the method \ct{style},which should return a string containing the style sheet.
(For deployed applications, it is more usual to refer to a style sheet located at a
static URL.)
\item Control flows can be programmed by sending \ct{x call: y}, in which case component
\ct{x} will be replaced by \ct{y} until \ct{y} answers by sending \ct{answer:} with a
result in a callback.
The receiver of \ct{call:} is usually \ct{self}, but may in general be any visible
component.
\item A control flow can also be specified as a \emph{task}\,---\,a instance of a
subclass of \ct{WATask}. It should implement the method \ct{go}, which should \ct{call:}
a series of components in a workflow.
\item Use \ct{WAComponents}'s convenience methods \ct{request:}, \ct{inform:},
\ct{confirm:} and \ct{chooseFrom:caption:} for basic interactions.
\item To prevent the user from using the browser's ``back'' button to access a previous
execution state of the web application, you can declare portions of the workflow to be a
\emph{transaction} by enclosing them in an \ct{isolate:} block.
\end{itemize}
%-----------------------------------------------------------------

%=================================================================
\ifx\wholebook\relax\else 
   \bibliographystyle{jurabib}
   \nobibliography{scg}
   \end{document}
\fi
%=================================================================
