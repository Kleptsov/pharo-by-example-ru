% $Author: oscar $
% $Date: 2009-08-27 16:59:05 +0600 (чт, 27 авг 2009) $
% $Revision: 28624 $

% HISTORY:
% 2007-06-06 - Stef started
% 2007-08-21 - Oscar edit
% 2007-09-06 - Lukas corrections
% 2007-09-11 - Orla corrections

%=================================================================
\ifx\wholebook\relax\else
% --------------------------------------------
% Lulu:
	\documentclass[a4paper,10pt,twoside]{book}
	\usepackage[
		papersize={6.13in,9.21in},
		hmargin={.75in,.75in},
		vmargin={.75in,1in},
		ignoreheadfoot
	]{geometry}
	\input{../common.tex}
	\pagestyle{headings}
	\setboolean{lulu}{true}
% --------------------------------------------
% A4:
%	\documentclass[a4paper,11pt,twoside]{book}
%	\input{../common.tex}
%	\usepackage{a4wide}
% --------------------------------------------
    \graphicspath{{figures/} {../figures/}}
	\begin{document}
	\renewcommand{\nnbb}[2]{} % Disable editorial comments
	\sloppy
\fi
%=================================================================
%\chapter{Understanding message syntax}
\chapter{\ugh{Подробнее о синтаксисе сообщений}}
\chalabel{understanding}

%Although \st's message syntax is extremely simple, it is unconventional and can take some time getting used to.
Синтаксис сообщений в \st хотя и предельно прост, но всё же требует некоторого времени для освоения, ввиду своей непривычности.
%This chapter offers some guidance to help you get acclimatized to this special syntax for sending messages.
Эта глава поможет вам разобраться в специальном синтаксисе посылки сообщений.
%If you already feel comfortable with the syntax, you may choose to skip this chapter, or come back to it later.
Но если вы уже чувствуете себя в нем комфортно, можете смело её пропустить, или вернуться к ней позже.

%=============================================================
%\section{Identifying messages}
\section{Идентификация сообщений}

%In \st, except for the syntactic elements listed in \charef{syntax} (\ct+:= ^ . ; # () {} [ : | ]+), everything is a message send. 
В \st всё является посылкой сообщений, кроме элементов синтаксиса, описанных в \charef{syntax} (\ct+:= ^ . ; # () {} [ : | ]+).
%As in \ind{C++}, you can define operators like \ct{+} for your own classes, but all operators have the same precedence.
Как и в \ind{C++}, вы можете определять операторы (например, \ct{+}) для ваших классов, но все они будут иметь одинаковый приоритет выполнения.
%Moreover, you cannot change the arity of a method. ``\ct{-}'' is always a binary message; there is no way to have a unary ``\ct{-}'' with a different overloading.
Более того, вы не можете изменить арность (число аргументов) метода. ``\ct{-}'' -- всегда бинарное сообщение, и не существует способа, перегрузив (overload) его, определить унарный ``\ct{-}''.

%In \st the order in which messages are sent is determined by the kind of message. There are just three kinds of messages: \emphsubind{message}{unary}, \emphsubind{message}{binary}, and \emphsubind{message}{keyword} messages. Unary messages are always sent first, then binary messages and finally keyword ones. As in most languages, \ind{parentheses} can be used to change the order of evaluation. These rules make \st code as easy to read as possible. And most of the time you do not have to think about the rules.
В \st порядок в котором посылаются сообщения определяется видом сообщения. Таковых видов имеется три: \emphsubind{message}{унарные сообщения (unary)}, \emphsubind{message}{бинарные сообщения (binary)}, and \emphsubind{message}{сообщения с ключевыми словами (keyword)}. Унарные сообщения всегда посылаются первыми, затем идут бинарные сообщения, и, наконец, сообщения с ключевыми словами. Как и в большинстве языков, \ind{скобки} могут быть использованы для изменения порядка выполнения. Эти правила облегчают, насколько возможно, чтение \st кода, так, что в б'ольшую часть времени вы даже не будете о них вспоминать.

%As most computation in \st is done by message passing, correctly identifying messages is crucial. The following terminology will help us:
Т.к. практически все вычисления в \st выполняются путём обмена сообщениями, их идентификация является ключевым навыком. В этом нам поможет следующая терминология:

\begin{itemize}
  %\item A message is composed of the message \emphsubind{message}{selector} and the optional message arguments.
  \item Сообщение состоит из \emphsubind{message}{селектора (selector)} и необязательного списка аргументов.
  %\item A message is sent to a \emphsubind{message}{receiver}.
  \item Сообщение отправляется \emphsubind{message}{получателю (receiver)}.
  %\item The combination of a message and its receiver is called a \emph{message} \emphsubind{message}{send} as shown in \figref{firstScriptMessage}.
  \item Комбинация сообщения и его получателя называется \emphsubind{message}{посылкой сообщения (message send)}, как показано на \figref{firstScriptMessage}.
\end{itemize}

\begin{figure}[htb]
\begin{minipage}{0.53\textwidth}
	\begin{center}
	\includegraphics[width=0.95\textwidth]{message}
	%\caption{Two messages composed of a receiver, a method selector, and a set of arguments.\figlabel{firstScriptMessage}}\end{center}
	\caption{Два сообщения, состоящие из получателя и селектора, и набор аргументов.}\figlabel{firstScriptMessage}}\end{center}
\end{minipage}
\hfill
\begin{minipage}{0.43\textwidth}
	\begin{center}
	\ifluluelse
		{\includegraphics[width=0.9\textwidth]{uKeyUnOne}}
		{\includegraphics[width=6cm]{uKeyUnOne}}
	%\caption{\ct{aMorph color: Color yellow} is composed of two message sends: \ct{Color yellow} and \ct{aMorph color: Color yellow}.\figlabel{ellipse}}
	\caption{Выражение \ct{aMorph color: Color yellow}, включающее в себя посылку двух сообщений: \ct{Color yellow} и \ct{aMorph color: Color yellow}.\figlabel{ellipse}}
	\end{center}
\end{minipage}
\end{figure}

%\important{A message is always sent to a receiver, which can be a single literal, a block or a variable or the result of evaluating another message.}
\important{Сообщение всегда отправляется получателю, который может быть простым литералом, блоком, переменной или результатом выполнения другого сообщения.}
%sd-this is not totally right because we do not cover assignment but ok for now

%To help you identify the receiver of a message, we will underline it for you.
Мы будем выделять получатель сообщения подчёркиваением, чтобы помочь вам распознать его.
%We will also surround each message send with an ellipse and number message sends starting from the first one that will be sent to help you see the order in which messages are sent.
Мы также будем заключать каждое сообщение в овал, а также нумеровать сообщения, начиная с первого отправленного, чтобы помочь вам разобраться с порядком посылки сообщений.

%\begin{figure}[!ht]
%\begin{center}
%\includegraphics[width=6cm]{uKeyUnOne}
%\end{center}
%\caption{\ct{aMorph color: Color yellow} is composed of two expressions: \ct{Color yellow} and \ct{aMorph color: Color yellow}.\figlabel{ellipse}}
%\end{figure}

%\figref{ellipse} represents two message sends, \ct{Color yellow} and \ct{aMorph color: Color yellow}, hence there are two ellipses. The message send \ct{Color yellow} is executed first so its ellipse is numbered \ct{1}.  There are two receivers: \ct{aMorph} which receives the message \ct{color: ...} and \ct{Color} which receives the message \ct{yellow}. Both receivers are underlined.
На \figref{ellipse} показана посылка двух сообщений: \ct{Color yellow} и \ct{aMorph color: Color yellow}, поэтому там присутствуют два овала. Посылка сообщения \ct{Color yellow} выполняется первой, и соответствующий овал пронумерован единицей. Оба получателя: \ct{aMorph}, которому адресовано сообщение \ct{color: ...} и \ct{Color}, которому посылается сообщение \ct{yellow}, выделены подчёркиванием.

%A receiver can be the first element of a message, such as \ct{100} in the message send \ct{100 + 200} or \ct{Color} in the message send \ct{Color yellow}. However, a receiver can also be the result of other messages. For example in the message \ct{Pen new go: 100}, the receiver of the message \ct{go: 100} is the object returned by the message send \ct{Pen new}. In all the cases, a message is sent to an object called the \emph{receiver} which may be the result of another message send.
Получатель может быть первым элементом сообщения, как, например, число \ct{100} в посылке \ct{100 + 200}, или \ct{Color} в \ct{Color yellow}. Однако получатель может быть результатом другого сообщения. Например, в \ct{Pen new go: 100} получателем сообщения \ct{go: 100} является объект, возвращаемый в результате посылки сообщения \ct{Pen new}. Во всех этих случаях, сообщение посылается объекту, называемому \emph{получателем (receiver)}, который, в свою очередь, может быть результатом посылки другого сообщения.

\begin{table}\centering
	\begin{tabularx}{\linewidth}{llX}
		\toprule
		%Message send & Message type & Result \\
		Посылка сообщения & Тип сообщения & Результат
		\midrule
		\lct{Color yellow}
			%& unary
			& унарное 
			%& Creates a color.
			& Создаёт объект для представления цвета.
		\\
		\lct{aPen  go: 100.}
			%& keyword
			& с ключевыми словами
			%& The receiving pen moves forward 100 pixels.
			& Продвигает перо на 100 пикселей вперёд.
		\\
		\lct{100 + 20}
			%& binary
			& бинарное
			%& The number 100 receives the message + with the number 20.
			& Число 100 получает сообщение + с числом 20 в качестве аргумента.
		\\
		\lct{Browser open}
			%& unary
			& унарное 
			%& Opens a new browser.
			& Открывает новый браузер.
		\\
		\lct{Pen new  go: 100}
			%& unary and keyword
			& унарное и с ключевыми словами
			%& A pen is created and moved 100 pixels.
			& Создаётся перо и продигается на 100 пикселей.
		\\
		\lct{aPen go: 100 + 20}
			%& keyword and binary
			& с ключевыми словами и бинарное
			%& The receiving pen moves forward 120 pixels.
			& Перо продвигается вперёд на 120 пикселей.
		\\
		\bottomrule
	\end{tabularx}
	%\caption{Examples of Message Sends and their Types}\tablabel{messageExamples}
	\caption{Примеры посылок сообщений и их типов}\tablabel{messageExamples}
\end{table}

%\tabref{messageExamples} shows several examples of message sends.
\tabref{messageExamples} показывает несколько примеров посылки сообщений.
%You should note that not all message sends have arguments. Unary messages like \ct{open} do not have arguments. Single keyword and binary messages like \ct{go: 100} and \ct{+ 20} each have one argument. 
Заметьте, что не все сообщения посылаются с аргументами. Унарные сообщения, такие как \ct{open}, не имеют аргументов. Сообщения с одним ключевым словом и бинарные сообщения, как \ct{go: 100} и \ct{+ 20}, принимают по одному аргументу.
%There are also simple messages and composed ones. \ct{Color yellow} and \ct{100 + 20} are  simple:  a message is sent to an object, while the message send \ct{aPen go: 100 + 20} is composed of two messages: \ct{+ 20} is sent to \ct{100} and \ct{go:} is sent to \ct{aPen} with the argument being the result of the first message.
Существуют также простые и составные сообщения. \ct{Color yellow} и \ct{100 + 20} являются простыми, одно сообщение посылается одному объекту, в то время как посылка \ct{aPen go: 100 + 20} состоит из двух сообщений: \ct{+ 20} посылается объекту \ct{100} и \ct{go:} посылается объекту \ct{aPen} с результатом предыдущего сообщения в качестве аргумента.
%A receiver can be an expression (such as an assignment, a message send or a literal) which returns an object. In \ct{Pen new go: 100}, the message \ct{go: 100} is sent to the object that results from the execution of the message send \ct{Pen new}.
Получателем может быть выражение (такое, как присваивание, посылка сообщения или литерал), которое возвращает объект. В посылке \ct{Pen new go: 100}, сообщение \ct{go: 100} отправляется объекту, являющемуся результатом выполнения посылки \ct{Pen new}.


%=============================================================
%\section{Three kinds of messages}
\section{Три типа сообщений}

%\st defines a few simple rules to determine the order in which the messages are sent. These rules are based on the distinction between 3 different kinds of messages: 
В \st есть несколько простых правил, определяющих порядок посылки сообщений. Эти правила основываются на различях между тремя типами сообщений:
\begin{itemize}
%\item \emph{Unary messages} are messages that are sent to an object without any other information. For example in \ct{3 factorial}, \ct{factorial} is a unary message. 
\item \emph{Унарные сообщения} посылаются объекту безо всякой дополнительной информации. Например, в посылке \ct{3 factorial}, \ct{factorial} -- это унарное сообщение
%\item  \emph{Binary messages} are messages consisting of operators (often arithmetic). They are binary because they always involve only two objects: the receiver and the argument object. For example in \ct{10 + 20}, \ct{+} is a binary message sent to the receiver \ct{10} with argument \ct{20}. 
\item  \emph{Бинарные сообщения} составляются из операторов (часто арифметических). Они называются бинарными, потому что всегда затрагивают только два объекта: получателя и аргумент. Например, в \ct{10 + 20},  \ct{+} -- бинарное сообщение, отправляемое получателю \ct{10} с аргументом \ct{20}.
%\item  \emph{Keyword messages} are messages consisting of one or more keywords, each ending with a colon (\ct{:}) and taking an argument.
\item  \emph{Сообщения с ключевыми словами} состоят из одного или более слов, оканчивающихся двоеточием (\ct{:}), за которым следует аргумент.
%For example in \ct{anArray at: 1 put: 10}, the keyword \ct{at:} takes the argument \ct{1} and the keyword \ct{put:} takes the argument \ct{10}.
Например, в \ct{anArray at: 1 put: 10}, ключевому слову \ct{at:} соответствует аргумент \ct{1}, а ключевому слову \ct{put:} -- аргумент \ct{10}.
\end{itemize}

%-------------------------------------------------------------
%\subsection{Unary messages}
\subsection{Унарные сообщения}
%Unary messages are messages that do not require any argument. They follow the syntactic template: \ct{receiver messageName}. The selector is simply made up of a succession of characters not containing \ct{:} (\eg \ct{factorial}, \ct{open}, \ct{class}).
Унарные сообщения не принимают аргументов и соответсвуют следующему синтаксическому шаблону: \ct{получатель имяСообщения}. Селектор состоит из последовательности символов (characters) не содержащей двоеточия, например, \ct{factorial}, \ct{open}, \ct{class}).
\needlines{4}
\begin{code}{@TEST}
89 sin           --> 0.860069405812453
3 sqrt           --> 1.732050807568877
Float pi         --> 3.141592653589793
'blop' size     --> 4
true not        --> false
Object class --> Object class  "Класс класса Object -- это Object class (BANG)"
\end{code}
% ON: I changed the examples to things we can test

%\important{Unary messages are messages that do not require any argument.\\
\important{Унарные сообщения не принимают аргументов.\\
%They follow the syntactic template: \lct{receiver \textbf{selector}}}
Они соответствуют синтаксическому шаблону: \lct{получатель \textbf{селектор}}}

%-------------------------------------------------------------
%\subsection{Binary messages} 
\subsection{Бинарные сообщения} 
%Binary messages are messages that require exactly one argument \emph{and} whose selector consists of a sequence of one or more characters from the set: \ct{+}, \ct{-}, \ct{*}, \ct{/}, \ct{&}, \ct{=}, \ct{>}, \ct{|}, \ct{<}, \ct{~}, and \ct{@}. Note that \ct{--} is not allowed for parsing reasons.
Бинарные сообщения принимают ровно один аргумент \emph{и} их селектор состоит из одного или более следующих симвовлов: \ct{+}, \ct{-}, \ct{*}, \ct{/}, \ct{&}, \ct{=}, \ct{>}, \ct{|}, \ct{<}, \ct{~} и \ct{@}. Заметьте, что использовать \ct{--} не разрешается по причинам, связанным с синтаксическим анализом.

\begin{code}{@TEST}
100@100      --> 100@100  "создаёт объект класса Point"
3 + 4              --> 7
10 - 1            --> 9
4 <= 3            --> false
(4/3) * 3 = 4   --> true  "равенство -- это бинарное сообщение, а дроби (Fractions) -- обыкновенные"
(3/4) == (3/4) --> false  "равные дроби -- разные объекты"
\end{code}

%\important{Binary messages are messages that require exactly one argument \emph{and} whose selector is composed of a sequence of characters from: \ct{+}, \ct{-}, \ct{*}, \ct{/}, \ct{\&}, \ct{=}, \ct{>}, \ct{|}, \ct{<}, \ct{\~}, and \ct{@}. \ct{--} is not possible.\\
\important{Бинарные сообщения требуют ровно одного аргумента \emph{и} их селектор состоит из одного или более следующих символов: \ct{+}, \ct{-}, \ct{*}, \ct{/}, \ct{\&}, \ct{=}, \ct{>}, \ct{|}, \ct{<}, \ct{\~} и \ct{@}. Селектор \ct{--} запрещён.\\
%They follow the syntactic template: \lct{receiver \textbf{selector} argument}}
Они соответствуют синтаксическому шаблону: \lct{получатель \textbf{селектор} аргумент}}.

%-------------------------------------------------------------
%\subsection{Keyword messages} 
\subsection{Сообщения с ключевыми словами} 

%Keyword messages are messages that require one or more arguments and whose selector consists of one or more keywords each ending in \ct{:}.  Keyword messages follow the syntactic template: 
Сообщения с ключевыми словами принимают один или более аргументов и их селектор состоит из одного или более ключевых слов, каждое из которых оканчивается двоеточием. Сообщения с ключевыми словами соответствуют следующему синтаксическому шаблону:
%\lct{receiver \textbf{selectorWordOne:} argument\-One \textbf{wordTwo:} argumentTwo}
\lct{получатель \textbf{первоеКлючевоеСлово:} первыйАргумент \textbf{второеКлючевоеСлово:} второйАргумент}

%Each keyword takes an argument. Hence \ct{r:g:b:} is a method with three arguments, \ct{playFileNamed:} and \ct{at:} are methods with one argument, and \ct{at:put:} is a method with two arguments. To create an instance of the class \ct{Color} one can use the method \ct{r:g:b:} as in \ct{Color r: 1 g: 0 b: 0}, which creates the color red. Note that the colons are part of the selector.
Каждому ключевому слову сопоставлен один аргумент. Поэтому \ct{r:g:b:} -- это метод с тремя аргументами, \ct{playFileNamed:} и \ct{at:} -- методы с одним аргументом, а \ct{at:put:} -- с двумя. Чтобы создать экземпляр класса \ct{Color} можно использовать метод \ct{r:g:b:}, как в \ct{Color r: 1 g: 0 b: 0}, который \ugh{создаёт красный цвет}. Заметьте, что двоеточия -- это часть селектора.

%\important{In \ind{Java} or \ind{C++}, the \st method invocation \ct{Color r: 1 g: 0 b: 0}
\important{В \ind{Java} или \ind{C++}, вызов метода записанный на \st как \ct{Color r: 1 g: 0 b: 0}
 мог бы быть переписан на \ct{Color.rgb(1,0,0)}.}

\begin{code}{@TEST | nums |}
1 to: 10                        --> (1 to: 10)  "создаёт интервал"
Color r: 1 g: 0 b: 0       --> Color red  "создаёт новй цвет"
12 between: 8 and: 15 --> true

nums := Array newFrom: (1 to: 5).
nums at: 1 put: 6.
nums --> #(6 2 3 4 5)
\end{code}
% ON: Changed to real examples that we can test

%\important{Keyword based  messages are messages that require one or more arguments. Their selector consists of one or more keywords each ending in a colon (\lct{:}). They follow the syntactic template: \\
\important{Сообщения с ключевыми словами принимают один или более аргументов. Их селектор состоит из одного или более ключевых слов, каждое из которых оканчивается двоеточием. Они соответствуют следующему синтаксическому шаблону: \\
%\lct{receiver \textbf{selectorWordOne:} argumentOne \textbf{wordTwo:} argumentTwo}}
\lct{получатель \textbf{первоеКлючевоеСлово:} первыйАргумент \textbf{второеКлючевоеСлово:} второйАргумент}}

%=============================================================
%\section{Message composition}
\section{Композиция сообщений}
%The three kinds of messages each have different precedence, which allows them to be composed in an elegant way. 
Каждый из трёх типов сообщений имеет свой приоритет, что позволяет компоновать их весьма элегантным способом.

\begin{enumerate}
%\item Unary messages are always sent first, then binary messages and finally keyword messages. 
\item Унарные сообщения всегда посылаются первыми, затем идут бинарные сообщения и сообщения с ключевыми словами. 
%\item Messages in \ind{parentheses} are sent prior to any kind of messages. 
\item Сообщения в \ind{скобках (parentheses)} посылаются перед любым сообщением, не зависимо от типа.
%\item Messages of the same kind are evaluated from left to right. 
\item Сообщения одного типа вычисляются слева направо.
\end{enumerate}
\index{message!evaluation order}

%These rules lead to a very natural reading order. Now if you want to be sure that your messages are sent in the order that you want you can always put more parentheses as shown in \figref{uKeyUn}. In this  figure, the message \ct{yellow} is an unary message and the message \ct{color:} a keyword message, therefore the message send \ct{Color yellow} is sent first. However as message sends in parentheses are sent first, putting (unnecessary) parentheses around \ct{Color yellow} just emphasizes that it will be sent first. The rest of the section illustrates each of these points.
Благодаря этим правилам, порядок чтения становится очень близок к естественному. Если вы хотите быть уверенными, что ваши сообщения отправляются в нужном порядке, вы всегда можете добавить скобок, как показано на \figref{uKeyUn}. Здесь \ct{yellow} -- унарное сообщение, а \ct{color:} -- сообщение с ключевыми словами, поэтому посылка \ct{Color yellow} выполняется первой. Однако, т.к. сообщения в скобках отправляются первыми, заключение (необязательное) в них \ct{Color yellow} просто подчёркивает тот факт, что это сообщение будет отправлено первым. Остаток раздела по очереди иллюстрирует каждый из этих моментов.

\begin{figure}[ht]
\ifluluelse
	{\centerline{\includegraphics[width=0.9\textwidth]{uKeyUn}} }
	{\centerline{\includegraphics[width=10cm]{uKeyUn}} }
%\caption{Unary messages are sent first so \ct{Color yellow} is sent. This returns a color object which is passed as argument of the message \ct{aPen color:}.\figlabel{uKeyUn}}
\caption{\ct{Color yellow} посылается первым, поскольку это унарное сообщение. Возвращаемый им объект передаётся в качестве аргумента в \ct{aPen color:}.\figlabel{uKeyUn}}
\end{figure}

%---------------------------------------------------------
%\subsection*{Unary > Binary > Keywords}
\subsection*{Унарные > Бинарные > С ключевыми словами}
%Unary messages are sent first, then binary messages, and finally keyword messages. We also say that unary messages have a higher  priority over the other kinds of messages.
Первыми посылаются унарные сообщения, затем бинарные и, наконец, сообщения с ключевыми словами. Мы также говорим, что унарные сообещения имеют наивысший приоритет среди других типов сообщений.

%\important{\textbf{Rule One.} Unary messages are sent first, then binary messages, and finally keyword based messages.\\
\important{\textbf{Правило Первое.} Унарые сообщения посылаются первыми, затем бинарные сообщения и, наконец, сообщения с ключевыми словами.\\
\centerline{\ct{Унарные > Бинарные > С ключевыми словами}}
}

%As these examples show, \st's syntax rules generally ensure that message sends can be read in a natural way:
Как показано на примере, синтаксические правила \st гарантируют, что посылки сообщений могут быть прочитаны естественным способом:
\begin{code}{@TEST}
1000 factorial / 999 factorial --> 1000
2 raisedTo: 1 + 3 factorial     --> 128
\end{code}
\noindent

%Unfortunately the rules are a bit too simplistic for arithmetic message sends, so you need to introduce parentheses whenever you want to impose a priority over binary operators:
К сожалению, эти правила выглядят слишком упрощёнными, когда дело касается арифметических сообщений, -- вам приходится использовать скобки, чтобы наложить приоритет на бинарные операторы:
\begin{code}{@TEST}
1 + 2 * 3   --> 9
1 + (2 * 3) --> 7
\end{code}

%The following example, which is a bit more complex (!), offers a nice illustration that even complicated \st expressions can be read in a natural way:
Следующий пример (немного более сложный (!)) прекрасно иллюстрирует тот факт, что даже сложные \st выражения могут быть прочтены естественным способом:
\begin{code}{@TEST}
[:aClass | aClass methodDict keys select: [:aMethod | (aClass>>aMethod) isAbstract ]] value: Boolean --> an IdentitySet(#or: #| #and: #& #ifTrue: #ifTrue:ifFalse: #ifFalse: #not #ifFalse:ifTrue:)
\end{code}
\noindent
%Here we want to know which methods of the \ct{Boolean} class are abstract\footnote{In fact, we could also have written the equivalent but simpler expression: \ct{Boolean methodDict select: #isAbstract thenCollect: #selector}}\damien{I've added this footnote, just remove it if you don't like it :-)}.
В этом примере мы хотим узнать, какие методы класса \ct{Boolean} являются абстрактными\footnote{В действительности, мы могли бы записать это выражение проще: \ct{Boolean methodDict select: #isAbstract thenCollect: #selector}}.
%We ask some argument class, \ct{aClass}, for the keys of its method dictionary, and select those methods of that class that are abstract.
Мы запрашиваем у класса, передаваемого в качестве аргумента, список ключей словаря методов, а затем выбираем абстрактные методы.
%Then we bind the argument \ct{aClass} to the concrete value \ct{Boolean}.
Затем мы связываем аргумент \ct{aClass} с конкретным значением -- \ct{Boolean}.
%We need parentheses only to send the binary message \ct{>>}, which selects a method from a class, before sending the unary message \mbox{\ct{isAbstract}} to that method. The result shows us which methods must be implemented by \ct{Boolean}'s concrete subclasses \ct{True} and \ct{False}.
Мы нуждаемся в скобках только чтобы отправить бинарное сообщение \ct{>>}, которое получает у класса метод, перед тем, как послать этому методу унарное сообщение \mbox{\ct{isAbstract}}. Результат показывает нам, какие методы класса \ct{Boolean} должны быть реализованы в его конкретных подклассах \ct{True} и \ct{False}.

%\begin{code}{}
%Pen new go: 30 + 50          "create a turtle and moves it forward 80 pixels"
%Display restoreAfter: [WarpBlt test4] 					
%	"Keyword message, try test1, test12, test3, test4 and test 5"
%#($t $e $s $t) at: 3 --> $s 
%#($a $b $c $d) at: 2 factorial put: $z 
%\end{code}

%As you can see the syntax and in particular the keyword messages as in
%the example \ct{array at: 1 put: 4} make it possible to write code
%with a structure approaching that of natural language.
% This was one of the initial objectives so that the children can program.

%\paragraph{Example.}
\paragraph{Пример.}
%In the message \ct{aPen color: Color yellow}, there is one \emph{unary} message \ct{yellow} sent to the class \ct{Color} and a \emph{keyword} message \ct{color:} sent to \ct{aPen}. Unary messages are sent first so the message send \ct{Color yellow} is sent (1). This returns a color object which is passed as argument of the message \ct{aPen color: aColor} (2) as shown in \egref{decColor}.
В сообщении \ct{aPen color: Color yellow} присутствует только одно \emph{унарное} сообщение \ct{yellow}, посылаемое классу \ct{Color} и сообщение \emph{с ключевыми словами} \ct{color:}, отправляемую объекту \ct{aPen}. Сообщение \ct{Color yellow} унарное и посылается первым (1). Оно возвращает объект, представляющий цвет, который передаётся в качестве аргумента в сообщение \ct{aPen color: aColor} (2), как показано в \egref{decColor}.
%\figref{uKeyUn} shows graphically how messages are sent. 
\figref{uKeyUn} графически показывает, как отправляются сообщения.

\needlines{5}
%\begin{example}[decColor]{Decomposing the evaluation of \ct{aPen color: Color yellow}}{}
\begin{example}[decColor]{Декомпозиция процесса вычисления выражения \ct{aPen color: Color yellow}}{}
        aPen color: Color yellow
(1)                       Color yellow        "унарное сообщение посылается первым"
                        --> aColor
(2)   aPen color: aColor                 "затем посылается сообщение с ключевыми словами"
\end{example}

%\paragraph{Example.} In the message \ct{aPen go: 100 + 20}, there is a \emph{binary} message \ct{+ 20} and a \emph{keyword} message \ct{go:}. Binary messages are sent prior to keyword messages so \ct{100 + 20} is sent first (1): the message \ct{+ 20} is sent to the object \ct{100} and returns the number \ct{120}. Then the message \ct{aPen go: 120} is sent with \ct{120} as argument (2).
\paragraph{Пример.} В сообщении \ct{aPen go: 100 + 20} присутствует \emph{бинарное} сообщение \ct{+ 20} и сообщение \emph{с ключевыми словами} \ct{go:}. Бинарные сообщения посылаются перед сообщениями с ключевыми словами, поэтому посылка \ct{100 + 20} выполняется первой (1): сообщение \ct{+ 20} посылается числу \ct{100} и возвращает число \ct{120}. Затем посылается сообщение \ct{aPen go: 120} с числом \ct{120} в качестве аргумента (2).
%\Egref{decGo} shows how the message send is executed. 
\Egref{decGo} показывает, как выполняется посылка сообщения.

%\begin{example}[decGo]{Decomposing \ct{aPen go: 100 + 20}}{}
\begin{example}[decGo]{Декомпозиция \ct{aPen go: 100 + 20}}{}
      aPen go: 100 + 20   
(1)                 100 + 20           "сначала бинарное сообщение"
                   -->   120
(2)  aPen go: 120                   "потом сообщение с ключевыми словами"
\end{example}

\begin{figure}[htb]
\begin{minipage}{0.48\textwidth}
	\ifluluelse
		{\centerline{\includegraphics[width=0.9\textwidth]{uKeyBin}}}
		{\centerline{\includegraphics[width=6cm]{uKeyBin}}}
	%\caption{Binary messages are sent before keyword messages.\figlabel{uKeyBin}}
	\caption{Бинарные сообщения посылаются раньше сообщений с ключевыми словами.\figlabel{uKeyBin}}
\end{minipage}
\hfill
\begin{minipage}{0.48\textwidth}
	\begin{center}
	\ifluluelse
		{\includegraphics[width=0.9\textwidth]{uunKeyBin}}
		{\includegraphics[width=6cm]{uunKeyBin}}
%\caption{Decomposing \ct{Pen new go: 100 + 20}}\figlabel{unKeyBin}
\caption{Декомпозиция выражения \ct{Pen new go: 100 + 20}}\figlabel{unKeyBin}
\end{center}
\end{minipage}
\end{figure}

%\begin{figure}[ht]
%\centerline{\includegraphics[width=6cm]{uKeyBin}} 
%\caption{Unary messages are sent first so \ct{Color yellow} is sent. This returns a color object which is passed as argument of the message \ct{aPen color:}.\figlabel{uKeyBin}}
%\end{figure}

%\paragraph{Example 3.}
%The message \ct{aPen penSize: aPen penSize + 2} contains one unary message \ct{penSize}, one binary message \ct{+},  and one keyword message \ct{penSize:}.
%The unary message \ct{aPen penSize} is sent first (1), this message returns a number representing the current size of the receiver pen. Then the binary message is sent (2), the returned number is sent the message \ct{+ 2} which in its turn returns another number. Finally the keyword message 
%\ct{penSize:} is sent with the last number as argument. The expression increases the receiver pen size by two pixels. \Egref{decpen} illustrates the decomposition of message evaluation.

% penSize does not exist on Pen
%\

%\begin{scriptfigwithsize}[0.65]{\includegraphics[width=6cm]{uKeyUnBin}}{Decomposing \ct{ aPen penSize: aPen penSize + 2}}\scrlabel{decpen}
%      aPen penSize: aPen penSize + 2
%(1)                        aPen penSize            "unary"
%                              $\arrow$  aNumber
%(2)                              aNumber + 2	             "binary"
%                                    $\arrow$   anotherNumber	
%(3)   aPen penSize: anotherNumber           "keyword"      
%\end{scriptfigwithsize}

%\paragraph{Example.} As an exercise we let you decompose the evaluation of the message \ct{Pen new go: 100 + 20} which is composed of one unary, one keyword and one binary message (see \figref{unKeyBin}).
\paragraph{Пример.} В качестве упражнения, мы предлагаем вам произвести декомпозицию вычисления сообщения \ct{Pen new go: 100 + 20}, который состоит из одного унарного, одного бинарного, и одного сообщения с ключевыми словами (see \figref{unKeyBin}).
% The unary message \ct{Pen new} is first sent. It returns a new bot, then the binary message \ct{100 + 20} is sent and returns \ct{120}. Finally the message \ct{go:} is sent to the newly created robot with \ct{120}.

%\begin{figure}[ht]
%\begin{center}
%\includegraphics[width=8cm]{uunKeyBin}
%\caption{Decomposing \ct{Pen new go: 100 + 20}}\figlabel{unKeyBin}
%\end{center}
%\end{figure}

%-------------------------------------------------------------
%\subsection{Parentheses first}
\subsection{Сначала скобки}

%\important{\textbf{Rule Two.} Parenthesised messages are sent prior to other messages.\\
\important{\textbf{Правило Второе.} Сообщения в скобках посылаются раньше других сообщений.\\
%\centerline{\ct{(Msg) > Unary > Binary > Keyword}}}
\centerline{\ct{(Сообщение в скобках) > Унарное > Бинарное > С ключевыми словами}}}

\begin{code}{@TEST}
1.5 tan rounded asString = (((1.5 tan) rounded) asString) --> true    "здесь скобки не нужны"
3 + 4 factorial   --> 27    "(не 5040)"
(3 + 4) factorial --> 5040
\end{code}

%Here we need the \ind{parentheses} to force sending \ct{lowMajorScaleOn:} before \ct{play}.
Здесь нам нужны \ind{скобки}, чтобы отправить \ct{lowMajorScaleOn:} перед \ct{play}.
\begin{code}{}
(FMSound lowMajorScaleOn: FMSound clarinet) play 
"(1) отправить сообщение clarinet классу FMSound чтобы создать звук кларнета.
 (2) отправить этот звук классу FMSound как аргумент сообщения с ключевыми словами lowMajorScaleOn:.
 (3) проиграть полученный звук."
\end{code}

% ON: This has nothing to do with parentheses!
%RecordingControlsMorph new openInWorld
%"An instance of the digitizer is created then visualized. If your microphone is plugged in try a sampleBANG"

% ON: This link is broken, and the result does not understand display!
%(HTTPSocket httpShowGif:
%   'www.altavista.digital.com/av/pix/default/av-adv.gif') display

%\paragraph{Example.}
\paragraph{Пример.}
%The message \ct{(65@325 extent: 134 @ 100) center} returns the center of a rectangle whose top left point is $(65, 325)$ and whose size is $134{\times}100$. \Egref{decExtent} shows how the message is decomposed and sent. First the message between parentheses is sent: it contains two binary messages \ct{65@325} and \ct{134@100} that are sent first and return points, and a keyword message \ct{extent:} which is then sent and returns a rectangle. Finally the unary message \ct{center} is sent to the rectangle and a point is returned. 
Сообщение \ct{(65@325 extent: 134 @ 100) center} возвращает центр прямоугольника, чья левая верхняя вершина находится в точке $(65, 325)$, а размер равен $134{\times}100$. \Egref{decExtent} показывает, как производится декомпозиция и посылка этого сообщения. Сначала отправляется сообщение в скобках, которое включает в себя два бинарных сообщения \ct{65@325} и \ct{134@100}, которые посылаются первыми и возвращают объекты-точки, а также сообщение с ключевыми словами \ct{extent:}, которое посылается после них и возвращает прямоугольник. В конце прямоугольнику отправляется унарное сообщение \ct{center}, возвращающее точку.
Evaluating the message without parentheses would lead to an error because the object \ct{100} does not understand the message \ct{center}.

\needlines{9}
%\begin{example}[decExtent]{Example of Parentheses.}{}
\begin{example}[decExtent]{Пример использования скобок.}{}
      (65 @ 325 extent: 134 @ 100) center
(1)   65@325                                                    "бинарное"
    --> aPoint
(2)                                134@100                     "бинарное"
                                 --> anotherPoint
(3)   aPoint extent: anotherPoint                       "с ключевыми словами"
      --> aRectangle
(4)   aRectangle center                                     "унарное"
      --> 132@375
\end{example}

%\subsection{From left to right}
\subsection{Слева направо}
%Now we know how messages of different kinds or priorities are handled. The final question to be addressed is how messages with the same priority are sent. They are sent from the left to the right. Note that you already saw this behaviour in \egref{decExtent} where the two point creation messages (\ct{@}) were sent first.
Теперь мы знаем, как обрабатываются сообщения разных типов и приоритетов. Последний вопрос, на который нам осталось ответить, это в каком порядке посылаются сообщения, имеющие одинаковый приоритет. Отправляются они слева направо. Вспомните, вы уже встречались с этим поведением в \egref{decExtent}, где сначала посылались два сообщения, создающие объекты-точки.

%\important{{\textbf{Rule Three.} When the messages are of the same kind, the order of evaluation is from left to right.}}
\important{{\textbf{Правило Третье.} Когда посылаются сообщения одного типа, порядок вычисления определяется слева направо.}}

%\begin{figure}
%\centerline{\includegraphics[width=8cm]{ucompoUn}} 
%\caption{The message \ct{Pen new east} is composed of two unary messages. Therefore the leftmost one, \ct{new},  is sent and it returns a new robot to which the second message \ct{east} is sent. \figlabel{compoUn}}
%\end{figure}

%\paragraph{Example.} In the message sends \ct{Pen new down} all messages are unary messages, so the leftmost one, \ct{Pen new}, is sent first. This returns a newly created pen to which the second message \ct{down} is sent, as shown in \figref{unaryMessages}.
\paragraph{Пример.} В посылках \ct{Pen new down} все сообщения унарные, поэтому сначала посылается крайнее слева: \ct{Pen new}. Оно возвращает вновь созданный объект-перо, которому посылается второе сообщение \ct{down}, как показано на \figref{unaryMessages}.

\begin{figure}
	\centering
	\includegraphics[width=8cm]{ucompoUn}
	%\caption{Decomposing \ct{Pen new down}\figlabel{unaryMessages}}
	\caption{Декомпозиция посылки \ct{Pen new down}\figlabel{unaryMessages}}
\end{figure}

%-------------------------------------------------------------
%\subsection{Arithmetic inconsistencies}
\subsection{Нелогичная арифметика}
%The message composition rules are simple but they result in inconsistency for the execution of arithmetic message sends expressed in terms of binary messages. Here we see the common situations where extra parentheses are needed. 
Правила композиции сообщений очень просты, но они приводят к отличному от ожидаемого выполнению арифметических выражений, определённых через бинарные сообщения. Здесь мы видим часто встречающиеся ситуации, где нужны дополнительные скобки.

\needlines{6}
\begin{code}{@TEST}
3 + 4 * 5      --> 35    "(не 23)  Бинарные сообщения посылаются слева направо"
3 + (4 * 5)    --> 23
1 + 1/3         --> (2/3)    "и не 4/3"
1 + (1/3)       --> (4/3)
1/3 + 2/3       --> (7/9)    "и не 1"
(1/3) + (2/3)  --> 1
\end{code}

%\paragraph{Example.} 
\paragraph{Пример.} 
%In the message sends \ct{20 + 2 * 5}, there are only binary messages \ct{+} and \ct{*}. However in \st there is no specific priority for the operations \ct{+} and \ct{*}. They are just binary messages, hence \ct{*} does not have priority over \ct{+}. Here the leftmost message \ct{+} is sent first (1) and then the \ct{*} is sent to the result as shown in \egref{binaryMessages1}.  
В посылках \ct{20 + 2 * 5} только два бинарных сообщения \ct{+} и \ct{*}. Однако в \st ни для одной из этих операций не определён специфичный приоритет. Это всего лишь бинарные сообщения, поэтому \ct{*} не имеет приоритета перед \ct{+}. Сначала посылается крайнее слева сообщение \ct{+} (1), а уже его результату отправляется сообщение \ct{*}, как показано на \egref{binaryMessages1}.

%\begin{example}[binaryMessages1]{Decomposing \ct{20 + 2 * 5}}{}
\begin{example}[binaryMessages1]{Декомпозиция \ct{20 + 2 * 5}}{}
"Поскольку все бинарные сообщения имеют один приоритет, крайнее слева сообщение + выполняется первым, даже если правила арифметики гласят, что первым должно быть отправлено сообщение *."

      20 + 2 * 5 
(1)  20 + 2 --> 22
(2)  22       * 5 --> 110
\end{example}

\begin{figure}
\begin{center}\includegraphics[width=8cm]{ucompoNoBracketPar}\end{center}
\end{figure}
\noindent
%As shown in \egref{binaryMessages1} the result of this message send is not \ct{30} but \ct{110}. This result is perhaps unexpected but follows directly from the rules used to send messages. This is somehow the price to pay for the simplicity of the \st model. To get the correct result, we should use parentheses. When messages are enclosed in parentheses, they are evaluated first. Hence the message send \ct{20 + (2 * 5)} returns the result as shown in \egref{mathcorrect}.
Как показано на \egref{binaryMessages1} результат этой посылки не \ct{30}, а \ct{110}. Этот результат является, возможно, неожиданным, но он строго следует из правил посылки сообщений. Так или иначе, это цена, которую мы платим за простоту модели языка \st. Чтобы получить верный результат, следует использовать скобки, -- заключённые в них сообщения выполняются первыми. Поэтому посылка \ct{20 + (2 * 5)} возвращает результат, показанный на \egref{mathcorrect}.

\needlines{4}
%\begin{example}[mathcorrect]{Декомпозиция \ct{20 + (2 * 5)}}{}
"Сообщения, заключённые в скобки выполняются в первую очередь, поэтому * посылается перед +, что приводит к правильному результату."

    20 + (2 * 5) 
(1)        (2 * 5) --> 10
(2) 20 + 10      --> 30
\end{example}

\begin{figure}
\begin{center}
\includegraphics[width=8cm]{ucompoNumberBracket}
\end{center}
\end{figure}

%\important{In \st, arithmetic operators such as + and * do not have different priority. \ct{+} and \ct{*} are just binary messages, therefore \ct{*} does not have priority over \ct{+}. Use parentheses to obtain the desired result.}
\important{В \st, арифметические операторы, такие как + и * не имеют специфичного приоритета. \ct{+} и \ct{*} -- обычные бинарные сообщения, поэтому \ct{*} не имеет приоритета перед \ct{+}. Используйте скобки для получения желаемого результата.}

%  At the beginning put parenthesis when you have multiple binary messages.}  HUH?  At the beginning of what?!

\begin{figure}
\begin{center}
\ifluluelse
	{\includegraphics[width=\textwidth]{uKeyUnBinPar}}
	{\includegraphics[width=0.8\textwidth]{uKeyUnBinPar}}
\ifluluelse
	{\includegraphics[width=\textwidth]{uunKeyBinPar}}
	{\includegraphics[width=10cm]{uunKeyBinPar}}
\end{center}
%\caption{Equivalent messages using parentheses. \figlabel{uKeyUnBinPar}}
\caption{Эквивалентные сообщения, использующие скобки. \figlabel{uKeyUnBinPar}}
\end{figure}

%Note that the first rule stating that unary messages are sent prior to binary and keyword messages avoids the need to put explicit parentheses around them. \tabref{expressions} shows message sends written following the rules and equivalent message sends if the rules would not exist. Both message sends result in the same effect or return the same value. 
Обратите внимание, первое правило утверждающее, что унарные сообщения посылаются перед бинарными и сообщениями с ключевыми словами, делает бессмысленным заключение их в скобки. В \tabref{expressions} показаны посылки сообщений, записанные следуя этим правилам и эквивалентные им посылки, записанные так, будто бы этих правил не существует. Обе посылки приводят к одному и тому же результату.

\begin{figure}\centering
	\begin{tabular}{l@{\qquad}l}
	\toprule
	%Implicit precedence & Explicitly parenthesized equivalent \\
	Неявный приоритет & Явный эквивалент со скобками \\
	\midrule
	\lct{aPen color: Color yellow}
		& \lct{aPen color: (Color yellow)}
		\\
	\lct{aPen go: 100 + 20}
		& \lct{aPen go: (100 + 20)}
		\\
	\lct{aPen penSize: aPen penSize + 2}
		& \lct{aPen penSize: ((aPen penSize) + 2)}
		\\
	\lct{2 factorial + 4}
		& \lct{(2 factorial) + 4}
		\\
	\bottomrule
	\end{tabular}
	%\caption{Message sends and their fully parenthesized equivalents\tablabel{expressions}}
	\caption{Посылки сообщений и их эквиваленты использующие скобки\tablabel{expressions}}
\end{figure}

%=============================================================
%\section{Hints for identifying keyword messages}
\section{Советы по идентификации сообщений с ключевыми словами}
%Often beginners have problems understanding when they need to add parentheses. Let's see how keywords messages are recognized by the compiler. 
Начинающие часто имеют проблемы с пониманием того, когда нужно использовать скобки. Давайте посмотрим, как сообщения с ключевыми словами распознаются компилятором.

%-------------------------------------------------------------
%\subsection{Parentheses or not?}
\subsection{Скобки, нужны или нет?}
%The characters \ct{[}, \ct{]}, \ct{(} and \ct{)} 
Символы \ct{[}, \ct{]}, \ct{(} и \ct{)} 
%delimit distinct areas. Within such an area, a keyword message is the longest sequence of words terminated  by \ct{:} that is not cut by the characters \ct{.},  or \ct{;}. 
определяют границы различных областей. Внутри такой области, сообщение с ключевыми словами -- это последовательность наибольшей длины, состоящая из слов, оканчивающихся на \ct{:}, и не прерывающаяся символами \ct{.} или \ct{;}. 
%When the characters \ct{[}, \ct{]}, \ct{(} and \ct{)} surround some words with colons, these words participate in the keyword message \emph{local} to the area defined.
Когда символы \ct{[}, \ct{]}, \ct{(} and \ct{)} окружают слова с двоеточиями, эти слова входят в сообщение с ключевыми словами, \emph{локальное (local)} по отношению к определённой области.

%In this example, there are two distinct keyword messages:  \ct{rotatedBy:magnify:smoothing:} and \ct{at:put:}.
В следующем примере присутствуют два разных сообщения с ключевыми словами: \ct{rotatedBy:magnify:smoothing:} и \ct{at:put:}.

\begin{code}{}
aDict
   at: (rotatingForm 
          rotateBy: angle	
          magnify: 2 
          smoothing: 1)
   put: 3
\end{code}

\important{
%The characters \lct{[}, \lct{]}, \lct{(} and \lct{)} delimit distinct areas. Within such an area, 
Символы \lct{[}, \lct{]}, \lct{(} и \lct{)} определяют границы различных областей. Внутри такой области
%a keyword message is the longest sequence  of words terminated by \lct{:} that is not cut by the characters \lct{.},  or \lct{;}. 
сообщение с ключевыми словами определяется как последовательность наибольшей длины, состоящая из слов, завершающихся двоеточием, и не прерываемая символами \lct{.} или \lct{;}.
%When the characters \lct{[}, \lct{]}, \lct{(} and \lct{)} surround some words with colons, these words participate in the keyword message local to the area defined.}
Когда символы \lct{[}, \lct{]}, \lct{(} и \lct{)} окружают слова с двоеточиями, эти слова входят в сообщение с ключевыми словами, локальное для данной области.

\on{Sounds terribly complicated.}

%\paragraph{Hints.} If you have problems with these precedence rules, you may start simply by putting parentheses whenever you want to distinguish two messages having the same precedence.
\paragraph{Советы.} Если у вас возникают трудности с этими правилами, вы можете ставить скобки везде, где сочтёте необходимым для различения двух сообщений, имеющих одинаковый приоритет.

%The following piece of code does not require parentheses because the message send  \ct{x isNil} is unary hence is sent prior to the keyword message \ct{ifTrue:}.
Следующий код не требует скобок, потому что посылка \ct{x isNil} -- унарная и выполняется перед сообщением с ключевыми словами \ct{ifTrue:}.
\begin{code}{}
(x isNil)
   ifTrue:[...]
\end{code}

%The following piece of code requires parentheses because the messages \ct{includes:} and \ct{ifTrue:} are both keyword messages. 
Следующий код, наоборот, требует скобок, потому что оба сообщения \ct{includes:} и \ct{ifTrue:} являются сообщениями с ключевыми словами.
\begin{code}{}
ord := OrderedCollection new.
(ord includes: $a)
   ifTrue:[...]
\end{code}
\noindent
%Without parentheses the unknown message \ct{includes:ifTrue:} would be sent to the collection!
Не будь здесь скобок, коллекции было бы отправлено неопределённое сообщение \ct{includes:ifTrue:}!

%-------------------------------------------------------------
%\subsection{When to use \lct{[ ]} or \lct{( )}}
\subsection{Когда использовать \lct{[ ]} и когда \lct{( )}}
%You may also have problems understanding when to use square brackets rather than parentheses.
У вас также могут возникнуть проблемы с пониманием того, когда использовать квадратные скобки (brackets), а когда круглые (parentheses).
%The basic principle is that you should use \ct{[ ]} when you do not know how many times, potentially zero, an expression should be evaluated.
Базовый принцип здесь такой: используйте \ct{[ ]}, когда не знаете сколько раз будет выполнено заключённое в них выражение (возможно, ни разу).
%\lct{[\emph{expression}]} will create a \ind{block} closure (\ie an object) from \lct{\emph{expression}}, which may be evaluated any number of times (possibly zero), depending on the context. Here note that an expression can either be a message send, a variable, a literal, an assignment or a block.
\lct{[\emph{выражение}]} создаст \ind{блок}-замыкание (\ind{block} closure) (\ie некоторый объект) из \lct{\emph{выражения}}, который может быть выполнен произвольное число раз (в том числе ноль) в зависимости от контекста. 

Hence the conditional branches of \ct{ifTrue:} or \ct{ifTrue:ifFalse:} require blocks. Following the same principle both the receiver and the argument of a \ct{whileTrue:} message require the use of square brackets since we do not know how many times either the receiver or the argument should be evaluated.

Parentheses, on the other hand, only affect the order of sending messages.
So in \lct{(\emph{expression})}, the \lct{\emph{expression}} will \emph{always} be evaluated exactly once.

\begin{code}{}
[ x isReady ] whileTrue: [ y doSomething ]   "both the receiver and the argument must be blocks"
4 timesRepeat: [ Beeper beep ]                   "the argument is evaluated more than once, so must be a block"
(x isReady) ifTrue: [ y doSomething ]           "receiver is evaluated once, so is not a block"
\end{code}

%=============================================================
\section{Expression sequences}
Expressions (\ie messages sends, assignments\dots) separated by periods are evaluated in sequence.
Note that there is no period between a variable definition and the following expression.
The value of a sequence is the value of the last expression. The values returned by all the 
expressions except the last one are ignored. Note that the period is a \subind{statement}{separator} and not a terminator. Therefore a final period is optional.

\begin{code}{@TEST}
| box |
box := 20@30 corner: 60@90.
box containsPoint: 40@50 --> true
\end{code}

%=============================================================
\section{Cascaded messages}
\st offers a way to send multiple messages to the same receiver using a semicolon (\ct{;}). This is called the \emphind{cascade} in \st jargon.

\important{Expression Msg1 ; Msg2}

\begin{minipage}{0.35\textwidth}
\begin{code}{}
Transcript show: 'Pharo is '.
Transcript show: 'fun '.
Transcript cr.
\end{code}
\end{minipage}
\emph{is equivalent to:}
\begin{minipage}{0.35\textwidth}
\begin{code}{}
Transcript        
   show: 'Pharo is';
   show: 'fun ';
   cr
\end{code}
\end{minipage}

Note that the object receiving the cascaded messages can itself be the result of a message send. 
In fact the receiver of all the cascaded messages is the receiver of the first message involved in a cascade. In the following example, the first cascaded message is \ct{setX:setY} since it is followed by a cascade. The receiver of the cascaded message \ct{setX:setY:} is the newly created point resulting from the evaluation of \ct{Point new}, and \emph{not} \ct{Point}. The subsequent message \ct{isZero} is sent to that same receiver. 

\begin{code}{}
Point new setX: 25 setY: 35; isZero --> false
\end{code}

%=============================================================
\section{Chapter summary}

\begin{itemize}
\item A message is always sent to an object named the \emph{receiver} which may be the result of other message sends.

\item Unary messages are messages that do not require any argument.\\
They are of the form of \lct{receiver \textbf{selector}}.

\item Binary messages are messages that involve two objects, the receiver and another object \emph{and} whose selector is composed of  one or more characters from the following list: \ct{+}, \ct{-}, \ct{*}, \ct{/}, \ct{|}, \texttt{\&}, \ct{=}, \ct{>}, \ct{<}, \texttt{\~}, and \ct{@}.
They are of the form: \lct{receiver \textbf{selector} argument}
\item Keyword messages are messages that involve more than one object and that contain at least one colon character (\ct{:}). \\
They are of the form:
\lct{receiver \textbf{selectorWordOne:} argumentOne \textbf{wordTwo:} argumentTwo}

\item \textbf{Rule One.} Unary messages are sent first, then binary messages, and finally keyword messages.
\item \textbf{Rule Two.} Messages in parentheses are sent before any others.
\item \textbf{Rule Three.} When the messages are of the same kind, the order of evaluation is from left to right.
\item In \st, traditional arithmetic operators such as + and * have the same priority. \ct{+} and \ct{*} are just binary messages, therefore \ct{*} does not have priority over \ct{+}. You must use parentheses to obtain a different result.
\end{itemize}

%\end{document}
% ON: Don't ever put an \end{document} in a chapter
% It will make the book stop there!
%=================================================================
\ifx\wholebook\relax\else\end{document}\fi
%=================================================================

%=================================================================
%%% Local Variables:
%%% coding: utf-8
%%% mode: latex
%%% TeX-master: t
%%% TeX-PDF-mode: t
%%% ispell-local-dictionary: "english"
%%% End:

%---------------------------------------------------------
